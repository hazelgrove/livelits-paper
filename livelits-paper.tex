\PassOptionsToPackage{svgnames,dvipsnames,svgnames}{xcolor}
\documentclass[acmsmall,review,anonymous,nonacm]{acmart}
\settopmatter{printfolios=true,printccs=false,printacmref=false}
%% For double-blind review submission, w/ CCS and ACM Reference
%\documentclass[acmsmall,review,anonymous]{acmart}\settopmatter{printfolios=true}
%% For single-blind review submission, w/o CCS and ACM Reference (max submission space)
%\documentclass[acmsmall,review]{acmart}\settopmatter{printfolios=true,printccs=false,printacmref=false}
%% For single-blind review submission, w/ CCS and ACM Reference
%\documentclass[acmsmall,review]{acmart}\settopmatter{printfolios=true}
%% For final camera-ready submission, w/ required CCS and ACM Reference
%\documentclass[acmsmall]{acmart}\settopmatter{}

%% Conference information
%% Supplied to authors by publisher for camera-ready submission;
%% use defaults for review submission.
\acmConference[PL'18]{ACM SIGPLAN Conference on Programming Languages}{January 01--03, 2018}{New York, NY, USA}
\acmYear{2018}
\acmISBN{} % \acmISBN{978-x-xxxx-xxxx-x/YY/MM}
\acmDOI{} % \acmDOI{10.1145/nnnnnnn.nnnnnnn}
\startPage{1}

%% Copyright information
%% Supplied to authors (based on authors' rights management selection;
%% see authors.acm.org) by publisher for camera-ready submission;
%% use 'none' for review submission.
\setcopyright{none}
%\setcopyright{acmcopyright}
%\setcopyright{acmlicensed}
%\setcopyright{rightsretained}
%\copyrightyear{2018}           %% If different from \acmYear

%% Bibliography style
% \bibliographystyle{ACM-Reference-Format}
%% Citation style
%\citestyle{acmauthoryear}  %% For author/year citations
%\citestyle{acmnumeric}     %% For numeric citations
%\setcitestyle{nosort}      %% With 'acmnumeric', to disable automatic
                            %% sorting of references within a single citation;
                            %% e.g., \cite{Smith99,Carpenter05,Baker12}
                            %% rendered as [14,5,2] rather than [2,5,14].
%\setcitesyle{nocompress}   %% With 'acmnumeric', to disable automatic
                            %% compression of sequential references within a
                            %% single citation;
                            %% e.g., \cite{Baker12,Baker14,Baker16}
                            %% rendered as [2,3,4] rather than [2-4].

%% Some recommended packages.
\usepackage{booktabs}   %% For formal tables:
                        %% http://ctan.org/pkg/booktabs
\usepackage{subcaption} %% For complex figures with subfigures/subcaptions
                        %% http://ctan.org/pkg/subcaption

%% Cyrus packages
\usepackage{microtype}
\usepackage{mdframed}
\usepackage{colortab}
\usepackage{mathpartir}
\usepackage{enumitem}
\usepackage{bbm}
\usepackage{stmaryrd}
\usepackage{mathtools}
\usepackage{leftidx}
\usepackage{todonotes}
\usepackage{xspace}
\usepackage{wrapfig}

\newcommand{\cyrus}[1]{{\color{blue} #1}}

\usepackage{listings}%
\lstloadlanguages{ML}
\lstset{tabsize=2, 
basicstyle=\footnotesize\ttfamily, 
% keywordstyle=\sffamily,
commentstyle=\itshape\ttfamily\color{gray}, 
stringstyle=\ttfamily\color{purple},
mathescape=false,escapechar=\#,
numbers=left, numberstyle=\scriptsize\color{gray}\ttfamily, language=ML, showspaces=false,showstringspaces=false,xleftmargin=15pt, 
morekeywords={string, float, int, Int, Float, String, livelit},
classoffset=0,belowskip=\smallskipamount, aboveskip=\smallskipamount,
moredelim=**[is][\color{red}]{SSTR}{ESTR}
}
\newcommand{\li}[1]{\lstinline[basicstyle=\ttfamily\fontsize{9pt}{1em}\selectfont]{#1}}
\newcommand{\lismall}[1]{\lstinline[basicstyle=\ttfamily\fontsize{9pt}{1em}\selectfont]{#1}}

%% Joshua Dunfield macros
\def\OPTIONConf{1}%
\usepackage{joshuadunfield}

%% Can remove this eventually
\usepackage{blindtext}

\usepackage{enumitem}

%%%%%%%%%%%%%%%%%%%%%%%%%%%%%%%%%%%%%%%%%%%%%%%%%%%%%%%%%%%%%%%%%%%%%%%%%%%%%
%% Matt says: Cyrus, this package `adjustbox` seems directly related
%% to the `clipbox` error; To get rid of the error, I moved it last
%% (after other usepackages) and I added the line just above it, which
%% permits it to redefine `clipbox` (apparently also defined in
%% `pstricks`, and due to latex's complete lack of namespace
%% management, these would otherwise names clash).
\let\clipbox\relax
\usepackage[export]{adjustbox}% http://ctan.org/pkg/adjustbox
%%%%%%%%%%%%%%%%%%%%%%%%%%%%%%%%%%%%%%%%%%%%%%%%%%%%%%%%%%%%%%%%%%%%%%%%%%%%%%%%%


%%%%%%%%%%%%%%%%%%%%%%%%%%%%%%%%%%%%%%%%%%%%%%%%%%%%%%%%%%%%%%%%%%%%%%%%%%%%%%%%%
%\usepackage{draftwatermark}
%\SetWatermarkText{DRAFT}
%\SetWatermarkScale{1}
%%%%%%%%%%%%%%%%%%%%%%%%%%%%%%%%%%%%%%%%%%%%%%%%%%%%%%%%%%%%%%%%%%%%%%%%%%%%%%%%%


% A macro for the name of the system being described by ``this paper''
\newcommand{\HazelnutLive}{\textsf{Hazelnut Live}\xspace}
\newcommand{\Hazelnut}{\textsf{Hazelnut}\xspace}
% The mockup, work-in-progress system.
\newcommand{\Hazel}{\textsf{Hazel}\xspace}

% \newtheorem{theorem}{Theorem}[chapter]
% \newtheorem{lemma}[theorem]{Lemma}
% \newtheorem{corollary}[theorem]{Corollary}
% \newtheorem{definition}[theorem]{Definition}
% \newtheorem{assumption}[theorem]{Assumption}
% \newtheorem{condition}[theorem]{Condition}

\newtheoremstyle{slplain}% name
  {.15\baselineskip\@plus.1\baselineskip\@minus.1\baselineskip}% Space above
  {.15\baselineskip\@plus.1\baselineskip\@minus.1\baselineskip}% Space below
  {\slshape}% Body font
  {\parindent}%Indent amount (empty = no indent, \parindent = para indent)
  {\bfseries}%  Thm head font
  {.}%       Punctuation after thm head
  { }%      Space after thm head: " " = normal interword space;
        %       \newline = linebreak
  {}%       Thm head spec
\theoremstyle{slplain}
\newtheorem{thm}{Theorem}  % Numbered with the equation counter
\numberwithin{thm}{section}
\newtheorem{defn}[thm]{Definition}
\newtheorem{lem}[thm]{Lemma}
\newtheorem{prop}[thm]{Proposition}
% \newtheorem{cor}[section]{Corollary}     
% \newtheorem{lem}[section]{Lemma}         
% \newtheorem{prop}[section]{Proposition}  

% \setlength{\abovedisplayskip}{0pt}
% \setlength{\belowdisplayskip}{0pt}
% \setlength{\abovedisplayshortskip}{0pt}
% \setlength{\belowdisplayshortskip}{0pt}



\makeatletter\if@ACM@journal\makeatother
%% Journal information (used by PACMPL format)
%% Supplied to authors by publisher for camera-ready submission
\acmJournal{PACMPL}
\acmVolume{1}
\acmNumber{1}
\acmArticle{1}
\acmYear{2018}
\acmMonth{3}
\acmDOI{10.1145/nnnnnnn.nnnnnnn}
\startPage{1}
\else\makeatother
%% Conference information (used by SIGPLAN proceedings format)
%% Supplied to authors by publisher for camera-ready submission
% \acmConference[]{ACM SIGPLAN Conference on Programming Languages}{January 01--03, 2017}{New York, NY, USA}

\acmYear{2018}
\acmISBN{978-x-xxxx-xxxx-x/YY/MM}
\acmDOI{10.1145/nnnnnnn.nnnnnnn}
\startPage{1}
\fi


%% Copyright information
%% Supplied to authors (based on authors' rights management selection;
%% see authors.acm.org) by publisher for camera-ready submission
\setcopyright{none}             %% For review submission
%\setcopyright{acmcopyright}
%\setcopyright{acmlicensed}
%\setcopyright{rightsretained}
%\copyrightyear{2017}           %% If different from \acmYear


\fancyfoot{} % suppresses the footer (also need \thispagestyle{empty} after \maketitle below)


%% Bibliography style
\bibliographystyle{ACM-Reference-Format}
%% Citation style
%% Note: author/year citations are required for papers published as an
%% issue of PACMPL.
\citestyle{acmauthoryear}   %% For author/year citations

\input{commands}
\input{macros}

\setlength{\abovecaptionskip}{4pt plus 3pt minus 2pt} % Chosen fairly arbitrarily
\setlength{\belowcaptionskip}{-4pt plus 3pt minus 2pt} % Chosen fairly arbitrarily


\begin{document}

%% Title information
\title
  [Filling Typed Holes with Live GUIs]
  {Filling Typed Holes with Live GUIs}
  %% [Programming by Direct Manipulation of Palettes]
  %% {Functional Programming by\\Direct Manipulation of Live Palette Expressions}

                                        %% when present, will be used in
                                        %% header instead of Full Title.
% \titlenote{with title note}             %% \titlenote is optional;
                                        %% can be repeated if necessary;
                                        %% contents suppressed with 'anonymous'
% \subtitle{Subtitle}                     %% \subtitle is optional
% \subtitlenote{with subtitle note}       %% \subtitlenote is optional;
                                        %% can be repeated if necessary;
                                        %% contents suppressed with 'anonymous'


%% Author information
%% Contents and number of authors suppressed with 'anonymous'.
%% Each author should be introduced by \author, followed by
%% \authornote (optional), \orcid (optional), \affiliation, and
%% \email.
%% An author may have multiple affiliations and/or emails; repeat the
%% appropriate command.
%% Many elements are not rendered, but should be provided for metadata
%% extraction tools.

%% Author with single affiliation.
\author{Cyrus Omar}
% \authornote{with author1 note}          %% \authornote is optional;
                                        %% can be repeated if necessary
% \orcid{nnnn-nnnn-nnnn-nnnn}             %% \orcid is optional
\affiliation{
  % \position{Position1}
  % \department{Department1}              %% \department is recommended
  \institution{University of Michigan}            %% \institution is required
  % \streetaddress{Street1 Address1}
  % \city{City1}
  % \state{State1}
  % \postcode{Post-Code1}
  % \country{Country1}
}
\email{comar@cs.uchicago.edu}          %% \email is recommended

\author{Nick Collins}
% \authornote{with author1 note}          %% \authornote is optional;
                                        %% can be repeated if necessary
% \orcid{nnnn-nnnn-nnnn-nnnn}             %% \orcid is optional
\affiliation{
  % \position{Position1}
  % \department{Department1}              %% \department is recommended
  \institution{University of Chicago}            %% \institution is required
  % \streetaddress{Street1 Address1}
  % \city{City1}
  % \state{State1}
  % \postcode{Post-Code1}
  % \country{Country1}
}
\email{nickmc@uchicago.edu}          %% \email is recommended

\author{David Moon}
% \authornote{with author1 note}          %% \authornote is optional;
                                        %% can be repeated if necessary
% \orcid{nnnn-nnnn-nnnn-nnnn}             %% \orcid is optional
\affiliation{
  % \position{Position1}
  % \department{Department1}              %% \department is recommended
  \institution{University of Michigan}            %% \institution is required
  % \streetaddress{Street1 Address1}
  % \city{City1}
  % \state{State1}
  % \postcode{Post-Code1}
  % \country{Country1}
}
\email{dmoo@umich.edu}          %% \email is recommended


\author{Ian Voysey}
% \authornote{with author1 note}          %% \authornote is optional;
                                        %% can be repeated if necessary
% \orcid{nnnn-nnnn-nnnn-nnnn}             %% \orcid is optional
\affiliation{
  % \position{Position1}
  % \department{Department1}              %% \department is recommended
  \institution{Carnegie Mellon University}            %% \institution is required
  % \streetaddress{Street1 Address1}
  % \city{City1}
  % \state{State1}
  % \postcode{Post-Code1}
  % \country{Country1}
}
\email{iev@cs.cmu.edu}          %% \email is recommended

\author{Ravi Chugh}
% \authornote{with author1 note}          %% \authornote is optional;
                                        %% can be repeated if necessary
% \orcid{nnnn-nnnn-nnnn-nnnn}             %% \orcid is optional
\affiliation{
  % \position{Position1}
  % \department{Department1}              %% \department is recommended
  \institution{University of Chicago}            %% \institution is required
  % \streetaddress{Street1 Address1}
  % \city{City1}
  % \state{State1}
  % \postcode{Post-Code1}
  % \country{Country1}
}
\email{rchugh@cs.uchicago.edu}          %% \email is recommended

\author{Ravi Chugh}
% \authornote{with author1 note}          %% \authornote is optional;
                                        %% can be repeated if necessary
% \orcid{nnnn-nnnn-nnnn-nnnn}             %% \orcid is optional
\affiliation{
  % \position{Position1}
  % \department{Department1}              %% \department is recommended
  \institution{University of Chicago}            %% \institution is required
  % \streetaddress{Street1 Address1}
  % \city{City1}
  % \state{State1}
  % \postcode{Post-Code1}
  % \country{Country1}
}
\email{rchugh@cs.uchicago.edu}          %% \email is recommended

% %% Author with two affiliations and emails.
% \author{First2 Last2}
% \authornote{with author2 note}          %% \authornote is optional;
%                                         %% can be repeated if necessary
% \orcid{nnnn-nnnn-nnnn-nnnn}             %% \orcid is optional
% \affiliation{
%   \position{Position2a}
%   \department{Department2a}             %% \department is recommended
%   \institution{Institution2a}           %% \institution is required
%   \streetaddress{Street2a Address2a}
%   \city{City2a}
%   \state{State2a}
%   \postcode{Post-Code2a}
%   \country{Country2a}
% }
% \email{first2.last2@inst2a.com}         %% \email is recommended
% \affiliation{
%   \position{Position2b}
%   \department{Department2b}             %% \department is recommended
%   \institution{Institution2b}           %% \institution is required
%   \streetaddress{Street3b Address2b}
%   \city{City2b}
%   \state{State2b}
%   \postcode{Post-Code2b}
%   \country{Country2b}
% }
% \email{first2.last2@inst2b.org}         %% \email is recommended


%% Paper note
%% The \thanks command may be used to create a "paper note" ---
%% similar to a title note or an author note, but not explicitly
%% associated with a particular element.  It will appear immediately
%% above the permission/copyright statement.
% \thanks{with paper note}                %% \thanks is optional
                                        %% can be repeated if necesary
                                        %% contents suppressed with 'anonymous'


%% Abstract
%% Note: \begin{abstract}...\end{abstract} environment must come
%% before \maketitle command
% !TEX root = livelits-paper.tex

%% no citations in Abstract
%%
%% \cite{popl-paper}

\begin{abstract}
    Text editing is powerful, but 
    some types of expressions are more naturally represented and manipulated
    graphically.
    Examples include expressions that compute 
    colors,
    music,
    animations,
    tabular data, 
    plots,
    GUI widgets, 
    mathematical diagrams, 
    and other domain-specific data structures.
    This paper introduces \emph{live literals}, or \emph{livelits}, 
    which allow
    clients to fill holes of types like these   
    by directly manipulating a     
    user-defined GUI embedded persistently into code. 
    Uniquely, livelits are \emph{compositional}: a livelit GUI can itself embed 
    spliced expressions, which are typed, lexically scoped, enjoy full editor support and can in turn embed 
    other livelits. 
    Livelits are also uniquely \emph{live}: 
    a livelit can provide {continuous feedback} about the run-time implications 
    of the client's choices 
    even when splices mention bound variables, 
    because the system continuously gathers closures associated with the 
    hole that the livelit is tasked with filling.
    We integrate livelits into Hazel, 
    a live programming environment able to 
    typecheck and run programs with holes, 
    and describe several case studies that exercise these novel capabilities. 
    We then define a minimal typed livelit calculus, which precisely 
    specifies how livelits operate as 
    live graphical macros.     
    The metatheory of macro expansion has been mechanized in Agda.
\end{abstract}



%% 2012 ACM Computing Classification System (CSS) concepts
%% Generate at 'http://dl.acm.org/ccs/ccs.cfm'.
% \begin{CCSXML}
% <ccs2012>
% <concept>
% <concept_id>10011007.10011006.10011008</concept_id>
% <concept_desc>Software and its engineering~General programming languages</concept_desc>
% <concept_significance>500</concept_significance>
% </concept>
% <concept>
% <concept_id>10003456.10003457.10003521.10003525</concept_id>
% <concept_desc>Social and professional topics~History of programming languages</concept_desc>
% <concept_significance>300</concept_significance>
% </concept>
% </ccs2012>
% \end{CCSXML}

% \ccsdesc[500]{Software and its engineering~General programming languages}
% \ccsdesc[300]{Social and professional topics~History of programming languages}
%% End of generated code


%% Keywords
%% comma separated list
% \keywords{keyword1, keyword2, keyword3}  %% \keywords is optional
% \keywords{Live Programming, Palettes, Hazel, Code Editors}


%% \maketitle
%% Note: \maketitle command must come after title commands, author
%% commands, abstract environment, Computing Classification System
%% environment and commands, and keywords command.
\maketitle
% \thispagestyle{empty} % suppresses the footer

% \section{TODO List}

% \begin{itemize}

% \item[\cmark] palette macros
% \item action semantics
% \item \Hazel{} in \Hazel{} (sums, productions, recursive types)
% \item palette functions
% \item live palettes
% \item synthetic (or fancy record-type-level-computation) (or not-fancy untyped) palettes
% \item parameterized palettes
% \item palette-specific actions
% \item option to render palette macro expansion as expression or value
% \item \Hazel{} UI: ``formula bar'' for expressions
% \item \sns{} implementation
% \item examples!

% \end{itemize}

\section{Introduction}\label{sec:intro}
Text-based program editors are flexible and expressive user interfaces
so it is little wonder that they remain dominant decades after the teletype.
However, textual user interfaces are not the best tool for every computational job.
% In particular, there are countless 
% data types for which a non-textual
% user interface may situationally be more appropriate.

As a simple example, consider a record type
classifying RGBA-encoded colors. 
It is possible to select a particular color by entering
an expression of this type in a text editor, e.g. \li{\{ r: 255, g: 178, b: 45, a: 100 \}}. 
The problem with this textual user interface for color selection is that 
it offers no live feedback about which color has been selected 
and limited editing affordances for tweaking the selected color.
Analagous critiques apply to strictly textual user interfaces for 
countless other data structures,
such as vector graphics,
animation parameters,
musical sequences,
audio filters,
board game states, 
GUI widgets and layouts,
tabular data, 
plots,
geospatial data, 
neural network diagrams, 
biological neuron models, 
mathematical diagrams, 
and so on.

% It is difficult for the programmer, or anyone subsequently reading or modifying the code, to know which color is represented
% and to interactively tweak that color.

Practitioners in domains where manipulating data of types like these is 
a central activity 
have largely eschewed general-purpose programming environments 
in favor of more specialized graphical end-user applications, like %
image and video editors, music composition software, level design tools, 
and bespoke GUIs written by students or lab technicians, 
in large part because these applications 
take seriously the need for domain-specific forms of live feedback, 
graphical data representations, and 
direct manipulation affordances, 
e.g. color palettes, visual timelines, interactive plots, and maps.

The tragedy is that these applications have 
limited support for abstraction and composition.
It is difficult, for example, to bind a
color to a variable for use in multiple locations in an
otherwise directly constructed game map,
or to define a function that computes portions of an 
otherwise directly constructed vector graphic,
or to transform a directly constructed musical sequence 
by passing it through a series of symbolically defined functions.
Moreover, it is difficult to add new affordances or to compose
affordances in ways that the application developer did not anticipate.
Users cannot easily make even simple changes like replacing a numeric text box in a dialog with a slider,
much less more ambitious changes like installing an alternative visual interface for expressing geospatial data queries 
into a database frontend.

% Better support for manipulating data of types like these would be particularly helpful for users engaging in
% live and exploratory programming in domains like web design, media production,
% and data analysis. Indeed, 

This paper aims to resolve this tension between
programmatic and direct manipulation user interfaces by designing 
a programming environment that
is able to surface GUIs when working with types for which
they are useful, while retaining full support for symbolic program manipulation
and the abstraction and composition mechanisms
available in modern general-purpose programming languages.

\subsection{Background}\label{sec:background}
\definecolor{mygray}{rgb}{0.93, 0.93, 0.93}
\definecolor{shadecolor}{named}{mygray}

\begin{figure*}
  \begin{minipage}[t]{0.37\textwidth}
    \begin{subfigure}[t]{\linewidth}
    \begin{snugshade}
      \vspace*{-2mm}
      \caption{\textbf{Prior Work:} Graphite \cite{Graphite}}
      \vspace*{1mm}
     \end{snugshade}
      \vspace*{-1mm}
      \includegraphics[width=\linewidth]{graphite-color-palette-green.png}
      \vspace*{-5mm}
    \end{subfigure}
    \hspace{8mm}
    \begin{subfigure}[t]{\linewidth}
     \begin{snugshade}
      \vspace*{-2mm}
      \caption{\textbf{This Paper:} Livelits are live and compositional}
      \vspace*{1mm}
     \end{snugshade}
      \vspace*{-1mm}
      \includegraphics[width=\linewidth]{slider-color-livelits.png}
    \end{subfigure}
  \end{minipage}
  \hspace{3mm}
  \begin{subfigure}[t]{0.50\textwidth}
  \begin{snugshade}
   \vspace*{-2mm}
    \caption{\textbf{Case Study}: Grading with Livelits}
    \vspace*{1mm}
     \end{snugshade}
    \vspace*{-1mm}
    \includegraphics[width=\linewidth]{grade-cutoff-livelit.png}
  \end{subfigure}
   \vspace*{-7mm}
   \caption{}
   \vspace*{-2mm}
   \label{fig:color}
\end{figure*}

Of course, we are not the first to integrate direct manipulation interfaces
into symbolic programming environments.
% Prior work on projectional editing
% and active code completion, detailed in Sec. \ref{sec:related-work},
% has also considered the problem of entering expressions
% of certain types, like \li{Color},
% using specialized GUIs integrated into a program editor.
% We detail prior work in Sec.~\ref{sec:related-work}, but 
The prior work most relevant to this paper is the {Graphite} system for Eclipse for Java,
demonstrated in Fig.~\ref{fig:color}(a) \cite{Graphite}.
Graphite allows a library provider to associate a GUI, called a \emph{palette}, with a type 
(via a Java class annotation).
Wherever an expression of this type is needed,
i.e. wherever there is a \emph{hole} of that type in the program
(as determined by Eclipse's online parser and typechecker),
the environment offers the client the option, via the code completion menu,
to interact with the palette.
Once the interaction is finished, the palette generates a
Java expression to fill the hole.
Figure~\ref{fig:color}(a), adapted from this prior work, demonstrates a color palette invoked using Graphite.
When the user presses the \li{Enter} key, the Java expression \li{new Color(173, 173, 173, 85)} is inserted at the cursor and the palette disappears.
Several related systems, such as the  
\textbf{mage} system for the Jupyter notebook environment \cite{DBLP:conf/uist/KeryRHMWP20}
and the interactive visual syntax system for Racket \cite{interactive-visual-syntax}, behave fundamentally similarly.
asdfasdfa
fasdfasdf
asdfasd\todo{placeholder for future text that might be added}

\citet{Graphite} evaluated Graphite by surveying 473 developers 
and \citet{DBLP:conf/uist/KeryRHMWP20} evaluated \textbf{mage} by interviewing 9 developers.
Both studies found that
participants viewed the proposed mechanism favorably and 
would use a suitable GUI some or all of the time.
% \footnote{When presented with a color palette,
% many participants remarked that they rarely entered colors directly into Java code,
% but rather into stylesheets.
% Other palettes, e.g. a palette
% that supported regular expression construction, were viewed as more
% suitable for Java code.
% The mechanisms being considered are suitable both for general-purpose languages
% and typed domain-specific languages like typed stylesheets, which can often be embedded into modern
% general-purpose languages.}
This and other prior work also collectively showcase a wide variety of use cases \cite{Graphite,DBLP:conf/uist/KeryRHMWP20,interactive-visual-syntax}\todo{more cites}, 
and the Graphite survey solicited dozens of additional use cases from participants,
which the authors systematically taxonomize \cite{Graphite}. 
We take these extensive empirical findings   
as evidence for, and a showcase of, 
the value of this class of mechanisms for integrating GUIs into code.

\subsection{Contributions}

We turn our attention in this paper to several fundamental technical
deficiencies that limit GUI providers and clients using these prior systems.
To address these, 
we introduce a system of \emph{live literals}, or \emph{livelits}, 
demonstrated in Fig.~\ref{fig:color}(b). 
Livelits are unique in achieving all of the following properties.
(We describe which subset of these properties 
are achieved by prior systems,
including those just mentioned, 
in Sec.~\ref{sec:related-work}.)

\newcommand{\llproperty}[1]{\vspace{5px}\noindent\textbf{#1}.}

\llproperty{Decentralized Extensibility}
    Providers define livelits in libraries, and 
    clients invoke livelits by name. Livelit names, e.g. \li{\$color},  are prefixed by \li{\$},    
    to distinguish them from variables.
    % We call this \emph{decentralized extensibility}.
    % Both mage and Racket's visual syntax system are similarly extensible, 
    % but 
    % In Graphite, palettes are associated with class definitions, 
    % so there is a tighter association between 
    % Other systems, discussed in Sec.~\ref{sec:related-work}, 
    % are either non-extensible, or cannot be extended by library providers 
    % (only by, for example, separately installed editor extensions, or by 
    % regenerating the editor using a language workbench.)
 
\llproperty{Persistence}
  % In Graphite and \textbf{mage}, GUIs are {ephemeral},
  % i.e. they disappear after the initial interaction,
  % leaving behind only the generated textual code.
  % Only the programmer that initially enters the expression
  % benefits from the feedback and affordances that the GUI provides.
  % %
  % \footnote{Graphite does include an \emph{ad hoc} mechanism that
  % allows palettes to parse the code that is selected in the editor
  % when the palette loads, but this requires that each palette implement
  % a parser for the subset of Java used in the code that it generates,
  % and therefore this mechanism is quite brittle. It is also difficult
  % to persist GUI state that is not included in the generated code.}
  Livelits are persistent elements of the syntax tree. They operate as  
  graphical literals, rather than as the ephemeral code generation GUIs of Graphite and \textbf{mage}. 
  We define a pure model-view-update-expand architecture
  (a variation on Elm's model-view-update architecture \cite{ElmArchitecture}) 
  where only the model needs to be persisted.
  The dynamic meaning of a livelit is determined by macro expansion.
  % We chose the word ``literal'' rather than ``palette'' because,  persistence, livelits
  % operate as graphical literals.

\llproperty{Hygienic Composition}
% Prior systems have limited or no support for {entering sub-expressions within the GUI}, 
% so they are useful mainly for generating expressions composed of constants,
% e.g. color constants in Fig.~\ref{fig:color}(a).
%
  Livelits support sub-expressions directly in the GUI, which we call \emph{splices} (after
   \citet{TLMs}).
  Fig.~\ref{fig:color}(b) demonstrates splicing: the RGBA components  
  are splice editors, so the client can define a variable, \li{baseline},
  to relate the color components\todo{tweak figure}{} 
  and use a slider livelit inline to specify the alpha component.
  % Similarly, Fig.~\ref{fig:grading}(b)\todo{subfigure labels}{} demonstrates a 
  %  data table livelit where each entry is a splice.
  
  Crucially, composition is strictly 
  governed by a hygiene discipline that ensures
  (1) \textbf{capture avoidance}, i.e. that variables that the client uses in splices 
  will not capture expansion-internal bindings; and 
  (2) \textbf{context independence}, i.e. that the livelit 
  can be invoked in any program context.%without pre-condition or conflict.

\llproperty{Parameterization} Livelits can form parameterized families.
  For example, \li{\$slider} in Fig.~\ref{fig:color}(b) is parameterized by the slider's bounds.
  Parameters operate like splices, differing in that they can be partially applied in
  livelit abbreviations. For example, Fig.~\ref{fig:color}(b)\todo{add slider}{} 
  partially applies \li{\$slider} to \li{0} and \li{100} to define a \li{\$percentage} slider\todo{todo}{}.
  % \begin{lstlisting}[numbers=none]
  % let $percentage = $slider 0 100 in ...
  % \end{lstlisting}
  % Parameterization also underlies the hygiene mechanism, as we will discuss in Sec.~\ref{sec:livelit-definitions}.

\llproperty{Typing} Each livelit specifies the type of expansions it generates, 
and parameters and splices also specify types, 
so livelits are compatible with type-driven methodologies and tools.
Together with the hygiene discipline, this allow clients to reason abstractly about expansions, i.e.  
without inspecting the expansions or livelit implementations directly.

\llproperty{Liveness} Uniquely, livelits can evaluate splices 
  throughout the editing process 
  (i.e. in a \emph{live} manner \cite{DBLP:conf/icse/Tanimoto13}) 
  to provide feedback related to run-time behavior.
  For example, in Fig.~\ref{fig:color}(b), 
  displaying the selected color requires evaluating the RGBA
  component splices to numeric values.
  Evaluation occurs in a run-time environment (i.e. closure) determined by
  leaving the hole being filled by the livelit temporarily unfilled and then evaluating
  using a two-phased variant of the semantics for 
  live programming with typed holes developed by \citet{HazelnutLive}.
  We support live evaluation even for livelits 
  that appear inside a function. Multiple function calls lead 
  to multiple closures that the client 
   can select between.

\paragraph{Outline.} We begin in Sec.~\ref{sec:case-studies} by introducing
livelits from the perspective of client programmers. 
Our examples are organized into case studies 
and are chosen to demonstrate the novel contributions of this paper.
In Sec.~\ref{sec:livelit-definitions}, 
we consider the livelit provider's perspective by introducing livelit
definitions with a detailed example.
In Sec.~\ref{sec:livelit-calculus}, we define the \emph{typed livelit calculus}. 
We have mechanically specified the central mechanism, livelit expansion, and proven the associated metatheorems in Agda.
This calculus 
serves to capture the essential nature of livelits 
independent of the particularities of syntax, GUI frameworks, 
and other orthogonal design details,
because we believe livelits can be integrated into a wide variety of programming systems. 
In Sec.~\ref{sec:implementation}, we provide a more detailed account of our two implementations of livelits.
Our primary implementation, used in the screenshots in the paper,
is integrated into Hazel, a live programming environment designed 
around hole-driven development. 
We have also prototyped livelits within a standard text editor. 
Additionally, we discuss factors that must be considered when integrating livelits into languages with side effects.
In Sec.~\ref{sec:related-work}, we compare livelits to related work using the design properties outlined above 
as a rubric.
Finally, we conclude in Sec.~\ref{sec:discussion} after a discussion of present limitations and future work.
\section{Livelits by Example}\label{sec:case-studies}


In this section, we will detail the livelits mechanism by way of  
two domain-specific case studies:
a course grade assignment case study in Sec.~\ref{sec:live-grading}
and an image transformation case study in Sec.~\ref{sec:image-transformation}.
% We also briefly mention several other examples in Sec.~\ref{sec:additional-examples}.\todo{do we?}{}
These case studies have been implemented
in Hazel, a browser-based live programming environment for a dialect of Elm. 
Elm is an industrial pure typed functional language in the
ML family used for client-side web development. 
We assume basic familiarity with ML.

\subsection{Case Study: Grading with Livelits}\label{sec:live-grading}
Consider this familiar scenario: an instructor needs
(1) to record numeric grades for various assignments and exams, and
(2) to visualize and perform various computations with these numeric grades
in order ultimately to assign final letter grades.
(In fact, this case study is not contrived: one author is using Hazel to compute grades this semester.)

The most common end-user application for this task is the spreadsheet, because 
it allows the instructor to record grades using a natural tabular interface,
visualize this data in one of a finite number of plot styles, 
and perform basic computations,
with results updated live.
However, these affordances are limited. 
It is difficult to package up common operations 
into reusable libraries, interact with the data using domain-specific visualizations,
and perform complex, unanticipated operations 
(e.g. preparing the data in an idiosyncratic format demanded by the university's grading system).

General-purpose programming languages 
can handle these scenarios, but users 
lose the ability to receive live feedback and  
directly manipulate data and visualizations in the editor.

Livelits are able to address this tension.
Fig.~\ref{fig:grading}(c)\todo{sublabels} shows a Hazel program where 
the instructor alternates between programmatic and direct manipulation in several situations.

First, the instructor defines a value \li{grades} 
that records the grades for each student using a livelit, \li{\$dataframe}, 
that implements a tabular user interface. The formula bar 
allows the selected cell to be filled with an arbitrary Hazel expression. 
For the sake of demonstration, we show a cell that has been filled using another livelit, 
\li{\$slider}, in combination with symbolic manipulation.\todo{do this?}{}
The table itself displays not the expression itself but rather its value, just as in a spreadsheet.

Next, the instructor computes \li{averages} 
for each student by applying \li{compute_averages}, a helper function 
defined in a library  (not shown) shared between multiple courses.

Next, the instructor wants to ``eyeball'' reasonable \li{cutoffs} between letter grades 
by directly manipulating a highly domain-specific livelit, \li{\$grade_cutoffs}, that provides draggable ``paddles'' 
superimposed on a live visualization of the distribution of the \li{averages} value provided as a parameter 
(for the sake of space, we plot dots rather than a histogram).
% The value of \li{cutoffs} is a labeled 4-tuple containing each cut-off. 

Finally, 
the instructor programmatically assigns grades to students 
based on these \li{cutoffs} 
by calling \li{assign_grades} 
and \li{format_for_university},
again shared functions.

\subsection{Livelit Expansion}\label{sec:livelit-expansion}
Livelit invocations 
expand to expressions.
For example, the expansion of Fig.~\ref{fig:grading}(c) is:

\begin{lstlisting}[xleftmargin=0.2cm]
let grades = Dataframe (
  ["A1", "A2", "A3", "Midterm", "Final"],
  [("Alice", [24. +. 36. +. 33., 
              92., 83.5, 95., 88.]),
   ("Bob", [61., 64., 98., 70., 85.]),
   ("Ciri", [75., 81., 73., 82., 79.]),
   (* ... *) ]) in
let averages = compute_averages grades weights in
let cutoffs = (.A 90., .B 80., .C 70., .D 60.) in
format_for_university 
  (assign_grades averages cutoffs)
\end{lstlisting}

The client can inspect this expansion in Hazel via a toggle (not shown).
Ideally, however, reasoning about types and binding
should not require the client to inspect the expansion 
nor the livelit implementation (which specifies the expansion logic 
as we will describe in Sec.~\ref{sec:livelit-definitions}).
After all, function clients do not need to look inside
function bodies to reason about types and binding.
Instead, in the words of \citet{DBLP:conf/ifip/Reynolds83},
``type structure is a syntactic discipline for maintaining levels of abstraction''.
Livelits maintain this discipline by
several means, described next in Sec.~\ref{sec:expansion-typing}-\ref{sec:hygiene}.

\subsubsection{Expansion Typing} 
\label{sec:expansion-typing}
To support abstract reasoning about the type of the expansion,
livelit definitions declare an \emph{expansion type}.
The declarations of the two livelits in Fig.~\ref{fig:grading},
eliding their implementations, are:
\begin{lstlisting}[numbers=none,xleftmargin=0cm]
livelit $dataframe at Dataframe {...}
livelit $grade_cutoffs(averages: List(Float)) at 
  (.A Float, .B Float, .C Float, .D Float) {...}
\end{lstlisting}
The expansion type of \li{\$dataframe} is \li{Dataframe},
which classifies tabular floating point data together with string row and column names (see the expansion above).
The expansion type of \li{\$grade_cutoffs} is a labeled product of grade cutoffs (field labels are written \li{.label}
rather than \li{label:} in Hazel).
Hazel displays the information in the livelit declaration when the cursor is on the livelit's name,
just as it displays typing information in other situations (not shown).\todo{cite HATRA}{}

\subsection{Compositionality}\label{sec:splicing-and-parameterization}
Livelits are compositional: they can work with sub-expressions  
in the form of parameters and splices.

\subsubsection{Parameters}\label{sec:parameterization} 
Livelit can declare a finite number of parameters of specified types. 
For example, \li{\$grade_cutoffs} above declares one parameter,
the averages to be plotted, of type \li{List(Float)}.
Parameters are applied  
using function application notation 
as seen in Fig.~\ref{fig:grading}(c) or 
using the pipelining (i.e. reverse function application) operators, \li{<|} and \li{|>}, 
which allow multiple livelits to form dataflows (not shown).

Livelit abbreviations can partially apply parameters. For example, consider the slider livelit
from Fig.~\ref{fig:color}\todo{where?}{}\todo{int or float?}:
\begin{lstlisting}[numbers=none,xleftmargin=0cm]
livelit $slider (min: Int) (max: Int) at Int {...}
\end{lstlisting}
We can partially apply the first parameter to define a parameterized non-negative slider livelit:
\begin{lstlisting}[numbers=none,xleftmargin=0cm]
let $nnslider = $slider 0 in ...
\end{lstlisting}
Only livelits with no remaining parameters can be invoked, 
so writing \li{\$nnslider} in expression position will display as a ``missing livelit parameter'' error.%
\footnote{\label{footnote:typing}In Hazel, erroneous expressions 
are automatically placed inside holes so that they do not prevent other parts of the program from evaluating
\cite{HazelnutLive}.}


\begin{figure*}
  \begin{center}
    \begin{subfigure}[t]{0.5\textwidth}
      \centering
      \includegraphics[width=15pc]{img-filter-1.png}
      \caption{}
    \end{subfigure}\begin{subfigure}[t]{0.5\textwidth}
      \centering
      \includegraphics[width=15pc]{img-filter-2.png}
      \caption{}
    \end{subfigure}
  \end{center}
  \caption{Case Study: Image Transformation. The image shown is determined based on the selected closure.}
  \label{fig:img-transformation}
\end{figure*}


\subsubsection{Splices}\label{sec:splices}
Spliced expressions, or \emph{splices}, appear directly inside the livelit GUI.
Splices can be filled with Hazel expressions of any form, including other livelit invocations.
For example, each cell in the \li{\$dataframe} GUI in Fig.~\ref{fig:grading}
has a corresponding splice. The formula bar at the top 
allows the user to edit the splice corresponding to the selected cell,
and all of Hazel's editing affordances are available when the client does so.
% Not all splices necessarily appear in the GUI.
Unlike parameters, the number of splices can change 
as the user interacts with the livelit, e.g. when changing the number of rows or columns in a \li{\$dataframe}.
% The cell itself displays the live value of the spliced expression---
% (we return to live evaluation in Sec.~\ref{sec:live-evaluation} below.

The livelit provides an expected type for each splice when it is created.
For example, the splices for the row and column keys in Fig.~\ref{fig:grading}(c)\todo{go through subfigures}{} have expected type \li{String}, 
and the remaining cells have expected type \li{Float}.
Hazel surfaces and uses the expected type when the cursor is in the splice.\todo{cite HATRA paper}{}
% If an expression of invalid type is entered, it will display in an error hole as usual,
% and in Hazel this will not prevent evaluation of other expressions (see Footnote \ref{footnote:typing}).

% Unlike parameters, the number of splices is not fixed in the livelit declaration. Splices can be created,
% deleted, and filled through user interaction with the livelit. For example, clicking the \li{+} buttons
% in Fig.~\ref{fig:grading} will create new rows or columns, which will in turn generate new splices.

\subsubsection{Hygienic Composition}\label{sec:hygiene}
 

Ensuring that clients can reason about binding while leaving expansions
 invisible 
 requires a hygiene discipline that enforces \emph{capture avoidance}
and \emph{context independence} \cite{TLMs}\todo{cite michael hygiene paper}.

\paragraph{Capture Avoidance}
Splices and parameters can appear anywhere in the expansion. 
This becomes potentially problematic when
the expansion places the parameter or splice under a binder, e.g. in the body of a function or \li{let} binding.
Na\"ively, this could cause inadvertent capture of the bound variable by a free variable
in the parameter or splice. For example, consider a livelit that generates an expansion
of the following seemingly innocuous form:
\begin{lstlisting}[numbers=none]
let len = strlen <splice1> in
Some (<splice2> + len)
\end{lstlisting}
Here, \li{<splice2>} appears under the binding of \li{len}. If the client has filled
\li{<splice2>} with an expression that refers to a client-side binding of \li{len},
these references would na\"ively be captured. This would not occur in \li{<splice1>}, 
because the \li{let} is not recursive.
This breaks abstraction and is notoriously difficult to debug,
both for the livelit provider, who has no way to predict which variables a client will use,
 and the client, who does not know which variables the provider used.

To avoid this situation, parameters and splices are placed in the expansion
in a capture-avoiding manner: variables in splices
always refer to the bindings visible to the client, 
rather than bindings that are hidden inside the expansion.
We discuss how this is implemented in Sec.~\ref{sec:expansion}.
% This is implemented by alpha-renaming bindings internal to the expansion as necessary.
% (We discuss potentially relaxed variations of this hygiene discipline in Sec.~\ref{sec:discussion}.)

\paragraph{Context Independence}
The example expansion above used a library function, \li{strlen}.
Na\"ively, this expansion would break if placed 
in client contexts where \li{strlen} is not bound, or bound to 
an unexpected value.
To avoid requiring clients to determine and satisfy these invisible 
dependencies, the livelits mechanism enforces \emph{context independence}:
generated expansions are valid in any context. Dependencies are bound 
relative to the livelit definition site (see Sec.~\ref{sec:expansion}).

\subsection{Live Evaluation}\label{sec:live-evaluation}
Livelits have the ability to evaluate a splice or a parameter 
in order to provide better feedback about run-time behavior to the client.
The \li{\$dataframe} livelit uses this facility to display
the evaluation result for each cell, like a spreadsheet.
The \li{\$grade_cutoffs} livelit uses this facility to plot the grades, which were 
passed in as a parameter, on the number line.

\subsubsection{Closure Collection} The subtlety here is that 
evaluation in Hazel is defined for closed expressions as is typical,
but parameters and splices can be open, i.e. refer to variables in the surrounding
context. In order to provide a suitable run-time environment that binds these variables, 
Hazel performs \textbf{closure collection} in two phases.

In the first phase, \emph{proto-closure collection}, 
Hazel replaces each livelit with a uniquely numbered hole and then evaluates the program 
using the semantics for evaluating programs with holes developed by \citet{HazelnutLive}.
Evaluation proceeds around these holes, producing a result with hole closures, i.e. holes with environments.

For example, there is one closure for \li{\$dataframe} in Fig.~\ref{fig:grading}.
It contains the value of the \li{q1max}\todo{what?}{} variable from Line 1 and any other variables in the context
around that livelit. 
These values can be used to evaluate 
splices that use the corresponding variables, such as the cell selected in Fig.~\ref{fig:grading}(c). 

Similarly, the closure for \li{\$grade_cutoffs} in Fig.~\ref{fig:grading} includes a result for
the \li{grades} variable. However, there is a problem: this variable's result depends on the
expansion of the \li{\$dataframe} livelit. If we stop after the first phase of closure collection,
where livelits have not yet been expanded,
then no useful value for the \li{grades} variable will be available after proto-closure collection:
it will just be a \li{\$dataframe} livelit hole closure.
For this reason, there is a second phase of closure collection, called \emph{closure resumption}, 
where any livelit holes that appear
in the collected livelit closures are \emph{resumed}, i.e. the expansion is generated
(in this case, for the \li{\$dataframe} livelit), the hole is filled, 
and evaluation resumes.
Livelit expansions do not contain livelits, so no subsequent resumption phases are necessary.

Hazel does not need to evaluate the program multiple times in order to support live closure collection
as well as the usual live evaluation services it offers, because the final evaluation result 
can also be determined by resumption.
(We discuss how to address subtleties that arise 
in the presence of non-commutative side effects in Sec.~\ref{sec:calculus-closure-collection}.)

\subsubsection{Indeterminate Results}
We used the phrase \emph{evaluation result} rather than \emph{value} above purposefully:
splices and parameters can be incomplete, i.e. they can contain holes, so
evaluation does not always result in a value \cite{HazelnutLive}.
In particular, when a hole appears in elimination position, e.g. in function position of a function application,
 the result is instead an indeterminate expression.
When a livelit requests an evaluation result, it must be able to handle indeterminate results.
For example, if the parameter to \li{\$grade_cutoffs} were a hole or some other indeterminate expression,
then it would have degraded functionality
(in this case, it would display the list elements that are values on the timeline, skipping indeterminate elements).
We will return to how livelit implementations handle this situation in Sec.~\ref{sec:live-evaluation-def}.



\subsubsection{Case Study: Live Image Transformations}\label{sec:image-transformation}
\todo{revise this once we have final figure}{}
Our next case study is a simple image transformation livelit, \li{\$img_filter},
demonstrated in Fig.~\ref{fig:img-transformation}. This livelit takes 
a URL to an image as a parameter, and contains two splices of type \li{Int},
one to adjust the contrast, and the other to adjust the brightness.
In this example, we have filled those splices with a \li{\$slider}, but 
as above we could enter any expression of type \li{Int}.
The livelit shows a live preview of the transformed image.
The expansion generates the necessary calls to image processing functions, 
not shown.

The novelty in this example is that the livelit appears inside a function, 
\li{classic}, which takes an image URL as input. At the bottom 
of the figure, \li{classic} is mapped over a list (here, containing only 
two elements for the sake of presentation). Consequently, there are now 
two closures associated with the application of \li{\$img_filter} in the 
body of \li{classic}. Rather than disabling live evaluation in situations like 
this, Hazel instead allows the programmer to select between the closures when 
the cursor is on the livelit expression (via a simple sidebar toggle, not shown, which is shared 
with the general closure selection UI \cite{HazelnutLive}). 
The live feedback is based on the selected closure.
This makes it easy to see how the image filter being designed here will affect a
number of example images by quickly toggling between closures. The underlying expansion remains abstract, i.e. it refers to the image via the \li{url} variable.

% \subsection{Additional Examples}\label{sec:additional-examples}
% It would be nice to have a gallery-style figure and a brief overview of some other case studies
% and how they exercise the novel features of the livelits mechanism. Maybe some statistics on how
% many lines of code it took.

% Ideas:
% \begin{itemize}
%   \item derivation trees like Joomy's system (\url{https://joom.github.io/proof-tree-builder/src/})
% \end{itemize}
\section{Livelit Definitions}\label{sec:livelit-definitions}

\begin{figure}
\begin{lstlisting}
  type Color = (.r Int, .g Int, .b Int, .a Int)

  livelit $color at Color {
    capture { }

    type Model = (.r SpliceRef, .g SpliceRef, .b SpliceRef, .a SpliceRef)
    let init : UpdateMonad(Model) = do 
      r <- new_splice(`Int`, Some(`0`))
      g <- new_splice(`Int`, Some(`0`))
      b <- new_splice(`Int`, Some(`0`))
      a <- new_splice(`Int`, Some(`100`))
      return (r, g, b, a)

    type Action = 
    | ClickOnColor(Color)
  
    let view : Model -> ViewMonad(Html(Action)) = 
      fun model -> do 
        (* determine a color to display to the client, if possible *)
        r_res <- eval_splice(model.r)
        g_res <- eval_splice(model.g)
        b_res <- eval_splice(model.b)
        a_res <- eval_splice(model.a)
        let cur_color : Color = 
          case (r_res, g_res, b_res, a_res) 
          | (Some(Val(IntLit(r))), 
             Some(Val(IntLit(g))), 
             Some(Val(IntLit(b))), 
             Some(Val(IntLit(a)))) -> 
            Some((r, g, b, a))
          | _ -> None
        in 
        
        r_editor <- editor(model.r, FixedWidth(20))
        g_editor <- editor(model.g, FixedWidth(20))
        b_editor <- editor(model.b, FixedWidth(20))
        a_editor <- editor(model.a, FixedWidth(20))
        
        (* ...render the UI using these splice editors... *)
      
    let update : Model -> Action -> UpdateMonad(Model) =
      fun model (ClickOnColor(r, g, b, a)) -> do 
        set_splice(model.r, IntLit(r))
        set_splice(model.g, IntLit(g))
        set_splice(model.b, IntLit(b))
        set_splice(model.a, IntLit(a))
        return (r, g, b, a)
    
    let expand : Model -> (Exp, List(SpliceRef)) = 
      fun model -> 
        (`fun r g b a -> (r, g, b, a)`, [model.r, model.g, model.b, model.a])
  }
\end{lstlisting}
\caption{Livelit Implementation: \li{\$color} (simplified for presentation)}
\label{fig:color-impl}
\end{figure}

\noindent
Let us now look in more detail at a livelit definition. Fig.~\ref{fig:color-impl}
outlines the \li{\$color} livelit from Fig.~\ref{fig:color}. (We omit certain 
incidental implementation details and use unimplemented syntactic sugar, including
Haskell-style \li{do} notation and simplified quasiquotation, for the sake of presentation.
We say more about the implementation in Sec.~\ref{sec:implementation}.)

Line 3 of Fig.~\ref{fig:color-impl} specifies the livelit's name, \li{\$color}, and its \emph{expansion type},
\li{Color}, which was discussed in Sec.~\ref{sec:livelit-expansion}.
The remainder of the definition is the livelit's implementation.
Livelits are implemented using a variation on the functional model-view-update
framework popularized by the Elm programming language \cite{ElmArchitecture}. We add a fourth component,
expansion generation. In addition, we add a simple monadic framework to provide the necessary  
interface between the livelit and the programming environment, discussed below, while retaining
a pure functional programming style.

\subsection{Model}\label{sec:model}
Line 6 of Fig.~\ref{fig:color-impl} specifies the livelit's \emph{model type},
here a labeled 4-tuple of \emph{splice references}, one for each of the four splices
that appear in the GUI in Fig.~\ref{fig:color}.
Models are persisted in the syntax tree of the program, so in practice,
 the system would check that the model type supports automatic serialization. 
 (Hazel is able to serialize all types because it is implemented as a 
 simple interpreter.)

The initial value of the model is specified starting on Line 7, 
which defines a command in the \li{UpdateMonad} that returns the 
initial model value after generating four new splice references
using the \li{new_splice} command:
\begin{lstlisting}[numbers=none]
new_splice : (Typ, Option(Exp)) -> UpdateMonad(SpliceRef)
\end{lstlisting}
This command creates a fresh splice
of the given type and, optionally, with the given initial expression,
and returns a unique splice reference, which uniquely and abstractly identifies that splice.
Bolded types like \li{Typ}, \li{Exp}, \li{UpdateMonad}, and \li{SpliceRef}
are provided by the standard library. The \li{Typ} and \li{Exp} types 
encode the syntax of Hazel's types and expressions.
These two arguments are provided here in quasiquoted form, e.g. \li{`0`} \cite{bawden1999quasiquotation}. 
If no initial expression is provided, the splice is initialized with 
an empty hole.

The system checks that the provided type and initial content is valid 
assuming only the explicitly specified set of captured bindings on Line 4.
Here, we do not need to capture any bindings because \li{Int} is a built in type
and the initial content is closed. However, 
for splices of user-defined types, those types need to be explicitly captured.
Similarly, helper functions that are used in initial splices (and in the expansion,
discussed below) must be explicitly captured. 
This serves to ensure \emph{context independence} -- the livelit needs to make no 
assumptions about the typing context at use sites.
We use an explicit capture set, rather than implicitly capture of all bindings at the definition site, 
both to prevent inadvertent capture 
and to ensure that the capture set is part
of the interface of the livelit definition, which simplifies matters related to 
module export and determining paths to the captured bindings at use sites \cite{TLMs}.
Formally, captured variables can be understood as parameters that have been 
immediately partially applied.

\subsection{Action}
Line 14 defines the \li{Action} type for the \li{\$color} livelit, which 
specifies a single user-initiated action: clicking on a color using the user 
interface on the right side of Fig.~\ref{fig:color}. Actions are emitted
from event handlers (e.g. click handlers) defined in the computed \li{view}, 
and actions are consumed by the \li{update} function, causing a change in the model. 
Let us discuss each of these functions in turn.

\subsection{View}
The \li{view} function computes the view given a model and access to the commands in 
the \li{ViewMonad}. The computed view is a value of type \li{Html(Action)}.  
The type family \li{Html(a)} provides a simple immutable
encoding of an HTML tree, where the type parameter \li{a} is the type of actions that 
are emitted by event handlers that can be attached to elements of the tree, e.g.
\li{on_click} and so on. 
We elide the details of the particular user interface in Fig.~\ref{fig:color}
to focus on two key mechanisms exposed by \li{ViewMonad}: live evaluation 
and splice rendering.

\subsubsection{Live Evaluation}
In order to provide the client with live feedback about the dynamic implications 
of the programmer's choices, 
the view function can ask the system to evaluate a splice, or an encoded function of 
multiple splices as long as the function is well-typed using only the capture set, 
under the closure that the client has selected using one of the following
commands:
\begin{lstlisting}[numbers=none]
eval_splice : SpliceRef -> ViewMonad(Option(Result))
eval_fun    : (Exp, List(SpliceRef)) -> ViewMonad(Option(Result))
\end{lstlisting}

The \li{None} case 
arises when evaluation is not possible, either because no closures are available
or because the free variables in the selected closure overlap with the free 
variables in the provided expression. This occurs only when the user has selected
a closure that appears under a binder in the evaluation result (e.g. because 
the result is a lambda). 
We discuss this further in Sec.~\ref{sec:calculus-closure-collection}.

If a result is available, the \li{Result} type distinguishes two possibilities:
\begin{lstlisting}[numbers=none]
type Result = 
| Val(Exp)
| Indet(Exp)
\end{lstlisting}
The \li{Val} case arises when evaluation produces a true value, whereas the 
\li{Indet} case arises when evaluation results in an indeterminate expression,
i.e. an expression that cannot be further evaluated due to one or more holes 
in critical elimination positions \cite{HazelnutLive}.

In our example, the livelit determines a color to display to the client 
if all four splices evaluate to integer literals. Otherwise, there is not 
enough information to determine a color, and the livelit can decide how to 
communicate this information to the client, e.g. with an "X" over the color 
preview.

Livelits can attempt to offer feedback even when the result is indeterminate,
because indeterminate expressions might nevertheless contain useful information.
For example, a livelit that previews a sequence of notes as audio might be able 
to handle a list of notes where certain notes are missing, i.e. holes, by 
simply playing silence or some other default sound when it encounters them.
This behavior is highly domain-specific, so each livelit provider must decide 
whether and how indeterminate results are supported.

\subsubsection{Splice Rendering}
The remainder of the \li{view} function is responsible for computing the view.
This is straightforward but for the question of splices. In Fig.~\ref{fig:color},
we want the splices to render as expression editors, with all of Hazel's editor 
services available to the user. In order to support this, the \li{view} function
can request an editor for a given splice, with given dimensions:
\begin{lstlisting}[numbers=none]
editor : (SpliceRef, Dimensions) -> ViewMonad(Html(a))
\end{lstlisting}
The result is an opaque \li{Html(a)} value that the remainder of the function 
can place as needed. When the livelit is rendered, this part of the tree is 
under the control of Hazel. The \li{Dimensions} parameter currently supports only a fixed
character width, with overflow causing scrolling, but in the future we plan to offer 
to offer more flexible layout options.

Another situation that can arise is when the UI needs to include not an editor but 
a rendered evaluation result. For example, each of the cells in the \li{\$dataframe}
livelit in Fig.~\ref{fig:grading} shows the evaluation result for the corresponding 
cell. Only the formula bar at the top is an editor. To support this, the \li{view}
function can use the \li{result_view} commands, which mirror the \li{eval} commands:
\begin{lstlisting}[numbers=none]
result_view_splice : (SpliceRef, Dimensions) -> ViewMonad(Html(a))
result_view_fun    : (Exp, List(SpliceRef), Dimensions) -> ViewMonad(Html(a))
\end{lstlisting}


\subsection{Update}
When the user triggers an event in a livelit view, it emits an \li{Action}.
The system responds by calling the \li{update} function to determine how 
this action should affect the model and the splices. 

We discussed in Sec.~\ref{sec:model} that the \li{UpdateMonad}
allows new splices to be generated in response to user actions. 
For example, the \li{\$dataframe} livelit uses this when new rows
or columns are added.

In Fig.~\ref{fig:color-impl}, we see that the \li{\$color} livelit responds to 
the \li{ClickOnColor} action by invoking the \li{set_splice} command to overwrite 
the current splices with integer literals determined based on which color the user
clicked on:
\begin{lstlisting}[numbers=none]
  set_splice : (SpliceRef, Exp) -> UpdateMonad(Unit)
\end{lstlisting}
The provided expression must satisfy the splice type and only make use of the capture set.

When the model is updated, a new view is 
computed. The system then performs a diff between the old and new view in order to 
efficiently perform the necessary imperative updates to the editor's visual state.
Changes to splices can also cause the view to be recomputed, because the view might 
need to evaluate the splices. The \li{update} function does not itself 
have the ability to request evaluation, because the model should not depend directly  
which closure the user has selected. Of course, the \li{view} might emit 
result-dependent actions when appropriate.

\subsection{Expansion}
The ultimate purpose of a livelit is to fill the hole where it appears by generating an expansion,
i.e. a standard symbolic expression of the expansion type, here \li{Color}.
The \li{expand} function, shown in Fig.~\ref{fig:color-impl}, is responsible for generating 
an expansion given the current model.

The expansion can include splices, but we the system does not make the contents of a splice 
available directly. Instead, the \li{expand} must return a pair consisting of an encoded expression, of type 
\li{Exp}, with a list of \li{SpliceRef}s that the expansion depends on. 
We call the \li{Exp} the \emph{parameterized expansion}
 because it must take an argument for each listed \li{SpliceRef}s. 
 That argument will be the corresponding splice 
type, which was provided when the splice was initialized. 
 The return type of the parameterized expansion is the expansion type, here \li{Color}.

This parameterization strategy makes it simple to enforce hygiene: the parameterized expansion 
can depend only on the capture set, whereas the splices are entered by the client and so they can 
depend only on the client site typing context. The use of standard function application ensures
that splices are capture avoiding. We consider this more formally in the next section.

\section{A Simply Typed Livelit Calculus}\label{sec:livelit-calculus}

\begin{figure}
    \[
    \arraycolsep=3pt\begin{array}{rlcl}
        \mathsf{Typ} & \tau & ::= &
                                    % \tnum ~\vert~
                                    \tarr{\tau_1}{\tau_2} ~\vert~
                                    \tprod{\tau_1}{\tau_2} ~\vert~
                                    \tunit ~\vert~
                                    \tsum{\tau_1}{\tau_2} ~\vert~
                                    t ~\vert~
                                    \trec{t}{\tau}\\
        \mathsf{UExp} & \hat{e} & ::= & 
                                 x ~\vert~
                                 \hlam{x}{\hat e} ~\vert~
                                 \hap{\hat e_1}{\hat e_2} ~\vert~
                                 ... ~\vert~
                                %  \hpair{\hat e_1}{\hat e_2} ~\vert~
                                %  \hprl{\hat e} ~\vert~
                                %  \hprr{\hat e} ~\vert~
                                %  \htriv ~\vert~
                                %  \hinL{\hat e} ~\vert~
                                %  \hinR{\hat e} ~\vert~
                                %  \hroll{\hat e} ~\vert~
                                %  \hunroll{\hat e} \\
                                 \hehole{u} ~\vert~
                                 \haplivelit{u}{a}{\dtxt{model}}{\splices}\\
        \mathsf{EExp} & e & ::= & x ~\vert~ \hlam{x}{e} ~\vert~ \hap{e_1}{e_2} ~\vert~ ... ~\vert~ \hehole{u}\\
        \mathsf{IExp} & d & ::= & x ~\vert~ \hlam{x}{d} ~\vert~ \hap{d_1}{d_2} ~\vert~ ... ~\vert~ \dehole{u}{\sigma}\\
        \mathsf{Splice} & \psi & ::= & \hsplice{\hat e}{\tau}
    \end{array}
    \]
    \caption{Syntax of types, $\tau$, unexpanded expressions, $\hat{e}$, expanded expressions, $e$, and internal expressions, $d$.
    Metavariable $x$ ranges over variables, $u$ over hole names, and $\livelitname{a}$ over livelit names.
    We write $\splat{\psi_i}$ for a finite sequence of $n \geq 0$ splices,
    and $\sigma$ for finite substitutions of $n \geq 0$ internal expressions for variables, $[d_1/x_1, \cdots, d_n/x_n]$.
    We elide standard expression forms
    related to product, sum, and recursive types.
    }
    \label{fig:syntax}
    \end{figure}

In order to precisely capture the semantics of livelits
independently
of the specifics of the Hazel programming environment and GUI
framework, we will now specify a simply typed \emph{livelit calculus}.

Fig.~\ref{fig:syntax} specifies the syntax of the livelit calculus.
Programs are written as \emph{unexpanded expressions}, $\hat e$, which are \emph{expanded} to
\emph{external expressions}, $e$, before being \emph{elaborated}
to internal expressions, $d$, for evaluation. All three stages are governed
by the same types, $\tau$. We include numbers, partial functions, products, sums, and recursive
types, all in their standard form \cite{pfpl}, but this specific type structure is not critical.
Any language expressive
enough to encode its own syntax (here as values of a recursive sum type)
would be a suitable basis. We begin with a terse
overview of the external and internal languages,
which are adapted straightforwardly from Hazelnut Live \cite{HazelnutLive} and therefore
serve as background material, in Sec.~\ref{sec:external-and-internal-lang}.
Only unexpanded expressions contain
livelits, so they are the main focus of the remaining sections,
which cover livelit expansion (Sec.~\ref{sec:calculus-expansion}) and 
liveness via closure collection (Sec.~\ref{sec:calculus-closure-collection}). 
% and parameterization (Sec.~\ref{sec:calculus-parameterization}).


\subsection{Background: External and Internal Language}\label{sec:external-and-internal-lang}
The external and internal languages are straightforward adaptations of the
external and internal languages of \emph{Hazelnut Live},
a typed lambda calculus that assigns static and dynamic meaning to programs with holes,
notated externally $\hehole{u}$ where $u$ is a \emph{hole name} \cite{HazelnutLive}.
We omit non-empty holes (which internalize type inconsistencies \cite{Hazelnut}) and type holes
(which operate like the unknown type from gradual type theory \cite{Siek06a,Hazelnut})
because these mechanisms are orthogonal to livelits.

The external language is governed by a typing judgement of the standard form, $\hasType{\Gamma}{e}{\tau}$,
where the typing context, $\Gamma$,
is a finite set of typing assumptions of the form $x : \tau$ \cite{pfpl}.

The internal language is a contextual type theory \cite{Nanevski2008}, i.e. the typing judgement is
of the form $\hasTypeD{\Delta}{\Gamma}{d}{\tau}$ where $\Delta$ is a finite set of hole typing
assumptions of the form $u :: \tau[\Gamma]$.
We need this hole typing context only for the internal language because, although hole names are assumed
unique in the external language, they can be duplicated during evaluation of internal expressions.
Consequently, we need a context to ensure that each closure associated with hole $u$, written $\dehole{u}{\sigma}$,
has the same type and can be filled
with expressions valid under the same context.

External expressions, $e$, are given dynamic meaning by typed elaboration to internal expressions, $d$,
according to the typed elaboration judgement, $\elabs{\Gamma}{e}{d}{\tau}{\Delta}$.
The main purpose of this elaboration step is to initialize the substitution $\sigma$ on each hole closure,
which operates to capture
the substitutions that have occurred around that hole during evaluation. The key rule is:
\begin{mathpar}
\inferrule[Elab-Hole]{ }{
    \elabs{\Gamma}{\hehole{u}}{\dehole{u}{\idof{\Gamma}}}{\tau}{u :: \tau[\Gamma]}
}
\end{mathpar}
The substitution is initially the identity substitution, $\idof{\Gamma}$, i.e. the
substitution that maps each variable in $\Gamma$ to itself, because no substitutions have yet occurred. For example,
\[ \elabs{ }{\helet{x}{\hnum{5}}{\hehole{u}} + \hnum{6}}{\helet{x}{\hnum{5}}{\dehole{u}{[x/x]} + \hnum{6}}}{\tnum}{u :: \tnum[x : \tnum]} \]
During evaluation, $\evalsTo{d}{d'}$, the closure's substitution accumulates the substitutions that occur. For example,
the internal expression above evalues as follows:
\[
  \evalsTo{\helet{x}{\hnum{5}}{\dehole{u}{[x/x]} + \hnum{6}}}{
      \dehole{u}{[\hnum{5}/x]} + \hnum{6}
  }
\]
We will use hole closures to support live programming with livelits in Sec.~\ref{sec:calculus-closure-collection}.


All of these judgements are adapted directly from those of Hazelnut Live,
differing only in that we use a simpler declarative formulation rather than an algorithmic (bidirectional)
formulation for the sake of simplicity and generality. Rather than restating the rules, we will simply state the governing
metatheorems and refer the reader to the prior work for the full detalis \cite{HazelnutLive}.

First, we have that an elaboration always exists and it preserves typing.
\begin{theorem}[Typed Elaboration]
    If $\hasType{\Gamma}{e}{\tau}$ then $\elabs{\Gamma}{e}{d}{\tau}{\Delta}$ for some $d$ and $\Delta$ such
    that $\hasTypeD{\Delta}{\Gamma}{d}{\tau}$.
\end{theorem}

Next, we have that evaluation of an expression with holes leads to a final (i.e. irreducible) result of the same type (this is a corollary
of the type safety properties established in the prior work.)
\begin{theorem}[Preservation]
    If $\hasTypeD{\Delta}{\cdot}{d}{\tau}$ and $\evalsTo{d}{d'}$ then $\isFinal{d'}$ and $\hasTypeD{\Delta}{\cdot}{d'}{\tau}$.
\end{theorem}

\subsection{Expansion}\label{sec:calculus-expansion}

The novelty of the livelit calculus lies entirely in its handling of unexpanded expressions,
$\hat e$, which are given meaning by typed expansion to external expressions,
$e$, according to the judgement $\expands{\Phi}{\Gamma}{\hat e}{e}{\tau}$
defined in Fig.~\ref{fig:expansion}.
Unexpanded expressions mirror external expressions but for the presence of livelits, discussed below.
The rules for standard
forms like variables, functions and function application, shown in Fig.~\ref{fig:expansion},
are straightforward.

\subsubsection{Livelit Contexts}

Livelit definitions are collected in the livelit context, $\Phi$, which
maps livelit names to livelit definitions of the following form:
\[ \livelitCtxEntry{a}{\taut{expand}}{\taut{model}}{\dtxt{expand}} \]
Here, $\taut{expand}$ is the expansion type, $\taut{model}$ is the model type,
and $\dtxt{expand}$ is the expansion function, which generates an expansion given a model.
We omit logic related to view computations and actions, which are tied to a particular 
UI framework.

For a livelit context to be well-formed, we require that each livelit definition be well-formed.
This in turn requires that the expansion function be hole-free and have the correct type.
\begin{definition}[Livelit Context Well-Formedness]
    A livelit context $\Phi$ is well-formed if and only if for each livelit definition,
    $\livelitCtxEntry{a}{\taut{expand}}{\taut{model}}{\dtxt{expand}} \in \Phi$,  we have
    $\hasType{}{\dtxt{expand}}{\tarr{\taut{model}}{\expType}}$.
\end{definition}
Here, $\expType$ stands for a type whose values isomorphically encode
external expressions. The isomorphism is mediated in one direction by
the encoding judgement $\encodeExp{e}{d}$ and the other by the decoding judgement $\decodeExp{d}{e}$.
Any isomorphism is sufficient, so we leave the details as a matter of implementation.
The simplest approach would be to define $\expType$ as a recursive sum type,
with one arm for each form of external expression (cf. \cite{TSLs} for an example).

For simplicity, we assume that the livelit context is provided \emph{a priori} and therefore
that functions like $\dtxt{expand}$ are already closed and fully elaborated.
In practice, the livelit context would be controlled by a definition form in the language
that allows the expansion function to itself use livelits.
This would require a staging mechanism,
because we need to execute expansion functions in prior definitions to be able to
expand subsequent definitions.
There are a number of ways to support the necessary staging in practice, e.g.
via explicit staging primitives \cite{DBLP:conf/icfp/Flatt02},
by requiring that these definitions appear in separately compiled packages \cite{TLMs},
or by using live
programming mechanisms such as those available in Hazel
to evaluate ``up to'' each definition before proceeding \cite{HazelnutLive}.
This choice is orthogonal to the mechanisms of interest in this section.

\subsubsection{Hygienic Livelit Expansion}
\begin{figure}
    \begin{mathpar}
        \inferrule[EVar]{
            x : \tau \in \Gamma
        }{
            \expands{\Phi}{\Gamma}{x}{x}{\tau}
        }

        \inferrule[ELam]{
            \expands{\Phi}{\Gamma, x : \taut{in}}{\hat e}{e}{\taut{out}}
        }{
            \expands{\Phi}{\Gamma}{\hlam{x}{\hat e}}{\hlam{x}{e}}{\tarr{\taut{in}}{\taut{out}}}
        }

        \inferrule[EAp]{
            \expands{\Phi}{\Gamma}{\hat e_1}{e_1}{\tarr{\taut{in}}{\taut{out}}}\\\\
            \expands{\Phi}{\Gamma}{\hat e_2}{e_2}{\taut{in}}
        }{
            \expands{\Phi}{\Gamma}{\hap{\hat e_1}{\hat e_2}}{\hap{e_1}{e_2}}{\taut{out}}
        }

        \cdots

        \inferrule[EApLivelit]{
            \livelitCtxEntry{a}{\taut{expand}}{\taut{model}}{\dtxt{expand}} \in \Phi\\\\
            \hasType{ }{\dtxt{model}}{\taut{model}}\\
            \evalsTo{\hap{\dtxt{expand}}{\dtxt{model}}}{\dtxt{encoded}}\\
            \decodeExp{\dtxt{encoded}}{\etxt{pexpansion}}\\\\
            \hasType{ }{\etxt{pexpansion}}{\tarr{\{\tau_i\}_{i < n}}{\taut{expand}}}\\
            \{ \expands{\Phi}{\Gamma}{\hat e_i}{e_i}{\tau_i} \}_{i < n}
        }{
            \expands{\Phi}{\Gamma}{\haplivelit{u}{a}{\dtxt{model}}{\splat{\hsplice{\hat e_i}{\tau_i}}}}{
                \hap{\etxt{pexpansion}}{\{e_i\}_{i < n}}
            }{\taut{expand}}
        }
    \end{mathpar}
    \caption{Expansion}
    \label{fig:expansion}
    \end{figure}
Unexpanded expressions are unique in that they include livelits:
 \[\haplivelit{u}{a}{\dtxt{model}}{\splices}\]
Here, $\livelitname{a}$ names the livelit
 being applied. Livelits can be understood as filling holes, so $u$ identifies the hole
 that is, conceptually, being filled.
The current state of the livelit is determined by the current model value, $\dtxt{model}$,
together with a collection of splices, $\splices$. Each splice $\psi_i$
is of the form $\hsplice{\hat e_i}{\tau_i}$, where $\hat e_i$ is the spliced expression
itself (unexpanded, so it may contain other livelits) and $\tau_i$ is the type of that splice,
as determined when the livelit definition first requested the splice (discussed in Sec.~\ref{sec:livelit-definitions}).

Rule \rulename{EApLivelit} performs livelit expansion. Its premises, in order, operate as follows:
\begin{enumerate}
    \item \textbf{Lookup.} The first premise looks up the livelit name in the livelit context.
    \item \textbf{Model Validation.} The second premise serves to ensure that the model value, $\dtxt{model}$, is of the
    specified model type, $\taut{model}$.
    \item \textbf{Expansion.} The third premise applies the expansion function, $\dtxt{expand}$, to the model value, $\dtxt{model}$,
    producing the encoded expansion, $\dtxt{encoded}$, which, by the definitions given above, is of type $\mathsf{Exp}$.
    \item \textbf{Decoding.} The fourth premise decodes $\dtxt{encoded}$, producing the external expression it encodes, $\etxt{pexpansion}$.
    \item \textbf{Expansion Validation.} We call $\etxt{expansion}$ the \emph{parameterized expansion} because, according to the fifth premise,
    it must be a function that returns a value of the expansion type, $\taut{expand}$, when applied (in curried fashion, though this is not critical)
    to the splices, whose types, $\splat{\tau_i}$, are explicitly recorded in the livelit.
    We must take care to ensure that the parameterized expansion is \emph{context independent},
    i.e. that it cannot depend on the particular bindings available in the call site typing context, $\Gamma$.
    Context independence is one facet of what is colloquially known as \emph{hygiene} \cite{TLMs}.
    In this simple formulation of the system, we maintain context independence simply by
    requiring that the parameterized expansion be
    closed. Consequently, any necessary helper functions used in the expansion must be provided explicitly by the client
    via a splice. In Sec.~\ref{sec:livelit-definitions}, we discussed how the use of explicit capture sets, which can be understood as 
    immediate partial application, can lower the cognitive burden of this strict context independence discipline.
    Parameters formally operate in the same way as splices, and partial application is a well-understood technique, so we omit these details from the calculus.
    \item \textbf{Splice Expansion.} Finally, the sixth premise inductively expands each of the spliced expressions in the same context as the livelit
    expression itself.
\end{enumerate}
The conclusion of the rule then applies the parameterized expansion to the expanded splices.
By applying the splices as arguments, we maintain \emph{capture avoidance} -- splices cannot capture variables
bound internally to the expansion. Capture avoidance is the other facet of \emph{hygiene} \cite{TLMs}.

The typed expansion process is governed by the following metatheorem, which establishes that the expansion
is indeed an external expression of the indicated type.



\begin{theorem}[Typed Expansion]
    If $\expands{\Phi}{\Gamma}{\hat e}{e}{\tau}$ then $\hasType{\Gamma}{e}{\tau}$.
\end{theorem}
\begin{proof}
    We proceed by rule induction on the assumption.
    The cases involving the standard forms follow by straightforward induction.
    The only interesting case is the \rulename{EApLivelit} case.
    In this case, we have that $e = \hap{\etxt{pexpansion}}{\splat{e_i}}$.
    By assumption, we have $\{ \expands{\Phi}{\Gamma}{\hat e_i}{e_i}{\tau_i} \}_{i < n}$.
    By induction, we therefore have $\{ \hasType{\Gamma}{e_i}{\tau_i} \}_{i < n}$.
    In addition, by assumption we have that $\hasType{}{\etxt{pexpansion}}{\tarr{\splat{\tau_i}}{\taut{expand}}}$.
    By weakening, we therefore have $\hasType{\Gamma}{\etxt{pexpansion}}{\tarr{\splat{\tau_i}}{\taut{expand}}}$.
    Finally, by iterated application of the typing rule for function application over the splices,
    using the judgements just established,
    we have $\hasType{\Gamma}{\hap{\etxt{pexpansion}}{\splat{e_i}}}{\taut{expand}}$ as desired.
\end{proof}

When composed with the Typed Elaboration theorem and the Type Safety properties established for the internal
language, we achieve end-to-end type safety for unexpanded expressions (every well-typed unexpanded
expression expands to a well-typed external expression, which in turn expands to a well-typed internal
expression, which in turn evaluates in a type safe manner.)

\subsubsection{Agda Mechanization}
We have mechanized the typed expansion mechanism and proven the Typed Expansion theorem
using the Agda proof assistant, using the same techniques as were used in the Agda mechanization 
of Hazelnut Live \cite{HazelnutLive}. This mechanization is available in the anonymous 
supplemental material. 

\subsection{Live Feedback via Closure Collection}\label{sec:calculus-closure-collection}
In order to support live feedback, a livelit needs to be able to ask the system
to evaluate expressions under one of the closures associated with the livelit.
This mechanism was introduced by example in Sec.~\ref{sec:image-transformation}.
In this section, we will formalize the process of efficiently collecting closures
for the livelits that appear in a program.

The key idea is that we generate an alternative expansion,
called the \emph{cc-expansion},
where each livelit expands to an empty hole applied to the splices. In other words,
we leave a hole in place of the parameterized expansion, which we expand and record in the \emph{cc-context}, $\Omega$, on the side.
The key rule for the cc-expansion judgement, $\ccexpands{\Phi}{\Gamma}{\hat e}{e}{\tau}{\Omega}$, is:
\begin{mathpar}
\inferrule[CCApLivelit]{
    \{ \ccexpands{\Phi}{\Gamma}{\hat e_i}{e_i}{\tau_i}{\Omega_i} \}_{i < n}\\
    \expands{\Phi}{\Gamma}{\haplivelit{u}{a}{\dtxt{model}}{\splat{\hsplice{\hat e_i}{\tau_i}}}}{
        \hap{\etxt{pexpansion}}{\splat{e_i}}
    }{\taut{expand}}\\
    \elabs{ }{\etxt{pexpansion}}{\dtxt{pexpansion}}{\tarr{\splat{\tau_i}}{\taut{expand}}}{\Delta}
}{
    \ccexpands{\Phi}{\Gamma}{\haplivelit{u}{a}{\dtxt{model}}{\splat{\hsplice{\hat e_i}{\tau_i}}}}{\hap{\hehole{u}}{\splat{e_i}}}{\taut{expand}}
    {\cup_{i<n} \Omega_i \cup \{\Omegaitem{u}{\dtxt{pexpansion}}\}}
}
\end{mathpar}

We then elaborate and evaluate the cc-expansion
following Hazelnut Live's dynamic semantics for programs with holes, which were summarized
in Sec.~\ref{sec:external-and-internal-lang}. The result will contain some number of hole closures
for each livelit hole. We call these the proto-closures and their associated environments the proto-environments for that livelit hole, formally defined as follows.
\begin{definition}[Proto-Environment Collection]
If $\ccexpands{\Phi}{\cdot}{\hat e}{e}{\tau}{\Omega}$ and $\elabs{}{e}{d}{\tau}{\Delta}$
and $\evalsTo{d}{d'}$ and $u \in \domof{\Omega}$ then $\protoEnvsOf{\Phi}{\hat e}{u} = \{ \sigma \mid \dehole{u}{\sigma} \in d' \}$.
    % protoclosures(\hat e, u) = { sigma | hole^u_sigma in d' } where \hat e cc-expands to e and e elaborates to d and d evaluates to d'
\end{definition}
A proto-environment for a livelit hole might itself contain a proto-closure for another livelit hole.
For example, in Fig.~\ref{fig:color}, the proto-environment for the \li{\$color} livelit contained the
proto-closure for the \li{\$slider} livelit, via the \li{gray\_level} variable.
If we use proto-environments for live evaluation, then the actual value of that variable
would not be available (here, the color could not be displayed).

Consequently, the second step of closure selection is to fill any livelit holes that appear in the proto-environments for other livelit holes.
We do so by filling them using the parameterized expansions gathered in $\Omega$ and then resuming
evaluation where appropriate.
Formally, this involves the hole filling operation $\instantiate{d_1}{u}{d_2}$ for Hazelnut Live
(which derives from the metavariable instantation operation of contextual modal type theory \cite{HazelnutLive,Nanevski2008}).
This operation
fills every closure for hole $u$ in $d_2$ with $d_1$.
Unlike substitution, hole filling is
not capture-avoiding. Instead, the environment on each of these closures is applied to $d_1$
as a substitution, i.e. the delayed substitutions captured in the environment are realized.
In this case, however, the parameterized expansion is guaranteed to be closed due to
the context independence discipline we maintain in Rule \rulename{EApLivelit},
so hole filling acts simply as a syntactic replacement.

Formally, we begin by defining an operation $\fillof{\Omega}{\sigma}$ which acts on proto-environments
to fill any livelit holes that appear.
\begin{definition}[Livelit Hole Filling] ~
    \begin{enumerate}
        \item $\fillof{\Omega}{[d_1/x_1, \ldots, d_n/x_n]} = [\fillof{\Omega}{d_1}/x_1, \ldots, \fillof{\Omega}{d_n}/x_n]$
        \item $\fillof{\Omega}{d} = \instantiateB{\dtxt{pexpansion}}{u}_{\Omegaitem{u}{\dtxt{pexpansion}} \in \Omega}{d}$
    \end{enumerate}
\end{definition}

This first step may cause certain expressions to become non-final, because the filled hole is no longer
blocking evaluation. We therefore define an operation $\resumeof{\sigma}$ that resumes evalution for all closed expressions in $\sigma$.
(Open expressions can remain because some closures appear under binders in the final result, so a substitution has not yet been recorded
for the bound variables.)
\begin{definition}[Environment Resumption] ~
    \begin{enumerate}
        \item $\resumeof{[d_1/x_1, \ldots, d_n/x_n]} = [\resumeof{d_1}/x_1, \ldots, \resumeof{d_n}/x_n]$
        \item $\resumeof{d} = d'$ if $\fvof{d} = \emptyset$ and $\evalsTo{d}{d'}$
        \item $\resumeof{d} = d$ if $\fvof{d} \neq \emptyset$
    \end{enumerate}
\end{definition}

Finally, we can produce the final set of environments by filling and resuming the proto-environments.
\begin{definition}[Environment Collection]
    If $\ccexpands{\Phi}{\cdot}{\hat e}{e}{\tau}{\Omega}$
    % and $\elabs{}{e}{d}{\tau}{\Delta}$
    % and $\evalsTo{d}{d'}$ and $u \in \domof{\Omega}$
    then \[\envsOf{\Phi}{\hat e}{u} = \{\resumeof{\fillof{\Omega}{\sigma}} \mid \sigma \in \protoEnvsOf{\Phi}{\hat e}{u}\}\]
\end{definition}

This same fill and resume operation can be used to avoid recomputation when evaluating the fully expanded version of the user's program.
If the editor has already performed environment collection, then it can simply continue from where it left off
by filling and resuming
the remaining top-level livelit holes (those that do not appear in a proto-environment).

The correctness of the mechanisms described in this section rest on the fact that evaluation commutes with hole filling.
This only holds when the language is pure: evaluation order does not matter
so the system is free to evaluate around holes initially,
and then return to finish the job once they have been filled.

\begin{theorem}[Post-Collection Resumption]
    If $\ccexpands{\Phi}{\cdot}{\hat e}{\etxt{cc}}{\tau}{\Omega}$ and $\elabs{}{\etxt{cc}}{\dtxt{cc}}{\tau}{\Delta}$
    and $\evalsTo{\dtxt{cc}}{\dtxt{cc}'}$ and $\resumeof{\fillof{\Omega}{\dtxt{cc}'}} = d_1$
    and $\expands{\Phi}{\cdot}{\hat e}{\etxt{full}}{\tau}$
    and $\elabs{}{\etxt{full}}{\dtxt{full}}{\tau}{\Delta}$
    and $\evalsTo{\dtxt{full}}{d_2}$ then $d_1 = d_2$.
\end{theorem}
\begin{proof}
    The key observation is that filling the livelit holes in the cc-expansion gives the full expansion,
    i.e. $\fillof{\Omega}{\dtxt{cc}} = \dtxt{full}$. Resumption is simply evaluation for closed expressions.
    By commutativity of hole filling, established in the prior work \cite{HazelnutLive},
    we can delay hole filling until $\dtxt{cc}$ has first been evaluated to
    $\dtxt{cc}'$.
\end{proof}

In a language with side effects, one would instead need to modify evaluation of
cc-expansions such that the full expansion of each livelit application is evaluated at the same time as the
temporary expansion,
with the result stored for subsequent hole filling. This would ensure that
side effects happen in the same order and only once. The full details are beyond the scope of this work.

% \subsection{Parameterization}

% Add livelit expressions, livelit abbreviations,
% partial application support. Talk about how to work around the strict context
% independence discipline we have imposed.

% \subsection{Edit Action Semantics}
% So far, we have considered only the semantics of an individual editor state.
% However, livelits are useful because they evolve as the user interacts
% with them. In particular, the livelit's model, $\dtxt{model}$ can evolve
% due to livelit-specific user interactions. Furthermore, a livelit can also
% create and delete splices when a user interaction demands it (unlike functions, which have a fixed
% number of arguments).


\section{Extensions}\label{sec:extensions}
Is this section necessary?

\section{Implementation}\label{sec:implementation}
Implementing livelits requires tight integration between a rich editor, 
a type checker, and a live evaluator 
capable of evaluating incomplete programs and gathering hole closures.

\subsection{Hazel}
Hazelnut Live is the foundation of the Hazel programming environment, 
and Hazel has support for all of the necessary mechanisms, so 
it was natural to choose Hazel for our primary implementation.
Hazel is implemented in OCaml and compiled to JavaScript using the \li{js_of_ocaml} compiler \cite{DBLP:conf/aplas/RadanneVB16}.
% The core of Hazel is written in a pure functional style, so our implementation closely 
% follows the formalism.  
%  Hazel is a browser-based live functional programming environment 
% organized from the ground up around typed holes, with full . 

The livelit definition mechanism described in Sec.~\ref{sec:livelit-definitions}
is implemented in Hazel, albeit without some of the syntactic sugar in Fig.~\ref{fig:color-impl}.
This, together with the fact that Hazel lacks a mature GUI widget libraries as of this writing, 
makes complex examples tedious to implement within Hazel, so we also 
added the ability to define livelits using JavaScript or OCaml. These are loaded when 
Hazel is compiled. As Hazel evolves, we expect to need to define such ``primitive'' livelits less 
frequently, using them mainly for livelits that would benefit from access to established JavaScript 
or OCaml libraries.

Uniquely, every editor state in Hazel is 
semantically meaningful: it has a type, it can be evaluated, and it can be transformed 
in a type-aware manner. This implies that livelits remain fully functional at all times, 
even when the program is incomplete or erroneous. Hazel achieves this ``gap-free'' liveness guarantee by automatically inserting explicit holes as necessary 
while the user edits 
the program. Formally, Hazel is a type-aware structure editor \cite{Hazelnut}, rather than a text editor, 
although the developers are aiming to maintain a text-like experience (this effort is orthogonal to our own). 

% The examples in this paper were instead implemented in OCaml as modules satisfying the signature 
% implied by Fig.~\ref{fig:color-impl}. A functor then adapts these OCaml-based livelit implementations 
% to livelit definitions suitable for inclusion in the livelit context used in the semantics.


% This means that it is conceptually straightforward to hide the model and splice data and instead display 
% the computed view. However, there are a number of non-trivial layout challenges that come up, which we discuss
% in Sec.~\ref{sec:layout} below.

\subsection{Text Editor Integration}
Livelits do not require the use of a structure editor. 
We have also developed a 
proof-of-concept implementation of livelits in a textual program editor, Sketch-n-Sketch \cite{sns-pldi,sns-uist},
which recently added support for the necessary mechanisms \cite{DBLP:journals/pacmpl/LubinCOC20}.
Livelit interaction causes the serialized model in the text buffer to be changed, which updates the view.
This proof-of-concept is not at feature parity with 
our main implementation, but it demonstrates that a syntax-recognizing text editor \cite{DBLP:journals/tosem/BallanceGV92,DBLP:conf/sde/HorganM84,interactive-visual-syntax} 
is sufficient to support livelits, albeit with some gaps in availability when 
there are syntax errors.

\subsection{Layout}\label{sec:layout}
Whether implemented in a structure editor or a text editor, livelits present 
interesting layout challenges. 
Hazel uses an optimizing pretty printer based on the work of \citet{DBLP:journals/pacmpl/Bernardy17} to determine layout. This system relies 
fundamentally on character counts. Consequently, our implementation asks each 
livelit to specify dimensions in terms of character counts rather than pixels.
Livelits can be laid out either as inline livelits, like \li{\$slider},
which are one character high and appear inline with the code,
or as \emph{multi-line livelits}, which occupy up to the full width 
and a specified number of lines. % We omitted these specifications in Fig.~\ref{fig:color-impl}. 
% The pretty printer lays out multi-line livelits
% on their own line, as seen in Fig.~\ref{fig:grading}. 

% We are currently exploring a more flexible system whereby 
% multi-line livelits can be named and applied within a larger expression, but
% the GUI is not displayed until the next line, even if there are intervening characters.
% This is inspired by the \verb|~| placeholder for large textual literals in the 
% Wyvern programming language \cite{TSLs}. 

One might also considering a number of other layout 
options, e.g. inline-block literals \emph{a la} Wyvern \cite{TSLs}, pop-up livelits, livelits pinned to sidebars, and livelits that are rendered  
on a separate canvas or document while still formally being located within an underlying functional program. 

This latter option 
would be particularly interesting for end-user programming scenarios: users with limited
programming experience 
could interact with a collection of livelits laid out separately in the popular ``dashboard'' style, 
without necessarily
even being aware that their interactions are actually edits to an underlying typed
functional program. 
% More experienced users on the team could then make programmatic use of the entered
% data and parameters in a seamless manner. We are also excited about the possibility of end users
% getting curious enough to ``View Source'' in this setting, and in so doing 
% being exposed to typed functional programming concepts gradually.

% The layout of splices within livelits is also an interesting problem, and our current approach, where 
% the provider must specify a fixed size for each splice, is not particularly flexible. 
% A more principled system for GUI layout that integrates with program pretty 
% printing could be of substantial use in some settings. Formula bars, like the one in the \li{\$dataframe}
% livelit, are a useful design pattern to avoid splice layout issues, and it may be possible to provide 
% a generalized formula bar in the editor user interface that multiple livelits could share.
% While splices can in principle contain other livelits, it is difficult for small splice editors to 
% visually accomodate large livelit views in practice. 
% Fortunately, variables can be used to support composition via substitution 
% rather than nesting. A linter might suggest a refactoring to a variable in these scenarios.

% The Hazel pretty printer uses memoization and incremental diffs to maintain reasonable performance,
% and adding livelits to the system did not substantially interfere with the existing approach.
% For the relatively simple livelits that we have developed, performance has been adequate.
% However, more complex  livelits or livelits that update quickly 
% could easily cause problems with editor responsiveness.
% This could perhaps be resolved by performing livelit-related computations asynchronously. Care 
% would need to be taken to ensure that intervening edits do not cause synchronization issues.

% Many livelit-related computations can be intelligently memoized, e.g. view computations and unchanged splice evaluations,
% especially with the edit awareness that Hazel provides. 
% We leave further optimizations along these lines to future work.
% Prior work on incrementally updating UIs after code changes could be relevant to this effort \cite{burckhardt2013s}.

\subsection{Integration into Imperative Languages}
\label{sec:imperative-langs}
Our focus in this paper was on pure languages.
Side effects pose a challenge on two fronts.
If they occur in a livelit implementation,
then the state exists at edit-time. 
The most viable approach would be to retain the model-view-update
architecture and specify that any transient mutable state will be reset between updates.
If side effects occur in livelit expansions, 
then the closure collection mechanism requires more 
care to avoid unsoundness, as described in Sec.~\ref{sec:calculus-closure-collection}.
% However, these problems are surmountable, and we look forward to 
% efforts to integrate livelits into imperative languages. 

% The Elm architecture has proven practical for sophisticated user interfaces, 
% e.g. the Sketch-n-Sketch system \cite{sns-uist}, 
% and its purity addresses means there are no issues 
% related to initialization, maintenance, and persistence of mutable state
% by the editor.
% Imperative languages can express the same architecture (indeed, a close cousin 
% is at the heart of the popular React framework for Javascript).

\section{Related Work}\label{sec:related-work}
As detailed in Sec.~\ref{sec:background},
the Graphite system developed the idea of filling 
typed holes using a type-specific user interface, and it was the starting 
point for our work \cite{Graphite}.
Subsequent work on \textbf{mage} further explores this design space \cite{DBLP:conf/uist/KeryRHMWP20}.
This prior work engaged in substantial qualitative evaluations, 
which due to the fundamental similarities between the two systems 
is as relevant to our design as theirs. However, the prior work 
left a number of core technical issues unresolved,
as summarized Sec.~\ref{sec:contributions}.
Livelits resolve these in large part by bringing together  ideas from other recent work.

In particular, recent work on type-specific languages (TSLs) \cite{TSLs} 
and typed literal macros (TLMs) \cite{TLMs}
explored similar ideas of user-defined
literal forms with support for hygienic splicing, with the former 
using type-directed dispatch similar to Graphite and the latter supporting 
decentralized extensibility via explicit naming as in our approach. 
However, these systems operate in a purely 
textual setting and have no support for live feedback.
The typed expansion judgement central to the typed livelit calculus in this paper 
is structured similarly 
to the corresponding judgements in the formal systems describing TLMs and TSLs,
and the reasoning principles are closely related. However, the  
approach to capture avoidance we take is both more restrictive and cleaner by its use of function application 
rather than direct insertion of splices (and could perhaps be ported to those formalisms).
Splices also operate quite differently in our work,
because they are placed automatically and structurally delimited in the user interface, rather than 
placed by the client and then parsed out of the text by a custom parser charged 
with retaining provenance information.
This substantially complicates the design of splices in that work.

Recent work on a interactive visual macro system in Racket 
is quite similar in spirit to our work \cite{interactive-visual-syntax}.
However, it supports only a limited form of splicing via a pop up text editor
that does not support full compositionality as described here, where livelits 
can appear within other livelits. Splicing is not strictly hygienic, though 
Racket's scope management machine can be used by macros to voluntarily 
maintain a binding discipline \cite{DBLP:conf/popl/Flatt16}. There is not any 
consideration of typing. Parameterization and partial application is also impossible, because invocation 
is via an editor command rather than a syntactic mechanism. Finally,
there is no support for liveness: macro evaluation occurs in the editor 
environment, not the run-time environment of the program being written, 
so splice evaluation would not work correctly (except in trivial cases
where the splice was, e.g., a closed expression, or used only standard 
library constructs).

Our structural delimitation of splices is reminiscent of work on 
\emph{language boxes} \cite{DBLP:conf/sle/RenggliDN09}, which focused on 
combining different notations, primarily textual, using structural delimiters inserted 
explicitly using a special-purpose editor. 

There has been a long line of research on \emph{projectional editing}, where
the user edits graphical representations (projections) of code constructs 
\cite{DBLP:journals/ile/MillerPMV94,read1996generating,DBLP:conf/sde/Reiss84,DBLP:conf/uist/KoM05}. Livelits are a form of projectional editing. 
Many of the oldest systems offer only a fixed set of projections and interaction techniques. 
More recently, language workbenches with support for projectional editing like Citrus \cite{DBLP:conf/uist/KoM05}
and MPS \cite{voelter2011language} have made it easier to define new projectional editors.
However, these systems generate entire editors, whereas livelit definitions 
are defined in libraries. 
Furthermore, livelit definitions are governed by a type and binding discipline.
% In many ways, an unfamiliar livelit can be approached much like an unfamiliar function
% can be approached: by inspecting its declarations and reasoning compositionally 
% about arguments, i.e. splices and parameters.
Finally, livelits are uniquely live, building on Hazelnut Live \cite{HazelnutLive}.

A number of notebook systems, including Mathematica, Jupyter \cite{Guo13}, and others, do 
support the insertion of simple widgets like sliders that respond live to changes.
These systems inspired our approach but they are not compositional: only constant values can be constructed, as with Graphite.

Related to projectional editors are 
a variety of systems that generate visualizations from code \cite{DBLP:conf/chi/Lerner20,koschke2003software,urquiza2004survey}. Livelits differ 
from these systems in directionality: visualization systems are given code and generate
a visualization, whereas livelits are generate code where there would otherwise
be a hole.

Conal Elliott's work on tangible functional programming \cite{DBLP:conf/icfp/Elliott07} similarly explored a
system allows editing a graphical representation of code in compositional ways, 
but the editing representation itself is fixed as a series of connected windows.
Only the visual representation is customizable.

The Vital programming environment for Haskell \cite{hanna2002interactive} 
supports type-specific stylesheets
that can be used to create custom user interfaces for both editing and displaying
values. The editors support splices, called \emph{cells}, that can contain 
Haskell code. Results are computed in a live manner, though there is no support
for evaluating incomplete programs. The visualizations themselves cannot provide
live feedback. Moreover, the system does not enforce any hygiene or typing principles:
the user is entirely responsible for syntactic and semantic correctness.


% Hazelnut
% Hazelnut Live
% Its Alive! paper (GUI widgets)
% Tangible FUnctional Programming 
% Vital https://dl.acm.org/doi/10.1145/581478.581493 and https://link.springer.com/chapter/10.1007/11964681_12
% Projection Boxes http://cseweb.ucsd.edu/~lerner/proj-boxes.html
% Tiny Structure Editors for Low Low Prices
% Typed Holes Work (synthesis stuff from Haskell, mention Haskell)
% Pharo / Smalltalk 
% GEC Toolkit: https://link.springer.com/chapter/10.1007/11546382_5
% Previous use of live literals https://homepages.cwi.nl/~storm/livelit/livelit.html
% Visual Haskell and related visual programming environments: https://pdfs.semanticscholar.org/6e43/5494ef1aaaa6ec0fcf47bb5b03d7c06ef937.pdf
% widgets in dataflow languages like puredata and labview

\section{Discussion and Conclusion}\label{sec:discussion}\label{sec:conclusion}
\begin{quote}
  %
  \textit{
  %
  The arithmetical symbols are written diagrams and the geometrical figures are graphic formulas.
  %
  }
  
  \vspace{3pt}
  
  \hfill{}--- David Hilbert~\cite{hilbert1902mathematical}
  \end{quote}

  \noindent
  Diagrams have played a pivotal role in mathematical thought since antiquity,
  indeed predating symbolic mathematics \cite{cajori1993history}. 
  Popular computing and creative tooling, too, has embraced visual representation and direct manipulation 
  interfaces for decades.
  Programming, however, has remained stubbornly mired in textual user interfaces. 
  % Certainly, one must acknowledge that textual and symbolic notation
  % is an indispensable tool for abstract thought. 
  % Indeed, it is now widely recognized that variables and functions are at the 
  % very foundation of computing. 
  Our hope with this paper is to demonstrate that principled, mathematically structured
  programming is not only compatible with live graphical user interfaces, but that the 
  combination of these two holds promise for the future of programming.

  This paper's contributions are 
  in advancing the expressive power of the mechanism in several directions, 
  most notably in terms of compositionality and liveness. These 
  technical contributions are a complement to the empirical findings by \citet{Graphite} and others.
  That said, the two case studies we considered in this paper were motivated by 
  real world problems. 
  As the implementation
  matures, we plan to introduce enthusiasts in a wide variety of problem domains
  to livelits and continue these empirical evaluations.

  % There remain a number of avenues for future work. First, we have only just started
  % to apply the livelits mechanism in various application domains.  
  % started . 

%   In particular, we would like to better understand the conceptual and practical 
%   difficulties that livelit providers face, with the aim of making livelit 
%   implementation an extremely low cost activity. Livelits could themselves be 
%   useful for this task, if we develop a GUI widget library with support for livelits.
% In addition, recent approaches to deriving type-specific structure editors automatically from pretty printing logic \cite{hempeltiny} could perhaps 
% be adapted to generate livelit implementations. 

Mechanisms for deriving simple 
livelit definitions from type definitions, perhaps similar to Haskell's \li{deriving} directive
or the GEC toolkit \cite{DBLP:conf/afp/AchtenEPW04}, or from \li{to_string} functions \cite{DBLP:conf/vl/HempelC20}, 
may prove fruitful in the future.

  The livelits mechanism as described in this paper operates only on expressions,
  but livelits might be useful for generating other sorts of terms, such as types,
  patterns, and entire modules. Prior work on literal macros has explored this \cite{TLMs}.
  % Support for type and module splices, too, would likely be useful.
  % With these, better support for reflecting on the provided type might allow for 
  % the development of more general livelits, e.g. a variant of \li{\$datatype} 
  % with individually typed columns, rather than only floating point data.

   The strict hygiene discipline has, we believe, substantial 
   benefits---programmers will inevitably encounter unfamiliar livelits, and 
   the reasoning principles that we enforce are likely to help them ``reason around''
   the situation. However, it may be useful in certain circumstances to 
   relax these, with the editor alerting the user to the unusual situation.

  Another direction for future work has to do with pushing edits from computed results
  back into livelits. For example, a slider expands to a number, which may 
  then flow through a computation. Bidirectional evaluation techniques may allow
  the user to edit a number in the result of a computation and see the necessary
  change to a slider in the program \cite{sns-pldi,sns-uist}.

  Programming and authoring have much in common. Documents often contain structured
  information, and programs are written to manipulate structured information.
  Another future direction for livelits is as the basis for a programmable authoring 
  system, where the non-symbolic elements on the page are revealed to be code after all,
  albeit code generated by a livelit invocation that presents a more natural editing experience. 
  Taking this further, a networked collection of these
  documents could form a powerful computational wiki.
  We present this paper as a foundation for such explorations.
  %  This would 
  % require addressing the difficult problem of supporting collaborative interactions 
  % involving edits to arbitrary user interfaces. 
  % We believe that these problems are surmountable.

\newpage
% \begin{itemize}
%   \item pattern matching
%   \item type splices
%   \item module splices
%   \item explicitly make bindings available in splices
%   \item better UI for closure provenance / connect with control flow / call stack better
%   \item deriving livelits from type definitions
%   \item bidirectional evaluation ala SnS
%   \item integration with structure editing more cleanly
%   \item \dots
%   \item collaborative editing?
%   \item side effects?
%   \item full screening livelits (or compositions thereof) as a way to create end-user workflows
%   \item big vision: authoring environment based on livelits
% \end{itemize}


\clearpage
\appendix
\section{Misc. Material from Prior Draft}

To address these limitations, we introduce live literals, or livelits. Figure~\ref{fig:cutoffs} shows a mockup of the definition and application of a livelit named \li{$grade_cutoffs} for adjusting grade cutoffs, represented as values of  record type \li{grade_cutoffs}. Like textual literal forms (e.g. list literals), livelits are alternative representations of expressions of the associated type \cite{DBLP:journals/pacmpl/OmarA18}. % From the perspective of the remainder of the program, \li{cutoffs} is a value of type \li{grade_cutoffs} like any other.
We are implementing livelits in Hazel (\url{hazel.org}), a live functional programming environment with support for typed holes \cite{popl-paper}. We plan to perform a live demo.




The definition of
the livelit, outlined in Figure~\ref{fig:cutoffs}, follows the Elm architecture,
i.e. there are types representing the abstract model and the messages that the 
GUI generates. Livelits are \textbf{persistent} rather than ephemeral, i.e. the model is recorded in the underlying syntax tree. The \li{view} function generates the GUI, implemented using HTML, on demand (so the view is not persisted). Another function, \li{to_exp}, is responsible for generating the underlying expression, called the \emph{expansion}, from the \li{model}. 
While other projectional editors, e.g. those generated by Citrus \cite{DBLP:conf/uist/KoM05}, also support persistent GUIs in code, they are not user-extensible. 

The GUI can itself contain typed holes, represented by the \li{HtmlHole} constructor. In this case, there is a single hole
for entering an expression of type \li{list(float)}, i.e. the list of weighted
averages. In this example, the user has filled this hole with a variable,
\li{weighted_averages} (the definition of which precedes the content of this figure and is not shown). In other words,
livelits support \textbf{open expressions} and therefore interact cleanly with 
standard abstraction mechanisms, i.e. they can appear under binders. We follow the reasoning principles for literals with spliced sub-expressions established  by \citet{DBLP:journals/pacmpl/OmarA18}.

The main complication when dealing with open expressions relates to how live feedback
is to be generated. Given just the symbolic expression in the hole, it would be 
impossible to plot (as orange dots) the actual data from the list that the variable refers to. To resolve this issue, the system evaluates the program 
as if it the livelits were empty holes, 
relying on the support for evaluating incomplete programs described recently 
by \citet{DBLP:journals/pacmpl/OmarVCH19}. The result of evaluation is an expression containing
hole closures, i.e. holes equipped with environments. 
%Hazel allows the user to interactively select which closure
%interests them if multiple closures for a given hole appear in the result (e.g. if a livelit appears inside a function called multiple times). 
The livelit 
can then evaluate the expression in the hole against the closure selected by the user (not shown) using \li{Env.run}.


%%%%%

\todo{this is the WIP one from last year}{}
Programs are most often represented as text.
%
This representation format provides expert programmers access to a variety of
text editors and text edit actions---to affect fine-grained control over
programming decisions---and a large ecosystem of other text-based
utilities---for example, for line-based file differencing of source code
versions.
%
On the other hand, the flexibility and low-level nature of text comes at some
costs.
%
For experts, some tedious and manual edits could lead to
inefficiency---especially because many routine code editing operations do not
require the full flexibility of text~\citep{XXX}.
%
For novices, the large class of syntax errors that stem from text-edits presents
a steep learning curve.

Motivated to overcome the disadvantages of represent programs as text, a variety
of alternative code editing interfaces have been investigated.
%
At the opposite end of text are \emph{visual programming languages}, which often
completely represent a program with graphical elements~\citep{XXX}.
%
To forgo text completely, such approaches often target domain-specific
languages, such as dataflow programing~\citep{XXX} and pedagogical languages
that are, by design, restricted to relatively few building blocks.
%
Other \emph{structure}, or \emph{projectional} editors, still use a significant
amount of text to render programs, but forgo text-edits in favor of
\emph{structure edit actions} which transform the program (represented as a tree
or some other structure), sidestepping the danger of invalid intermediate states
of concrete syntax.

On the spectrum somewhere between fully text- and fully structure-based are
``hybrid'' editors, which augment text with additional ways to visualize and
manipulate the structure of the program.
%
Victor Scrubbing, APX~\citep{APX}, Sketch-n-Sketch~\citep{sns-pldi}, and Carbide
IDE~\citep{XXX}, among others, allow numeric values to be ``scrubbed'' by
directly manipulating sliders rather than text-editing numeric literals.
%
Barista~\cite{Barista} is a hybrid Java editor where custom \emph{structure
view} GUIs provide alternate representations of expressions instead of text.
%
For example, an arithmetic expression may be rendered with mathematical symbols,
a method may be accompanied by interactive documentation with input-output
examples, and structures may be selectively collapsed, expanded, or zoomed.
%
Graphite~\citep{Graphite} allows custom GUIs---called \emph{palettes}---to help
the programmer fill missing expressions---``holes''---in the program.
%
For example, a color palette can provide visual previews of different candidate
color values, and a regular expression palette can show input-output examples
for different candidate regular expressions.

The GUI representations and interactions enabled by the above hybrid editors are
useful, but several limitations likely preclude wider utility:
%
the types of expressions that benefit from alternative GUI editors are limited
to
%
privileged types that have baked-in interfaces~\citep{XXX}
%
or just of base type~\citep{XXX};
%
the GUI view is ephemeral, in that it disappears once it has been used to
generate an expression~\citep{XXX}; or
%
expressions generated by the GUI are not deeply integrated with the static type
system and interpreter of the language~\citep{XXX,XXX,XXX}.


%% refactoring
%% 
%% DNDRefactoring~\citep{DNDRefactoring}
%% Deuce~\citep{sns-deuce}


\parahead{Persistent, Composable, and Live GUIs for Filling Holes}

compared to the ``simple'' palettes

extend palettes with: persistence, composition, and live feedback

macro systems allow alternative syntactic (i.e. string) representations, to
expand into underlying expressions

palatable

by analogy to these ``literal'' macro systems, palettes are ``graphical
macros'': through interaction with the user, the GUI generates the underlying
expression.

TLMs~\cite{TLMs}


\parahead{Contributions and Paper Outline}


design for palettes. specifically, within the Hazelnut Live framework,
which is a system to address the gap problem that arises in traditional editor
frontends.

demonstration of the expressiveness of the approach through a series of
examples, many of which are drawn from the user study mentioned before.

a prototype implementation of palettes within Hazel, which currently
supports several examples using a core calculus with minimal features. the
implementation provides a clear path for scaling up to larger, full-featured
syntactic programming conveniences, as well as further UI design in future work.

prototype UIs:

  \begin{enumerate}
    \item \Hazel{}: (palette macros;) structure edit actions; expression formula bar; inline layout
    \item \sns{}: (palette functions;) text editor; pop-up menu
  \end{enumerate}

hazelnut~\citep{Hazelnut,HazelnutLive}



\begin{figure*}[t!]
  \begin{center}
  \includegraphics[width=34pc]{cutoffs-mockup.png}\end{center}
  \caption{A mockup of a livelit for adjusting grade cutoffs. Livelits are persistent and can access the live environment.}
  \label{fig:cutoffs}
  \end{figure*}
  

%% \section{Defining Palettes}
\section{Overview Examples: Using Palettes}
\label{sec:overview}

\begin{figure*}

\includegraphics[scale=0.50]{images/MrSmileyFace.png}

\caption{Example Palettes in Hazel.}
\end{figure*}

\begin{itemize}

\item figure with 2/3 Hazel palettes:

  \begin{enumerate}
    \item table (synthetic)
    \item grade cutoffs (analytic)
    \item color? (analytic)
  \end{enumerate}

\item emphasize persistence

\item emphasize composition

\item emphasize liveness

\item mention analytic (annotated) vs. synthetic (unannotated) palettes

  \begin{itemize}
    \item stick to palette macros for now
    \item introduce palette functions in ``Overview Examples: Defining Palettes''
          or ``Formal System''
  \end{itemize}

\end{itemize}

\section{Overview Examples: Defining Palettes}

Palette definitions take the following general form:
\begin{lstlisting}
palette $name 
  (arg1 : t1) 
  ... 
  (argn : tn) 
  at t 
  implementation P in package pkg;
\end{lstlisting}
The static semantics requires the following:
\begin{enumerate}
\item \li{arg1} ... \li{argn} for \li{n} $\geq 0$ are distinct labels
\item \li{t1} ...\li{t1} are valid types
\item \li{t} is a valid type
\item and \li{P} identifies a module that can be loaded from package \li{pkg} such that \begin{lstlisting}
P : IPalette
    with type args = { 
      arg1 : HoleRef.t<t1>, 
      ..., 
      argn : HoleRef.t<tn> 
    } 
    with type output = t
\end{lstlisting}
where \li{IPalette} is defined in Fig.~\ref{fig:IPalette} and \li{HoleRef} in Fig.~\ref{fig:HoleRefs}.
\end{enumerate}

\begin{figure}
\begin{lstlisting}
module type IPalette = {
  type args
  type output
  type model
  type msg
  val init    : args -> HRG.t(model)
  val update  : (model, msg) -> HRG.t(model)
  val view    : model -> HRE.t(Html.t(msg))
  val compute : model -> HRE.t(output)
}
\end{lstlisting}
\caption{\li{IPalette} module type}
\label{fig:IPalette}
\end{figure}

\begin{figure}
\begin{lstlisting}
(* Abstract type of hole refs *)
module HoleRef : {
  type t('a)
}

(* HRG is the hole ref generation monad *)
module HRG : {
  type t('a)
  val fresh('a) : t(HoleRef.t('a))
  val bind : t('a) -> 
             ('a -> t('b)) -> 
             t('b)
  val return : 'a -> t('a)
}

(* HRE is the hole ref evaluation monad *)
module HRE : {
  type t('a) 
  type result('a) = Value('a * Exp)
                  | Indet(Exp)
  val eval('a) : HoleRef.t('a) -> 
                 t(result('a))
  val bind : t('a) -> 
             ('a -> t('b)) -> 
             t('b)
  val return : 'a -> t('a)
}
\end{lstlisting}
\caption{Modules for working with hole refs}
\label{fig:HoleRefs}
\end{figure}

\subsection{Background: Elm architecture}

\subsection{GUIs with Typed Holes}

\subsection{Palette-Specific Actions?}

\subsection{Reasoning Principles}

\subsection{Deriving Palettes from Type Definitions?}

\clearpage
\begin{figure*}
        \newcommand{\absgrammar}[2]{c\,|\,#1 : #2\,|\,$x$\,|\,\hlam{$x$}{#1}\,|\,\halam{$x$}{#2}{#1}\,|\,\hehole{\mathbb{N}}\,|\,\hhole{#1}{\mathbb{N}}\,|\,#1\,#1}
\begin{grammar}
 Basic HTyps
 & $\htyp$
   & $\bnfas$ &
     b\,|\,$\tehole$\,|\,$\tarr{\htyp}{\htyp}$
%
\\[1ex]
 HTyps with cells
 & $\htypc$
   & $\bnfas$ &
     b\,|\,$\tehole$\,|\,$\tarr{\htypc}{\htypc}$
%
\\[1ex]
 HExps
 & $\hexp$
   & $\bnfas$ &
     $\absgrammar{\hexp}{\htyp}$
%
\\[1ex]
 HExps with cells
 & $\hexpc$
   & $\bnfas$ &
     $\absgrammar{\hexpc}{\htypc}$
 \\ &&& $\bnfaltbrk$ \cell{x}
%
\\[1ex]
 Palette definitions
 & $\pDef$
   & $\bnfas$ &
     $\pDefRecord{\hexpc}{\htypc}{\htyp}$
%
\\[1ex]
 Palette HExps
 & $\pexp$
   & $\bnfas$ &
     $\absgrammar{\pexp}{\htyp}$
 \\ &&& $\bnfaltbrk \pexpPalLet{p}{\pi}{\pexp}$ & Palette definition $\pi$ as $p$ in $\pexp$
 \\ &&& $\bnfaltbrk \pexpPalAp{p}{\hexpc}{x}{(\htyp, \pexp)}{C}$
                                                & $\hexpc$ (which can contain cells labeled with $x$ in $C$)
 \\ &&&                                         & is a model that is expanded by palette definition $p$
%
\end{grammar}
\begin{mathpar}
\newcommand{\icexpand}{\iota_{c_{expand}}}
\newcommand{\tcmodel}{\tau_{c_{model}}}
\newcommand{\texpansion}{\tau_{expansion}}
\newcommand{\tcexpansion}{\tau_{c_{expansion}}}
\newcommand{\icmodel}{\iota_{c_{model}}}
\newcommand{\icexpanded}{\iota_{c_{expanded}}}
\newcommand{\htypx}{\htyp_x}
\newcommand{\pexpx}{\pexp_x}
\newcommand{\hexpx}{\hexp_x}
\\\\
\inferrule[PELetPal]{
    \pi = \pDefRecord{\icexpand}{\tcmodel}{\texpansion} \\\\
    \hsyn{\EmptyhGamma}{\icexpand}{\tarr{\tcmodel}{\tcexpansion}} \\ \\
    \tconsistentc{\tcexpansion}{\texpansion} \\\\
    \pexpandAna{\hGamma}{\pPhi, p : \pi}{e}{\iota}{\tau}
  }{
    \pexpandAna{\hGamma}{\pPhi}{\pexpPalLet{p}{\pi}{e}}{\iota}{\tau}
  }
\\\\
\inferrule[PEApPal]{
    p:\pDefRecord{\icexpand}{\tcmodel}{\texpansion} \in \pPhi \\\\
    \hana{\EmptyhGamma}{\icmodel}{\tcmodel} \\ \\
    \ceval{\EmptyhGamma}{\hap{\icexpand}{\icmodel}}{\icexpanded} \\\\
    \forall x \in C, \left( \pexpandAna{\hGamma}{\pPhi}{\pexpx}{\hexpx}{\htypx} \right) \\ \\
    \celab{\EmptyhGamma}{\fmap{x}{C}{\hexpx}}{\icexpanded}{\iota}{\texpansion}
  }{
    \pexpandSyn{\hGamma}{\pPhi}{\pexpPalAp{p}{\icmodel}{x}{(\htypx, \pexpx)}{C}}{\iota}{\texpansion}
  }
\\
\end{mathpar}
\end{figure*}


\clearpage
\section{Evaluation}

\subsection{Prototype Implementations}

\parahead{\Hazel{}}

palette macros; structure edit actions; expression formula bar; inline layout

\parahead{\sns{}}

palette functions; text editor; pop-up menu

\subsection{Examples}

\rkc{old list of examples below:}

\subsubsection{Boolean}
Checkbox

\subsubsection{Matrix}
Simple example

\subsubsection{Grade Cutoffs}
Uses liveness more obviously

\subsubsection{Table}
Uses type reflection

\subsubsection{Pixel Art}
Cool example

\subsubsection{Regex}
Similar to Graphite paper, can cite the empirical study we did there

\subsubsection{Forms}
Show off composition + mention full-screening stuff

\subsubsection{Equation Editor, Judgement Editor and Category Diagrams}
Things that PL people like 

\subsubsection{TikZ diagrams}
...



\clearpage
\section{Related Work}
Graphite, Relit, projectional editors, notebooks / Mathematica, 

\url{https://twitter.com/johnregehr/status/1095018518737637376} -- ascii art in source code

\url{http://idl.cs.washington.edu/papers/lyra/} - plotting UIs

\begin{figure*}
\begin{lstlisting}
                  Graphite    Projectional Editors    TLMs     Live Palette Expressions
interactive       Y           Y                       N        Y
extensible        Y           N                       Y        Y
persistent        N           Y                       Y        Y
compositional     N           N                       Y        Y
live              N           N                       N        Y
formalized        N           N                       Y        Y
\end{lstlisting}
\caption{Related Work Table}
\label{fig:related-work}
\end{figure*}

\section{Discussion}

\parahead{User Interface Design}

\begin{itemize}

\item
%
our \Hazel{} and \sns{} prototypes demonstrated two simple approaches to palette
layout: inline and floating, respectively.
%
might also want user-controlled layout and pinning.

\end{itemize}

\parahead{Deriving Palettes from Type Definitions}

\begin{itemize}

\item \cite{SSS}

\end{itemize}

\section{Conclusion}

\begin{quote}
%
\textit{
%
``The arithmetical symbols are written diagrams and the geometrical figures are graphic formulas.''
%
}

\vspace{3pt}

\hfill{}--- David Hilbert~\cite{XXX}
\end{quote}



%\clearpage
\bibliography{references,all.short,hazel_NSF}

\clearpage
\appendix
% !TEX root = hazelnut-dynamics.tex

% \begin{figure}[p]
% \judgbox
%   {\pexpandSyn{\hGamma}{\pPhi}{\pexp}{\hexp}{\htyp}}
%   {$\pexp$ expands to $\hexp$ which synthesizes type $\htyp$}
% \begin{mathpar}
% \inferrule[SPEConst]{ }{
%   \pexpandSyn{\hGamma}{\pPhi}{c}{c}{b}
% }
% \and
% \inferrule[SPEAsc]{
%   \pexpandAna{\hGamma}{\pPhi}{\pexp}{\hexp}{\htyp}
% }{
%   \pexpandSyn{\hGamma}{\pPhi}{\pexp : \htyp}{\hexp : \htyp}{\htyp}
% }
% \and
% \inferrule[SPEVar]{
%   x : \htyp \in \hGamma
% }{
%   \pexpandSyn{\hGamma}{\pPhi}{x}{x}{\htyp}
% }
% \and
% \inferrule[SPELam]{
%   \pexpandSyn{\hGamma, x : \htyp_1}{\pPhi}{\pexp}{\hexp}{\htyp_2}
% }{
%   \pexpandSyn{\hGamma}{\pPhi}{\halam{x}{\htyp_1}{\pexp}}{\halam{x}{\htyp_1}{\hexp}}{\tarr{\htyp_1}{\htyp_2}}
% }
% \and
% \inferrule[SPEAp]{
%   \pexpandSyn{\hGamma}{\pPhi}{\pexp_1}{\hexp_1}{\htyp_1} \\
%   \arrmatch{\htyp_1}{\tarr{\htyp_2}{\htyp}} \\\\
%   \pexpandAna{\hGamma}{\pPhi}{\pexp_2}{\hexp_2}{\htyp_2}
% }{
%   \pexpandSyn{\hGamma}{\pPhi}{\hap{\pexp_1}{\pexp_2}}{\hap{\hexp_1}{\hexp_2}}{\htyp}
% }
% \and
% \inferrule[SPEHole]{ }{
%   \pexpandSyn{\hGamma}{\pPhi}{\hehole{\mvar}}{\hehole{\mvar}}{\tehole}
% }
% \and
% \inferrule[SPNEHole]{
%   \pexpandSyn{\hGamma}{\pPhi}{\pexp}{\hexp}{\htyp}
% }{
%   \pexpandSyn{\hGamma}{\pPhi}{\hhole{\pexp}{\mvar}}{\hhole{\hexp}{\mvar}}{\tehole}
% }
% \end{mathpar}

% \vsepRule

% \judgbox
%   {\pexpandAna{\hGamma}{\pPhi}{\pexp}{\hexp}{\htyp}}
%   {$\pexp$ expands to $\hexp$ which must analyze against type $\htyp$}
% \begin{mathpar}
% \inferrule[APELam]{
%   \arrmatch{\htyp}{\tarr{\htyp_1}{\htyp_2}} \\
%   \pexpandAna{\hGamma, x : \htyp_1}{\pPhi}{\pexp}{\hexp}{\htyp_2}
% }{
%   \pexpandAna{\hGamma}{\pPhi}{\hlam{x}{\pexp}}{\hlam{x}{\hexp}}{\htyp}
% }
% \and
% \inferrule[APESubsume]{
%   \pexpandSyn{\hGamma}{\pPhi}{\pexp}{\hexp}{\htyp'} \\
%   \tconsistent{\htyp}{\htyp'}
% }{
%   \pexpandAna{\hGamma}{\pPhi}{\pexp}{\hexp}{\htyp}
% }
% \end{mathpar}
% \CaptionLabel{Palette Expansion, remaining rules}{fig:palexpapndx}
% \label{fig:expandSyn}
% \label{fig:expandAna}
% \end{figure}


\end{document}
