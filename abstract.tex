% !TEX root = livelits-paper.tex

%% no citations in Abstract
%%
%% \cite{popl-paper}

\begin{abstract}
    Although textual program manipulation is powerful, 
    some types of expressions are more naturally represented and manipulated
    graphically.     
    Examples include expressions computing   
    colors,
    animation parameters,
    musical sequences,
    tabular data, 
    plot parameters,
    GUI widgets, 
    and many other domain-specific data structures.
    This paper introduces \emph{live literals}, or \emph{livelits}, which allow
    a programmer to fill holes of types like these by directly manipulating a corresponding    
    GUI embedded persistently into the code. 
    Livelits are compositional: the GUI can itself contain
    spliced expressions, which in turn can contain other livelits. Splices are typed, and the system 
    requires that the livelit treat each splice hygienically. 
    Livelits are also live: they can provide {immediate feedback} about the dynamic implications 
    of the programmer's choices 
    even when the spliced expressions mention bound variables, because the system gathers closures associated with the hole that the livelit is tasked with filling.
    This paper details the livelits mechanism with several non-trivial case studies. 
    It then defines a typed lambda calculus that captures the essence of the mechanism.     
    Our implementation extends Hazel, a web-based live programming environment able to assign
    static and dynamic meaning to programs with holes. 
\end{abstract}
