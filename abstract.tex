% !TEX root = palettes-paper.tex

%% no citations in Abstract
%%
%% \cite{popl-paper}

\begin{abstract}

We present \emph{nested palettes}, an approach to programming custom graphical
user interfaces (GUIs) that are used to build program expressions---to ``fill
holes'' within an incomplete program.
%
Nested palettes serve as an alternative and complementary code editing mechanism
to traditional text-editing.
%
Compared to other ``hybrid'' text-and-GUI code editors, nested palettes provide
several unique capabilities.
%
First, a palette persists even after user interactions produce a complete
expression that fills the hole, so that the programmer can return to the GUI to
make subsequent changes.
%
Second, palettes for smaller expression types can appear nested within those for
larger expressions, allowing palettes to be composed in complex ways.
%
Third, palettes have access to the execution environment---even when the overall
program is not yet complete---and can, thus, provide live feedback to the
programmer while working to fill an incomplete expression.

This paper investigates the design and implementation of nested palettes in
several steps.
%
First, we propose a formal framework for nested palettes within Hazelnut, a
foundational semantics for live programming that provides continuous code editor
services through a combination of structured edits and dynamic evaluation of
incomplete programs.
%
Second, to demonstrate the expressiveness of our framework, we describe several
example palettes, many of which are informed by a prior study that identified
desirable use cases for hybrid editors.
%
Lastly, we provide a prototype implementation of nested palettes within Hazel, a
programming environment based on the Hazelnut theory extended with palettes.
%
Based on our experience, we believe nested palettes are an extensible and
expressive approach for mixing text- and GUI-based code editing.

\end{abstract}
