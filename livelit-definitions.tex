\section{Livelit Definitions}\label{sec:livelit-definitions}

\begin{figure}
\begin{lstlisting}
  type Color = (.r Int, .g Int, .b Int, .a Int)

  livelit $color at Color {
    capture { }

    type Model = (.r SpliceRef, .g SpliceRef, .b SpliceRef, .a SpliceRef)
    let init : UpdateMonad(Model) = do 
      r <- new_splice(`Int`, Some(`0`))
      g <- new_splice(`Int`, Some(`0`))
      b <- new_splice(`Int`, Some(`0`))
      a <- new_splice(`Int`, Some(`100`))
      return (r, g, b, a)

    type Action = 
    | ClickOnColor(Color)
  
    let view : Model -> ViewMonad(Html(Action)) = 
      fun model -> do 
        (* determine a color to display to the client, if possible *)
        r_res <- eval_splice(model.r)
        g_res <- eval_splice(model.g)
        b_res <- eval_splice(model.b)
        a_res <- eval_splice(model.a)
        let cur_color : Color = 
          case (r_res, g_res, b_res, a_res) 
          | (Some(Val(IntLit(r))), 
             Some(Val(IntLit(g))), 
             Some(Val(IntLit(b))), 
             Some(Val(IntLit(a)))) -> 
            Some((r, g, b, a))
          | _ -> None
        in 
        
        r_editor <- editor(model.r, FixedWidth(20))
        g_editor <- editor(model.g, FixedWidth(20))
        b_editor <- editor(model.b, FixedWidth(20))
        a_editor <- editor(model.a, FixedWidth(20))
        
        (* ...render the UI using these splice editors... *)
      
    let update : Model -> Action -> UpdateMonad(Model) =
      fun model (ClickOnColor(r, g, b, a)) -> do 
        set_splice(model.r, IntLit(r))
        set_splice(model.g, IntLit(g))
        set_splice(model.b, IntLit(b))
        set_splice(model.a, IntLit(a))
        return (r, g, b, a)
    
    let expand : Model -> (Exp, List(SpliceRef)) = 
      fun model -> 
        (`fun r g b a -> (r, g, b, a)`, [model.r, model.g, model.b, model.a])
  }
\end{lstlisting}
\caption{Livelit Implementation: \li{\$color} (simplified for presentation)}
\label{fig:color-impl}
\end{figure}

\noindent
Let us now look in more detail at a livelit definition. Fig.~\ref{fig:color-impl}
outlines the \li{\$color} livelit from Fig.~\ref{fig:color}. (We omit certain 
incidental implementation details and use unimplemented syntactic sugar, including
Haskell-style \li{do} notation and simplified quasiquotation, for the sake of presentation.
We say more about the implementation in Sec.~\ref{sec:implementation}.)

Line 1 of Fig.~\ref{fig:color-impl} specifies the livelit's name, \li{\$color}, and its \emph{expansion type},
\li{Color}, which was discussed in Sec.~\ref{sec:livelit-expansion}.
The remainder of the definition is the livelit's implementation.
Livelits are implemented using a variation on the functional model-view-update
framework popularized by the Elm programming language\todo{cites}. We add a fourth component,
expansion generation. In addition, we add a simple monadic framework to provide the necessary  
interface between the livelit and the programming environment, discussed below, while retaining
a pure functional programming model.\todo{cite monads in Haskell}

\subsection{Model}\label{sec:model}
\todo{final check of line numbers}{}Line 2 of Fig.~\ref{fig:color-impl} specifies the livelit's \emph{model type},
here a labeled 4-tuple of \emph{splice references}, one for each of the four splices
that appear in the GUI in Fig.~\ref{fig:color}.
Models are persisted in the syntax tree of the program, so in practice,
 the system would check that the model type supports automatic serialization. 
 (Hazel is able to serialize all types because it is implemented as a 
 simple interpreter.)

The initial value of the model is specified starting on Line 4, 
which defines a command in the \li{UpdateMonad} that returns the 
initial model value after generating four new splice references
using the \li{new_splice} command:
\begin{lstlisting}[numbers=none]
  new_splice : (Typ, Option(Exp)) -> UpdateMonad(SpliceRef)
\end{lstlisting}
This command creates a fresh splice
of the given type and, optionally, with the given initial expression,
and returns a unique splice reference, which uniquely and abstractly identifies that splice.
Bolded types like \li{Typ}, \li{Exp}, \li{UpdateMonad}, and \li{SpliceRef}
are provided by the standard library. The \li{Typ} and \li{Exp} types 
encode the syntax of Hazel's types and expressions.
These two arguments are provided here in quasiquoted form, e.g. \li{`0`}\todo{cite something}. 
If no initial expression is provided, the splice is initialized with 
an empty hole.

The system checks that the provided type and initial content is valid 
assuming only the explicitly specified set of captured bindings on Line 3.
Here, we do not need to capture anything because \li{Int} is a built in type
and the initial content is closed. However, 
for splices of user-defined types, those types need to be explicitly captured.
Similarly, helper functions that are used in initial splices (and in the expansion,
discussed below) must be explicitly captured. 
This serves to ensure \emph{context independence} -- the livelit needs to make no 
assumptions about the typing context at use sites.
We use an explicit capture set, rather than implicitly capture of all bindings at the definition site, 
both to prevent inadvertent capture\todo{cite heathers paper}{} 
and to ensure that the capture set is part
of the interface of the livelit definition, which simplifies matters related to 
module export and determining paths to the captured bindings at use sites \cite{TLMs}.

\subsection{Action}
Line 11 defines the \li{Action} type for the \li{\$color} livelit, which 
specifies a single user-initiated action: clicking on a color using the user 
interface on the right side of Fig.~\ref{fig:color}. Actions are emitted
from event handlers (e.g. click handlers) defined in the computed \li{view}, 
and actions are consumed by the \li{update} function, causing a change in the model. 
Let us discuss each of these functions in turn.

\subsection{View}
The \li{view} function computes the view given a model and access to the commands in 
the \li{ViewMonad}. The computed view is a value of type \li{Html(Action)}.  
The type family \li{Html(a)} provides a simple immutable
encoding of an HTML tree, where the type parameter \li{a} is the type of actions that 
are emitted by event handlers that can be attached to elements of the tree, e.g.
\li{on_click} and so on. 
We elide the details of the particular user interface in Fig.~\ref{fig:color}
to focus on two key mechanisms exposed by \li{ViewMonad}: live evaluation 
and splice rendering.

\subsubsection{Live Evaluation}
In order to provide the client with live feedback about the dynamic implications 
of the programmer's choices, 
the view function can ask the system to evaluate a splice, or an encoded function of 
multiple splices as long as the function is well-typed using only the capture set, 
under the closure that the client has selected using one of the following
commands:
\begin{lstlisting}[numbers=none]
eval_splice : SpliceRef -> ViewMonad(Option(Result))
eval_fun    : (Exp, List(SpliceRef)) -> ViewMonad(Option(Result))
\end{lstlisting}

The \li{None} case 
arises when evaluation is not possible, either because no closures are available
or because the free variables in the selected closure overlap with the free 
variables in the provided expression. This occurs only when the user has selected
a closure that appears under a binder in the evaluation result (e.g. because 
the result is a lambda). 
We discuss this further in Sec.~\ref{sec:calculus-closure-collection}.

If a result is available, the \li{Result} type distinguishes two possibilities:
\begin{lstlisting}[numbers=none]
type Result = 
| Val(Exp)
| Indet(Exp)
\end{lstlisting}
The \li{Val} case arises when evaluation produces a true value, whereas the 
\li{Indet} case arises when evaluation results in an indeterminate expression,
i.e. an expression that cannot be further evaluated due to one or more holes 
in critical elimination positions \cite{HazelnutLive}.

In our example, the livelit determines a color to display to the client 
if all four splices evaluate to integer literals. Otherwise, there is not 
enough information to determine a color, and the livelit can decide how to 
communicate this information to the client, e.g. with an "X" over the color 
preview.

Livelits can attempt to offer feedback even when the result is indeterminate,
because indeterminate expressions might nevertheless contain useful information.
For example, a livelit that previews a sequence of notes as audio might be able 
to handle a list of notes where certain notes are missing, i.e. holes, by 
simply playing silence or some other default sound when it encounters them.
This behavior is highly domain-specific, so each livelit provider must decide 
whether and how indeterminate results are supported.

\subsubsection{Splice Rendering}
The remainder of the \li{view} function is responsible for computing the view.
This is straightforward but for the question of splices. In Fig.~\ref{fig:color},
we want the splices to render as expression editors, with all of Hazel's editor 
services available to the user. In order to support this, the \li{view} function
can request an editor for a given splice, with given dimensions:
\begin{lstlisting}[numbers=none]
editor : (SpliceRef, Dimensions) -> ViewMonad(Html(a))
\end{lstlisting}
The result is an opaque \li{Html(a)} value that the remainder of the function 
can place as needed. When the livelit is rendered, this part of the tree is 
under the control of Hazel. The \li{Dimensions} parameter currently supports only a fixed
character width, with overflow causing scrolling, but in the future we plan to offer 
to offer more flexible layout options.

Another situation that can arise is when the UI needs to include not an editor but 
a rendered evaluation result. For example, each of the cells in the \li{\$dataframe}
livelit in Fig.~\ref{fig:grading} shows the evaluation result for the corresponding 
cell. Only the formula bar at the top is an editor. To support this, the \li{view}
function can use the \li{result_view} commands, which mirror the \li{eval} commands:
\begin{lstlisting}[numbers=none]
result_view_splice : (SpliceRef, Dimensions) -> ViewMonad(Html(a))
result_view_fun    : (Exp, List(SpliceRef), Dimensions) -> ViewMonad(Html(a))
\end{lstlisting}


\subsection{Update}
When the user triggers an event in a livelit view, it emits an \li{Action}.
The system responds by calling the \li{update} function to determine how 
this action should affect the model and the splices. 

We discussed in Sec.~\ref{sec:model} that the \li{UpdateMonad}
allows new splices to be generated in response to user actions. 
For example, the \li{\$dataframe} livelit uses this when new rows
or columns are added.

In Fig.~\ref{fig:color-impl}, we see that the \li{\$color} livelit responds to 
the \li{ClickOnColor} action by invoking the \li{set_splice} command to overwrite 
the current splices with integer literals determined based on which color the user
clicked on:
\begin{lstlisting}[numbers=none]
  set_splice : (SpliceRef, Exp) -> UpdateMonad(Unit)
\end{lstlisting}
The provided expression must satisfy the splice type and only make use of the capture set.

When the model is updated, a new view is 
computed. The system then performs a diff between the old and new view in order to 
efficiently perform the necessary imperative updates to the editor's visual state.
Changes to splices can also cause the view to be recomputed, because the view might 
need to evaluate the splices\todo{optimizaiton?}. The \li{update} function does not itself 
have the ability to request evaluation, because the model should not depend directly  
which closure the user has selected. Of course, the \li{view} might emit 
result-dependent actions when appropriate.

\subsection{Expansion}
The ultimate purpose of a livelit is to fill the hole where it appears by generating an expansion,
i.e. a standard symbolic expression of the expansion type, here \li{Color}.
The \li{expand} function, shown in Fig.~\ref{fig:color-impl}, is responsible for generating 
an expansion given the current model.

The expansion can include splices, but we the system does not make the contents of a splice 
available directly. Instead, the \li{expand} must return a pair consisting of an encoded expression, of type 
\li{Exp}, with a list of \li{SpliceRef}s that the expansion depends on. 
We call the \li{Exp} the \emph{parameterized expansion}
 because it must take an argument for each listed \li{SpliceRef}s. 
 That argument will be the corresponding splice 
type, which was provided when the splice was initialized. 
 The return type of the parameterized expansion is the expansion type, here \li{Color}.

This parameterization strategy makes it simple to enforce hygiene: the parameterized expansion 
can depend only on the capture set, whereas the splices are entered by the client and so they can 
depend only on the client site typing context. The use of standard function application ensures
that splices are capture avoiding. We consider this more formally in the next section.
