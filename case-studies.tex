\section{Case Studies}\label{sec:case-studies}
This section introduces livelits from the client programmer's perspective by way of 
two domain-specific case studies:
a course grading case study in Sec.~\ref{sec:live-grading} 
and an image transformation case study in Sec.~\ref{sec:image-transformation}. 
We also briefly provide several other examples in Sec.~\ref{sec:additional-examples}. 
These case studies and examples have been implemented\todo{make this not a lie}{}
in Hazel, a browser-based live programming environment for a pure typed functional language in the
ML family, with which we assume basic familiarity. 
We will describe the implementation of livelits in Hazel, and more generally, in Sec.~\ref{sec:implementation}.

\subsection{Case Study: Grading with Livelits}\label{sec:live-grading}
\begin{figure}
\begin{lstlisting}
let grades = $dataframe in 
            A1          A2          A3          A4       Midterm      Final  +
  Cyrus
  Nick
  David
    + 
let averages = compute_averages(grades) in 
let cutoffs = $cutoffs(averages) in 

  |---------------------------------------------------|

assign_grades(averages, cutoffs)
\end{lstlisting}
\caption{Grading Example}
\label{fig:grading}
\end{figure}

\subsubsection{Overview}\label{sec:live-grading-overview}
Talk about the scenario

Live matrix: what is a matrix structurally? (data frame? list of lists?) + how does liveness work
+ overview of hygiene guarantees

Grade cutoffs: how does it work wrt liveness? (do we use a parameter here instead of a splice?) -- 
focus on domain-specific benefits here.

\subsection{Case Study: Image Transformations}\label{sec:image-transformation}
Show example of an image transformation pipeline going through livelits

This might also benefit from Parameterization

Show multiple calls with different example images + closure selector UIs

Go into more detail about how evaluation works + fill-and-resume mechanics (and efficiency nod)

Talk about probes?

\subsection{Additional Examples}\label{sec:additional-examples}
It would be nice to have a gallery-style figure and a brief overview of some other case studies
and how they exercise the novel features of the livelits mechanism. Maybe some statistics on how
many lines of code it took.
