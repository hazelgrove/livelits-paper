\section{A Simply Typed Livelit Calculus}\label{sec:livelit-calculus}


In order to communicate the semantics of livelits
independently
of the specifics of the Hazel environment and the web platform, 
we now specify a \emph{simply typed livelit calculus}.

Fig.~\ref{fig:syntax} specifies the syntax of the livelit calculus.
Programs are written as \emph{unexpanded expressions}, $\hat e$, which are \emph{expanded} to
\emph{external }(or \emph{expanded}) \emph{expressions}, $e$, before being \emph{elaborated}
to \emph{internal expressions}, $d$, for evaluation. All three sorts are classified
by the same types, $\tau$. We include partial functions, products, sums, and recursive
types, all in their standard form \cite{pfpl}, but this specific type structure is not critical.
Any language expressive
enough to encode its own abstract syntax 
would be a suitable basis for a livelit calculus. 

We begin in Sec.~\ref{sec:external-and-internal-lang} with a terse
overview of the external and internal languages,
which are adapted from Hazelnut Live \cite{HazelnutLive}.
We then detail livelit expansion (Sec.~\ref{sec:calculus-expansion}) and 
liveness via closure collection (Sec.~\ref{sec:calculus-closure-collection}). 
% and parameterization (Sec.~\ref{sec:calculus-parameterization}).


\subsection{Background: External and Internal Language}\label{sec:external-and-internal-lang}
The external and internal languages are straightforward adaptations of the
external and internal languages of \emph{Hazelnut Live},
a typed lambda calculus that assigns static and dynamic meaning to programs with holes,
notated $\hehole{u}$ where $u$ is a \emph{hole name} \cite{HazelnutLive}.
We omit non-empty holes (which internalize type inconsistencies \cite{Hazelnut}) and type holes
(which operate like the unknown type from gradual type theory \cite{Siek06a,Hazelnut}).
These mechanisms are orthogonal to livelits and are included in our implementation.

External expressions, e, are governed by a typing judgement of the form, $\hasType{\Gamma}{e}{\tau}$,
where the typing context, $\Gamma$,
is a finite set of typing assumptions of the form $x : \tau$ \cite{pfpl}.
\begin{figure}
    \vspace{-3px}
    \[
    \arraycolsep=3pt\begin{array}{rlcl}
        \mathsf{Typ} & \tau & ::= &
                                    % \tnum ~\vert~
                                    \tarr{\tau_1}{\tau_2} ~\vert~
                                    \tprod{\tau_1}{\tau_2} ~\vert~
                                    \tunit ~\vert~
                                    \tsum{\tau_1}{\tau_2} ~\vert~
                                    t ~\vert~
                                    \trec{t}{\tau}\\
        \mathsf{UExp} & \hat{e} & ::= & 
                                 x ~\vert~
                                 \hlam{x}{\hat e} ~\vert~
                                 \hap{\hat e_1}{\hat e_2} ~\vert~
                                 ... ~\vert~
                                %  \hpair{\hat e_1}{\hat e_2} ~\vert~
                                %  \hprl{\hat e} ~\vert~
                                %  \hprr{\hat e} ~\vert~
                                %  \htriv ~\vert~
                                %  \hinL{\hat e} ~\vert~
                                %  \hinR{\hat e} ~\vert~
                                %  \hroll{\hat e} ~\vert~
                                %  \hunroll{\hat e} \\
                                 \hehole{u} ~\vert~
                                 \haplivelit{u}{a}{\dtxt{model}}{\splices}\\
        \mathsf{EExp} & e & ::= & x ~\vert~ \hlam{x}{e} ~\vert~ \hap{e_1}{e_2} ~\vert~ ... ~\vert~ \hehole{u}\\
        \mathsf{IExp} & d & ::= & x ~\vert~ \hlam{x}{d} ~\vert~ \hap{d_1}{d_2} ~\vert~ ... ~\vert~ \dehole{u}{\sigma}\\
        \mathsf{Splice} & \psi & ::= & \hsplice{\hat e}{\tau}
    \end{array}\]
    \caption{Syntax of types, $\tau$, unexpanded expressions, $\hat{e}$, external expressions, $e$, and internal expressions, $d$.
    Here, $x$ ranges over variables, $u$ over hole names, and $\livelitname{a}$ over livelit names.
    We write $\splat{\psi_i}$ for a finite sequence of $n \geq 0$ splices,
    and $\sigma$ for finite substitutions of $n \geq 0$ internal expressions for variables, $[d_i/x_i]_{i < n}$.
    We elide standard forms
    related to product, sum, and recursive types.
    }
    \label{fig:syntax}
    \end{figure}

The internal language is a contextual type theory \cite{Nanevski2008}, i.e. the typing judgement is
of the form $\hasTypeD{\Delta}{\Gamma}{d}{\tau}$ where the hole context $\Delta$ is a finite set of hole typing
assumptions of the form $u :: \tau[\Gamma]$ which mean that hole $u$ must be filled with an expression of type $\tau$ under $\Gamma$.
We need a hole typing context only for the internal language because, although hole names are assumed
unique in the external language, they can be duplicated during evaluation of internal expressions.
% The hole context ensures that each closure associated with hole $u$, written $\dehole{u}{\sigma}$,
% shares a type and can be filled
% with expressions valid under a shared context.

\newcommand{\tnat}{\mathsf{nat}}

External expressions elaborate to internal expressions, $d$,
by the judgement $\elabs{\Gamma}{e}{d}{\tau}{\Delta}$.
The main purpose of elaboration is to initialize the substitution $\sigma$ on each hole closure, $\dehole{u}{\sigma}$, 
which takes the form $[d_i/x_i]_{i<n}$ and serves to capture
the substitutions that occur around the hole during evaluation. The key rule is:
\begin{mathpar}
\inferrule[Elab-Hole]{ }{
    \elabs{\Gamma}{\hehole{u}}{\dehole{u}{\idof{\Gamma}}}{\tau}{u :: \tau[\Gamma]}
}
\end{mathpar}
In this declarative formulation, the type $\tau$ is arbitrary (in Hazelnut Live, elaboration is specified algorithmically \cite{HazelnutLive}).
The substitution is initially the identity substitution, $\idof{\Gamma}$, i.e. the
substitution that maps each variable in $\Gamma$ to itself, because no substitutions have yet occurred. For example,
\[ \elabs{ }{(\hlam{x}{\hehole{u}})~\hnum{5}}{(\hlam{x}{\dehole{u}{[x/x]}})~{\hnum{5}}}{\tnat}{u :: \tnum[x : \tnat]} \]


During evaluation, $\evalsTo{d}{d'}$, the closure's substitution accumulates the substitutions that occur. For example,
the internal expression above evaluates as follows:
\[
  \evalsTo{(\hlam{x}{\dehole{u}{[x/x]}})~{\hnum{5}}}{
      \dehole{u}{[\hnum{5}/x]}
  }
\]
% We will use hole closures to support live programming with livelits in Sec.~\ref{sec:calculus-closure-collection}.


Rather than restating the remaining rules, we simply state the key governing
metatheorems and defer to the prior work and our Agda mechanization (Sec.~\ref{sec:agda}) for the details \cite{HazelnutLive}.

Elaboration preserves typing.
\begin{theorem}[Typed Elaboration]
    If $\hasType{\Gamma}{e}{\tau}$ then $\elabs{\Gamma}{e}{d}{\tau}{\Delta}$ for some $d$ and $\Delta$ such
    that $\hasTypeD{\Delta}{\Gamma}{d}{\tau}$.
\end{theorem}

Evaluation of a closed well-typed expression with holes results in a final (i.e. irreducible) expression of the same type.
\begin{theorem}[Preservation]
    If $\hasTypeD{\Delta}{\cdot}{d}{\tau}$ and $\evalsTo{d}{d'}$ then $\isFinal{d'}$ and $\hasTypeD{\Delta}{\cdot}{d'}{\tau}$.
\end{theorem}

\subsection{Expansion}\label{sec:calculus-expansion}

The novelty of the livelit calculus is entirely in its handling of unexpanded expressions,
$\hat e$, which are given meaning by typed expansion to external expressions,
$e$, according to the judgement $\expands{\Phi}{\Gamma}{\hat e}{e}{\tau}$
defined in Fig.~\ref{fig:expansion}.
Unexpanded expressions mirror external expressions but for the presence of livelit invocations.
The rules for the mirrored 
forms like \rulename{EVar} and \rulename{EFun}, both shown in Fig.~\ref{fig:expansion},
are straightforward.

\subsubsection{Livelit Contexts}\label{sec:livelit-contexts}

Livelit definitions are collected in the livelit context, $\Phi$, which
maps livelit names $\livelitname{a}$ to livelit definitions of the following form:
\[ \livelitCtxEntry{a}{\taut{expand}}{\taut{model}}{\dtxt{expand}} \]
Here, $\taut{expand}$ is the expansion type, $\taut{model}$ is the model type,
and $\dtxt{expand}$ is the expansion function, which generates an expansion given a model.
We omit the logic related to view computations and actions, which are tied to a particular 
UI framework and have only indirect semantic significance.

% For a livelit context to be well-formed, we require that each livelit definition be well-formed.
% This in turn requires that the expansion function be hole-free and have the correct type.
\begin{definition}[Livelit Context Well-Formedness]
    A livelit context $\Phi$ is well-formed if and only if for each livelit definition,
    $\livelitCtxEntry{a}{\taut{expand}}{\taut{model}}{\dtxt{expand}} \in \Phi$,  we have
    $\hasType{}{\dtxt{expand}}{\tarr{\taut{model}}{\expType}}$.
\end{definition}
Here, $\expType$ stands for a type whose values isomorphically encode
external expressions. The isomorphism is mediated in one direction by
the encoding judgement $\encodeExp{e}{d}$ and the other by the decoding judgement $\decodeExp{d}{e}$.
Any scheme is sufficient, so we leave it as a matter of implementation.
The simplest approach is to define $\expType$ as a recursive sum type,
with one arm for each form of external expression (cf. \cite{TSLs}).

For simplicity, we assume that the livelit context is provided \emph{a priori} and therefore
that the expansion function is already closed and fully elaborated.
In practice, the livelit context would be controlled by a definition form in the language
that allows the definition to itself invoke livelits.
This would require a staging mechanism,
because we need to evaluate expansion functions in prior definitions to be able to
expand subsequent definitions.
There are a number of ways to support the necessary staging in practice, e.g.
via explicit staging primitives \cite{DBLP:conf/icfp/Flatt02},
by requiring that these definitions appear in separately compiled packages \cite{TLMs},
or by using live
programming mechanisms such as those in Hazel
to evaluate ``up to'' each definition before proceeding \cite{HazelnutLive}.
% This choice is orthogonal to the mechanisms of interest in this section.

\subsubsection{Livelit Expansion}
\begin{figure}
    \begin{mathpar}
        \inferrule[EVar]{
            x : \tau \in \Gamma
        }{
            \expands{\Phi}{\Gamma}{x}{x}{\tau}
        }
        \hspace{15px} 
        \inferrule[ELam]{
            \expands{\Phi}{\Gamma, x : \taut{in}}{\hat e}{e}{\taut{out}}
        }{
            \expands{\Phi}{\Gamma}{\hlam{x}{\hat e}}{\hlam{x}{e}}{\tarr{\taut{in}}{\taut{out}}}
        }\hspace{15px}\cdots

        % \inferrule[EAp]{
        %     \expands{\Phi}{\Gamma}{\hat e_1}{e_1}{\tarr{\taut{in}}{\taut{out}}}\\\\
        %     \expands{\Phi}{\Gamma}{\hat e_2}{e_2}{\taut{in}}
        % }{
        %     \expands{\Phi}{\Gamma}{\hap{\hat e_1}{\hat e_2}}{\hap{e_1}{e_2}}{\taut{out}}
        % }
        \inferrule[ELivelit]{
            \livelitCtxEntry{a}{\taut{expand}}{\taut{model}}{\dtxt{expand}} \in \Phi\\\\
            \hasType{ }{\dtxt{model}}{\taut{model}}\\
            \evalsTo{\hap{\dtxt{expand}}{\dtxt{model}}}{\dtxt{encoded}}\\
            \decodeExp{\dtxt{encoded}}{\etxt{pexpansion}}\\\\
            \hasType{ }{\etxt{pexpansion}}{\tarr{\{\tau_i\}_{i < n}}{\taut{expand}}}\\
            \{ \expands{\Phi}{\Gamma}{\hat e_i}{e_i}{\tau_i} \}_{i < n}
        }{
            \expands{\Phi}{\Gamma}{\haplivelit{u}{a}{\dtxt{model}}{\splat{\hsplice{\hat e_i}{\tau_i}}}}{
                \hap{\etxt{pexpansion}}{\{e_i\}_{i < n}}
            }{\taut{expand}}
        }
    \end{mathpar}
    \caption{Expansion}
    \label{fig:expansion}
    \end{figure}
Unexpanded expressions include livelit invocations:
 \[\haplivelit{u}{a}{\dtxt{model}}{\splices}\]
Here, $\livelitname{a}$ names the livelit
 being invoked. Livelits can be understood as filling holes, so $u$ identifies the hole
 that is, conceptually, being filled.
The current state of the livelit is determined by the current model value, $\dtxt{model}$,
together with the splice list, $\splices$. Each splice $\psi_i$
is of the form $\hsplice{\hat e_i}{\tau_i}$, where $\hat e_i$ is the spliced expression
itself (unexpanded, so it may contain other livelits) and $\tau_i$ is the type of that splice,
as determined when the livelit definition first requested the splice (as discussed in Sec.~\ref{sec:model}).  
Note that we do not formally model the edit actions that change the model or splice list here;
we focus on a single snapshot of the editor state.

Rule \rulename{ELivelit} performs livelit expansion. Its premises, in order, operate as follows:
\begin{enumerate}[itemsep=3px,leftmargin=*]
    \item \textbf{Lookup.} The first premise looks up the livelit definition in the livelit context.
    \item \textbf{Model Validation.} The second premise checks that the model value, $\dtxt{model}$, is of the
    specified model type, $\taut{model}$.
    \item \textbf{Expansion.} The third premise applies the expansion function, $\dtxt{expand}$, to the model value, $\dtxt{model}$,
    producing the \emph{encoded parameterized expansion}, $\dtxt{encoded}$, which, by the definitions and theorems given above, is of type $\mathsf{Exp}$.
    \item \textbf{Decoding.} The fourth premise decodes $\dtxt{encoded}$, producing the \emph{parameterized expansion}, $\etxt{pexpansion}$. The isomorphism between encodings and external expressions ensures that decoding cannot fail.
    \item \textbf{Expansion Validation.} The fifth premise checks that {parameterized expansion} is a function that returns a value of the expansion type, $\taut{expand}$, when applied (in curried fashion, though this is not critical)
    to the splices, whose types, $\splat{\tau_i}$, are given in the splice list.
    Validation can fail, in which case expansion fails. (In Hazel, validation failure is instead marked by a non-empty hole.)
    
    We ensure that the parameterized expansion is \emph{context independent},
    i.e. that it cannot depend on the particular bindings available in the call site typing context, $\Gamma$, 
    % Context independence is one facet of what is colloquially known as \emph{hygiene} \cite{TLMs}.
     by
    requiring that the parameterized expansion be
    closed. Consequently, any necessary helper functions used in the expansion must be provided by the client
    via a splice. 

    In Sec.~\ref{sec:livelit-definitions}, we discussed how the use of an explicit definition site \li{context} eliminates this client burden.
    The explicit context is formally just a value (tupled, typically) that can be passed as an additional argument to the parameterized expansion alongside the splices, 
    so we omit it from the calculus, but see \cite{TLMs} for a full treatment.
    
    \item \textbf{Splice Expansion.} The sixth premise inductively expands each of the spliced expressions in the same context as the livelit
    invocation itself.
\end{enumerate}
The conclusion of the rule then applies the parameterized expansion to the expanded splices.
By applying the splices as arguments, we maintain \emph{capture avoidance} -- splices cannot capture variables
bound internally to the expansion because beta reduction performs capture-avoiding substitution. 
% Capture avoidance is the other facet of \emph{hygiene} \cite{TLMs}.

The typed expansion process is governed by the following metatheorem, which establishes that the expansion
is indeed an external expression of the indicated type (i.e. the expansion type, in the case of livelit invocations).
\begin{theorem}[Typed Expansion]
    If $\expands{\Phi}{\Gamma}{\hat e}{e}{\tau}$ then $\hasType{\Gamma}{e}{\tau}$.
\end{theorem}
% \begin{proof}
%     We proceed by rule induction on the assumption.
%     The cases involving the standard forms follow by straightforward induction.
%     The only interesting case is the \rulename{ELivelit} case.
%     In this case, we have that $e = \hap{\etxt{pexpansion}}{\splat{e_i}}$.
%     By assumption, we have $\{ \expands{\Phi}{\Gamma}{\hat e_i}{e_i}{\tau_i} \}_{i < n}$.
%     By induction, we therefore have $\{ \hasType{\Gamma}{e_i}{\tau_i} \}_{i < n}$.
%     In addition, by assumption we have that $\hasType{}{\etxt{pexpansion}}{\tarr{\splat{\tau_i}}{\taut{expand}}}$.
%     By weakening, we therefore have $\hasType{\Gamma}{\etxt{pexpansion}}{\tarr{\splat{\tau_i}}{\taut{expand}}}$.
%     Finally, by iterated application of the typing rule for function application over the splices,
%     using the judgements just established,
%     we have $\hasType{\Gamma}{\hap{\etxt{pexpansion}}{\splat{e_i}}}{\taut{expand}}$ as desired.
% \end{proof}

When composed with the Typed Elaboration theorem and the type safety of the internal
language, we achieve end-to-end type safety: every well-typed unexpanded
expression expands to a well-typed external expression, which in turn elaborates to a well-typed internal
expression, which in turn evaluates in a type safe manner.

\subsubsection{Agda Mechanization}
\label{sec:agda}
The supplement mechanically proves these theorems 
using the Agda proof assistant, building on the Agda mechanization 
of Hazelnut Live \cite{HazelnutLive}.


\subsection{Live Feedback via Closure Collection}\label{sec:calculus-closure-collection}
In order to support live feedback, a livelit needs to be able to ask the system
to evaluate expressions under one of the closures associated with the livelit.
This mechanism was introduced by example in Sec.~\ref{sec:closure-collection-example}.
In this section, we will formalize the process of efficiently collecting closures.

\subsubsection{Proto-Environment Collection}
We begin by generating an alternative expansion,
called the \emph{cc-expansion},
where each livelit invocation expands to an empty hole applied to its splices. In other words,
a hole appears in place of the parameterized expansion (but not splices). On the side, we generate a \emph{cc-context}, $\Omega$,  that maps each livelit hole to the \emph{elaboration} of its parameterized expansion, $\Omegaitem{u}{\dtxt{pexpansion}}$.
The key rule for cc-expansion, $\ccexpands{\Phi}{\Gamma}{\hat e}{e}{\tau}{\Omega}$, is:
\begin{mathpar}
\inferrule[CCLivelit]{
    \{ \ccexpands{\Phi}{\Gamma}{\hat e_i}{e_i}{\tau_i}{\Omega_i} \}_{i < n}\\
    \expands{\Phi}{\Gamma}{\haplivelit{u}{a}{\dtxt{model}}{\splat{\hsplice{\hat e_i}{\tau_i}}}}{
        \hap{\etxt{pexpansion}}{\splat{e'_i}}
    }{\taut{expand}}\\
    \elabs{ }{\etxt{pexpansion}}{\dtxt{pexpansion}}{\tarr{\splat{\tau_i}}{\taut{expand}}}{ }\\
    \Omega = \cup_{i<n} \Omega_i \cup \{\Omegaitem{u}{\dtxt{pexpansion}}\}
}{
    \ccexpands{\Phi}{\Gamma}{\haplivelit{u}{a}{\dtxt{model}}{\splat{\hsplice{\hat e_i}{\tau_i}}}}{\hap{\hehole{u}}{\splat{e_i}}}{\taut{expand}}
    {\Omega}
}
\end{mathpar}

We then elaborate and evaluate the cc-expansion. 
The result will contain some number of hole closures
for each livelit hole. We call these the proto-closures and their environments the proto-environments for that livelit hole.
\begin{definition}[Proto-Closure Collection]
If $\ccexpands{\Phi}{\cdot}{\hat e}{e}{\tau}{\Omega}$ and $\elabs{}{e}{d}{\tau}{\Delta}$
and $\evalsTo{d}{d'}$ and $u \in \domof{\Omega}$ then $\protoEnvsOf{\Phi}{\hat e}{u} = \{ \sigma \mid \dehole{u}{\sigma} \in d' \}$.
    % protoclosures(\hat e, u) = { sigma | hole^u_sigma in d' } where \hat e cc-expands to e and e elaborates to d and d evaluates to d'
\end{definition}

\subsubsection{Closure Resumption}
A proto-environment for a livelit hole might itself contain a proto-closure for another livelit hole,
which is problematic for the reasons detailed in Sec.~\ref{sec:closure-collection-example}.
% For example, in Fig.~\ref{fig:color}, the proto-environment for the \li{\$color} livelit contained the
% proto-closure for the \li{\$slider} livelit, via the \li{baseline} variable.
% If we use proto-environments for live evaluation, then the actual value of that variable
% would not be available (here, the color could not be displayed).
Consequently, the second step of closure collection, called \emph{closure resumption}, is to fill any livelit holes that appear in the proto-environments for other livelit holes.
We do so by filling them using the parameterized expansions gathered in $\Omega$ and then resuming
evaluation where appropriate.
Formally, this involves the hole filling operation $\instantiate{d_1}{u}{d_2}$ for Hazelnut Live
(which derives from the metavariable instantation operation of contextual modal type theory \cite{HazelnutLive,Nanevski2008}).
This operation
fills every closure for hole $u$ in $d_2$ with $d_1$.
Unlike substitution, hole filling is
not capture-avoiding. Instead, the environment on each of these closures is applied to $d_1$
as a substitution, i.e. the delayed substitutions captured in the environment are realized.
In this case, however, the parameterized expansion is necessarily closed due to
the context independence discipline we maintain in Rule \rulename{ELivelit},
so hole filling amounts to syntactic replacement.

Formally, we begin by defining an operation $\fillof{\Omega}{\sigma}$ which acts on proto-environments
to fill the livelit holes.

\begin{definition}[Livelit Hole Filling] ~
    \begin{enumerate}
        \item $\fillof{\Omega}{[d_1/x_1, \ldots, d_n/x_n]} = $\\
        $\hspace{10px}[\fillof{\Omega}{d_1}/x_1, \ldots, \fillof{\Omega}{d_n}/x_n]$
        \item $\fillof{\Omega}{d} = \instantiateB{\dtxt{pexpansion}}{u}_{\Omegaitem{u}{\dtxt{pexpansion}} \in \Omega}{d}$
    \end{enumerate}
\end{definition}

This first step may cause certain expressions to become non-final, because the filled hole is no longer
blocking evaluation. We therefore define an operation $\resumeof{\sigma}$ that resumes evalution for all closed expressions in $\sigma$.
(The only open expressions that might remain are the initial variables from the identity substitution generated by elaboration. Some closures appear under binders in the final result, so these variables will not have yet recorded a substitution.)
\begin{definition}[Environment Resumption] ~
    \begin{enumerate}
        \item $\resumeof{[d_1/x_1, \ldots, d_n/x_n]} = $\\
        $[\resumeof{d_1}/x_1, \ldots, \resumeof{d_n}/x_n]$
        \item $\resumeof{d} = d'$ if $\fvof{d} = \emptyset$ and $\evalsTo{d}{d'}$
        \item $\resumeof{d} = d$ if $\fvof{d} \neq \emptyset$
    \end{enumerate}
\end{definition}

Finally, we can produce the final set of environments by filling and resuming the proto-environments.
\begin{definition}[Environment Collection]
    If $\ccexpands{\Phi}{\cdot}{\hat e}{e}{\tau}{\Omega}$
    % and $\elabs{}{e}{d}{\tau}{\Delta}$
    % and $\evalsTo{d}{d'}$ and $u \in \domof{\Omega}$
    then \[\envsOf{\Phi}{\hat e}{u} = \{\resumeof{\fillof{\Omega}{\sigma}} \mid \sigma \in \protoEnvsOf{\Phi}{\hat e}{u}\}\]
\end{definition}

This same fill and resume operation can be used to avoid recomputation when evaluating the fully expanded version of the user's program.
If the editor has already performed environment collection, then it can simply continue from where it left off
by filling and resuming
the remaining top-level livelit holes (those that do not appear in a proto-environment).

The correctness of the mechanisms described in this section rest on the fact that evaluation commutes with hole filling
in the pure setting.
% This only holds when the language is pure: evaluation order does not matter
% so the system is free to evaluate around holes initially,
% and then return to finish the job once they have been filled.

\begin{theorem}[Post-Collection Resumption]
    If $\ccexpands{\Phi}{\cdot}{\hat e}{\etxt{cc}}{\tau}{\Omega}$ and $\elabs{}{\etxt{cc}}{\dtxt{cc}}{\tau}{\Delta}$
    and $\evalsTo{\dtxt{cc}}{\dtxt{cc}'}$ and $\resumeof{\fillof{\Omega}{\dtxt{cc}'}} = d_1$
    and $\expands{\Phi}{\cdot}{\hat e}{\etxt{full}}{\tau}$
    and $\elabs{}{\etxt{full}}{\dtxt{full}}{\tau}{\Delta}$
    and $\evalsTo{\dtxt{full}}{d_2}$ then $d_1 = d_2$.
\end{theorem}
\begin{proof}
    The key observation is that filling the livelit holes in the cc-expansion gives the full expansion,
    i.e. $\fillof{\Omega}{\dtxt{cc}} = \dtxt{full}$. Resumption is simply evaluation for closed expressions.
    By commutativity of hole filling, established in the prior work \cite{HazelnutLive},
    we can delay hole filling until $\dtxt{cc}$ has first been evaluated to
    $\dtxt{cc}'$.
\end{proof}

In an imperative language with non-commutative side effects, resumption is not sound. We conjecture that one can specify an 
alternative evaluation mode where the full expansion of each livelit invocation is evaluated 
in the order that corresponding livelit hole is encountered during evaluation, with the closure recorded in a memory. 
This would ensure that
side effects happen in the same order and only once. 
% It would also allow substitutions that only occur conditionally on another livelit invocation's value to be recorded (as specified, these dependencies). 
%A formal specification is left as future work.
Alternatively, an imperative language might use a different mechanism to make
environments available to livelits, e.g. by environment snapshotting \emph{a la} Lamdu \cite{lamdu}.
This is more limited in situations where a livelit invocation appears in a function which has yet to be applied
 (where collected livelit closures essentially 
 capture the function's closure). 

Livelits invocations in branches that are not taken or that appear under unused variables do not have closures collected for them with our approach
(though in practice there is an implicit ``trailing hole'' after each cell in Hazel which means the latter situation is infrequent). 
Livelit invocations that determine which of several branches are taken prevent substitutions conditional on those branches from being recorded in 
downstream closures because resumption operates only within the proto-closures. 
This design gives clients the ability to control closure collection in branches that are not of interest (and therefore control the cost of closure collection), 
but may sometimes cause unexpected behavior. 
We leave further exploration of this design space 
to future work.

% \subsection{Parameterization}

% Add livelit expressions, livelit abbreviations,
% partial application support. Talk about how to work around the strict context
% independence discipline we have imposed.

% \subsection{Edit Action Semantics}
% So far, we have considered only the semantics of an individual editor state.
% However, livelits are useful because they evolve as the user interacts
% with them. In particular, the livelit's model, $\dtxt{model}$ can evolve
% due to livelit-specific user interactions. Furthermore, a livelit can also
% create and delete splices when a user interaction demands it (unlike functions, which have a fixed
% number of arguments).

