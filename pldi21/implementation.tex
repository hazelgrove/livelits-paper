\section{Implementation}\label{sec:implementation}
Implementing livelits requires tight integration between a rich editor, 
a type checker, and a live evaluator 
capable of evaluating incomplete programs and gathering hole closures.

\subsection{Hazel}
Hazelnut Live is the foundation of the Hazel programming environment, 
and Hazel has support for all of the necessary mechanisms, so 
it was natural to choose Hazel for our primary implementation.
Hazel is implemented in OCaml and compiled to JavaScript using the \li{js_of_ocaml} compiler \cite{DBLP:conf/aplas/RadanneVB16}.
% The core of Hazel is written in a pure functional style, so our implementation closely 
% follows the formalism.  
%  Hazel is a browser-based live functional programming environment 
% organized from the ground up around typed holes, with full . 

The livelit definition mechanism described in Sec.~\ref{sec:livelit-definitions}
is implemented in Hazel, albeit without some of the syntactic sugar in Fig.~\ref{fig:color-impl}.
This, together with the fact that Hazel lacks a mature GUI widget libraries as of this writing, 
makes complex examples tedious to implement within Hazel, so we also 
added the ability to define livelits using JavaScript or OCaml. These are loaded when 
Hazel is compiled. As Hazel evolves, we expect to need to define such ``primitive'' livelits less 
frequently, using them mainly for livelits that would benefit from access to established JavaScript 
or OCaml libraries.

Uniquely, every editor state in Hazel is 
semantically meaningful: it has a type, it can be evaluated, and it can be transformed 
in a type-aware manner. This implies that livelits remain fully functional at all times, 
even when the program is incomplete or erroneous. Hazel achieves this ``gap-free'' liveness guarantee by automatically inserting explicit holes as necessary 
while the user edits 
the program. Formally, Hazel is a type-aware structure editor \cite{Hazelnut}, rather than a text editor, 
although the developers are aiming to maintain a text-like experience (this effort is orthogonal to our own). 

% The examples in this paper were instead implemented in OCaml as modules satisfying the signature 
% implied by Fig.~\ref{fig:color-impl}. A functor then adapts these OCaml-based livelit implementations 
% to livelit definitions suitable for inclusion in the livelit context used in the semantics.


% This means that it is conceptually straightforward to hide the model and splice data and instead display 
% the computed view. However, there are a number of non-trivial layout challenges that come up, which we discuss
% in Sec.~\ref{sec:layout} below.

\subsection{Text Editor Integration}
Livelits do not require the use of a structure editor. 
We have also developed a 
proof-of-concept implementation of livelits in a textual program editor, Sketch-n-Sketch \cite{sns-pldi,sns-uist},
which recently added support for the necessary mechanisms \cite{DBLP:journals/pacmpl/LubinCOC20}.
Livelit interaction causes the serialized model in the text buffer to be changed, which updates the view.
This proof-of-concept is not at feature parity with 
our main implementation, but it demonstrates that a syntax-recognizing text editor \cite{DBLP:journals/tosem/BallanceGV92,DBLP:conf/sde/HorganM84,interactive-visual-notation} 
is sufficient to support livelits, albeit with some gaps in livelit UI availability when 
there are syntax errors.

\subsection{Layout}\label{sec:layout}
Whether implemented in a structure editor or a text editor, livelits present 
interesting layout challenges. 
Hazel uses an optimizing pretty printer based on the work of \citet{DBLP:journals/pacmpl/Bernardy17} to determine layout. This system relies 
fundamentally on character counts. Consequently, our implementation asks each 
livelit to specify dimensions in terms of character counts rather than pixels.
Livelits can be laid out either as inline livelits, like \li{\$slider},
which are one character high and appear inline with the code,
or as \emph{multi-line livelits}, which occupy up to the full width 
and a specified number of lines. % We omitted these specifications in Fig.~\ref{fig:color-impl}. 
% The pretty printer lays out multi-line livelits
% on their own line, as seen in Fig.~\ref{fig:grading}. 

% We are currently exploring a more flexible system whereby 
% multi-line livelits can be named and applied within a larger expression, but
% the GUI is not displayed until the next line, even if there are intervening characters.
% This is inspired by the \verb|~| placeholder for large textual literals in the 
% Wyvern programming language \cite{TSLs}. 

One might also considering a number of other layout 
options, e.g. inline-block literals \emph{a la} Wyvern \cite{TSLs}, pop-up livelits, livelits pinned to sidebars, and livelits that are rendered  
on a separate canvas or document while still formally being located within an underlying functional program. 

This latter option 
would be particularly interesting for end-user programming scenarios: users with limited
programming experience 
could interact with a collection of livelits laid out separately in the popular ``dashboard'' style, 
without necessarily
even being aware that their interactions are actually edits to an underlying typed
functional program. 
% More experienced users on the team could then make programmatic use of the entered
% data and parameters in a seamless manner. We are also excited about the possibility of end users
% getting curious enough to ``View Source'' in this setting, and in so doing 
% being exposed to typed functional programming concepts gradually.

% The layout of splices within livelits is also an interesting problem, and our current approach, where 
% the provider must specify a fixed size for each splice, is not particularly flexible. 
% A more principled system for GUI layout that integrates with program pretty 
% printing could be of substantial use in some settings. Formula bars, like the one in the \li{\$dataframe}
% livelit, are a useful design pattern to avoid splice layout issues, and it may be possible to provide 
% a generalized formula bar in the editor user interface that multiple livelits could share.
% While splices can in principle contain other livelits, it is difficult for small splice editors to 
% visually accomodate large livelit views in practice. 
% Fortunately, variables can be used to support composition via substitution 
% rather than nesting. A linter might suggest a refactoring to a variable in these scenarios.

% The Hazel pretty printer uses memoization and incremental diffs to maintain reasonable performance,
% and adding livelits to the system did not substantially interfere with the existing approach.
% For the relatively simple livelits that we have developed, performance has been adequate.
% However, more complex  livelits or livelits that update quickly 
% could easily cause problems with editor responsiveness.
% This could perhaps be resolved by performing livelit-related computations asynchronously. Care 
% would need to be taken to ensure that intervening edits do not cause synchronization issues.

% Many livelit-related computations can be intelligently memoized, e.g. view computations and unchanged splice evaluations,
% especially with the edit awareness that Hazel provides. 
% We leave further optimizations along these lines to future work.
% Prior work on incrementally updating UIs after code changes could be relevant to this effort \cite{burckhardt2013s}.

\subsection{Integration into Imperative Languages}
\label{sec:imperative-langs}
Our focus in this paper was on pure languages.
Side effects pose a challenge on two fronts.
If they occur in a livelit implementation,
then the state exists at edit-time. 
The most viable approach would be to retain the model-view-update
architecture and specify that any transient mutable state will be reset between updates.
If side effects occur in livelit expansions, 
then the closure collection mechanism requires more 
care to avoid unsoundness, as described in Sec.~\ref{sec:calculus-closure-collection}.
However, these problems are surmountable, and we look forward to 
efforts to integrate livelits into imperative languages. 

% The Elm architecture has proven practical for sophisticated user interfaces, 
% e.g. the Sketch-n-Sketch system \cite{sns-uist}, 
% and its purity addresses means there are no issues 
% related to initialization, maintenance, and persistence of mutable state
% by the editor.
% Imperative languages can express the same architecture (indeed, a close cousin 
% is at the heart of the popular React framework for Javascript).
