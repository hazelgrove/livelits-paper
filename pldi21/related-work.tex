\section{Related Work}\label{sec:related-work}
As detailed in Sec.~\ref{sec:background},
the Graphite system developed the idea of filling 
typed holes using a type-specific user interface, and it was the starting 
point for our work \cite{Graphite}.
Subsequent work on \textbf{mage} further explores this design space \cite{DBLP:conf/uist/KeryRHMWP20}.
This prior work engaged in substantial qualitative evaluations, 
which due to the fundamental similarities between the two systems 
is as relevant to our design as theirs. However, the prior work 
left a number of core technical issues unresolved, 
as summarized in Sec.~\ref{sec:contributions}. 
Livelits resolve these issues in large part by bringing together 
seemingly disparate ideas from several other pieces of recent work.

In particular, recent work on type-specific languages (TSLs) \cite{TSLs} 
and typed literal macros (TLMs) \cite{TLMs}
explored similar ideas of user-defined
literal forms with support for hygienic splicing, with the former 
using type-directed dispatch similar to Graphite and the latter supporting 
decentralized extensibility via explicit naming as in our approach. 
However, these systems operate in a purely 
textual setting. There is no support for graphical notation or 
interaction, nor live feedback, which are the primary contributions of this paper.
The typed expansion judgement central to the typed livelit calculus in this paper 
is structured similarly 
to the corresponding judgements in the formal systems describing TLMs and TSLs,
and the reasoning principles (e.g. hygiene properties, the focus on reasoning abstractly 
about typing) was inspired directly by this work. However, the  
approach to hygiene we take is cleaner by its use of function application 
rather than direct insertion of splices (and could perhaps be ported to those formalisms).
Splices also operate quite differently in our work,
because they are placed automatically and structurally delimited in the user interface, rather than 
placed by the client and then parsed out of the text by a custom parser charged 
with retaining provenance information.
This substantially complicates the design of splices in the prior work.

Recent work on a interactive visual macro system in Racket 
is quite similar in spirit to our work \cite{interactive-visual-syntax}.
However, it supports only a limited form of splicing via a pop up text editor
that does not support full compositionality as described here, where livelits 
can appear within other livelits. Splicing is not hygienic, nor is there any 
consideration of typing. Parameterization and partial application is also impossible, because invocation 
is via an editor command rather than a language-integrated mechanism. Finally,
and most critically, there is no support for liveness. 

Our structural delimitation of splices is reminiscent of work on 
\emph{language boxes} \cite{DBLP:conf/sle/RenggliDN09}, which focused on 
combining different notations, primarily textual, using structural delimiters inserted 
explicitly using a special-purpose editor. 

There has been a long line of research on \emph{projectional editing}, where
the user edits graphical representations (projections) of code constructs 
\cite{DBLP:journals/ile/MillerPMV94,read1996generating,DBLP:conf/sde/Reiss84,DBLP:conf/uist/KoM05}. 
Many of the oldest systems offer only a fixed set of projections and interaction techniques. 
More recently, language workbenches with support for projectional editing like Citrus \cite{DBLP:conf/uist/KoM05}
and MPS \cite{voelter2011language} have made it easier to define new projectional editors.
However, these systems generate entire editors, whereas livelit definitions can be 
included granularly in a livelit context and support decentralized extensibility.
Furthermore, livelit definitions are governed by a type and binding discipline that 
serves to ensure that a programmer unfamiliar with the semantics of an unfamiliar livelit
can still be sure that it is not operating outside of the bounds of the type system.
In many ways, an unfamiliar livelit can be approached much like an unfamiliar function
can be approached: by inspecting its declarations and reasoning compositionally 
about arguments, i.e. splices and parameters.
Finally, livelits integrate cleanly with live evaluation of programs with holes \cite{HazelnutLive},
whereas many projectional editors do not support live evaluation or do 
not integrate liveness into projections.

A number of notebook systems, including Mathematica, Jupyter \cite{Guo13}, and others, 
support the insertion of simple widgets like sliders that respond live to changes.
 

Related to projectional editors are ``syntax-recognizing editors'' \cite{DBLP:journals/tosem/BallanceGV92,DBLP:conf/sde/HorganM84} and 
a variety of systems that generate visualizations from code \cite{DBLP:conf/chi/Lerner20,koschke2003software,urquiza2004survey}. Livelits differ 
from these systems in directionality: visualization systems are given code and generate
a visualization, whereas livelits are asked to create code where there would otherwise
be a hole.

Conal Elliott's work on tangible functional programming \cite{DBLP:conf/icfp/Elliott07} similarly explored a
system that provides visual feedback about the result of a computation, 
but the computation itself could be constructed using only a fixed structural syntax
manifest as a series of connected windows.

The Vital programming environment for Haskell \cite{hanna2002interactive} 
supports type-specific stylesheets
that can be used to create custom user interfaces for both editing and displaying
values. The editors support splices, called \emph{cells}, that can contain 
Haskell code. Results are computed in a live manner, though there is no support
for evaluating incomplete programs. The visualizations themselves cannot provide
live feedback. Moreover, the system does not enforce any hygiene or typing principles:
the user is entirely responsible for syntactic and semantic correctness, and the 
underlying semantics have not been formally defined.


% Hazelnut
% Hazelnut Live
% Its Alive! paper (GUI widgets)
% Tangible FUnctional Programming 
% Vital https://dl.acm.org/doi/10.1145/581478.581493 and https://link.springer.com/chapter/10.1007/11964681_12
% Projection Boxes http://cseweb.ucsd.edu/~lerner/proj-boxes.html
% Tiny Structure Editors for Low Low Prices
% Typed Holes Work (synthesis stuff from Haskell, mention Haskell)
% Pharo / Smalltalk 
% GEC Toolkit: https://link.springer.com/chapter/10.1007/11546382_5
% Previous use of live literals https://homepages.cwi.nl/~storm/livelit/livelit.html
% Visual Haskell and related visual programming environments: https://pdfs.semanticscholar.org/6e43/5494ef1aaaa6ec0fcf47bb5b03d7c06ef937.pdf
% widgets in dataflow languages like puredata and labview
