\section{Livelits by Example}\label{sec:case-studies}


In this section, we will detail the livelits mechanism by way of
two domain-specific case studies:
a course grade assignment case study in Sec.~\ref{sec:live-grading}
and an image transformation case study in Sec.~\ref{sec:image-transformation}.
% We also briefly mention several other examples in Sec.~\ref{sec:additional-examples}.\todo{do we?}{}
These case studies have been implemented
in Hazel, a browser-based live programming environment for a 
variant of Elm \cite{Elm}.
Elm is an industrial pure typed functional language in the
ML family used for client-side web development.
We assume basic familiarity with ML.

\subsection{Case Study: Grading with Livelits}\label{sec:live-grading}
Consider this familiar scenario: an instructor needs
(1) to record numeric grades for various assignments and exams, and
(2) to visualize and perform various computations with these numeric grades
in order ultimately to assign final letter grades.
(In fact, this case study is not contrived: one author is using Hazel to compute grades this semester.)

The most common end-user application for this task is the spreadsheet, because
it allows an instructor to (1) record grades using a natural tabular interface,
(2) visualize this data in one of a finite number of plot styles,
and (3) perform basic computations,
with results updated live.
However, these affordances are limited.
It is difficult to package common operations
into reusable libraries, interact with the data using domain-specific visualizations,
and perform complex, unanticipated operations
(e.g. preparing the data in an idiosyncratic format demanded by the university registrar).

General-purpose programming languages
\emph{can} handle these scenarios, but users
lose the ability to 
directly manipulate data and visualizations
and receive live (in-editor) feedback.

Livelits are able to address this tension.
Fig.~\ref{fig:grading} shows a Hazel program where
the instructor alternates between programmatic and direct manipulation in several situations.

First, the instructor defines a value \li{grades}
that records the grades for each student using a livelit, \li{\$dataframe},
that implements a tabular user interface. The formula bar
allows the selected cell to be filled with an arbitrary Hazel expression,
here a floating point expression that adds together individual problem scores,
one of which is expressed using another livelit,
\li{\$slider}.
The table itself displays not the expression itself but rather its value, \li{93.}, just as in a spreadsheet.

Next, the instructor computes \li{averages}
for each student by applying \li{compute_weighted_averages}, a helper function
defined in a library  (not shown) shared between courses.

Next, the instructor wants to ``eyeball'' reasonable letter grade \li{cutoffs} 
by directly manipulating a domain-specific livelit, \li{\$grade_cutoffs}, that provides draggable ``paddles''
superimposed on a live visualization of the distribution of \li{averages}, 
which is provided as a livelit parameter.
The value of \li{cutoffs} is a labeled 4-tuple containing each cut-off.

Finally,
the instructor programmatically assigns grades to students
based on these \li{cutoffs}
by calling \li{assign_grades}
and \li{format_for_university},
again shared functions.

\subsection{Livelit Expansion}\label{sec:livelit-expansion}
Livelit invocations are expressions 
that are given meaning by expansion.
For example, the expansion of Fig.~\ref{fig:grading} is:

\begin{lstlisting}[xleftmargin=0.2cm]
let grades = Dataframe (
  ["A1", "A2", "A3", "Midterm", "Final"],
  [("Alice", [24. +. 36. +. 33.,
              92., 83.5, 95., 88.]),
   ("Bob", [61., 64., 98., 70., 85.]),
   ("Carol", [75., 81., 73., 82., 79.]),
   (* ... *) ]) in
let averages = compute_averages grades weights in
let cutoffs = (.A 90., .B 80., .C 70., .D 60.) in
format_for_university
  (assign_grades averages cutoffs)
\end{lstlisting}

The client can inspect this expansion in Hazel via a toggle (not shown).
Ideally, however, reasoning about types and binding
should not require the client to inspect the expansion
nor the livelit implementation (which specifies the expansion logic
as we will describe in Sec.~\ref{sec:livelit-definitions}).
After all, function clients do not need to look inside
function bodies to reason about types and binding.
Instead, in the words of \citet{DBLP:conf/ifip/Reynolds83},
``type structure is a syntactic discipline for maintaining levels of abstraction''.
Livelits maintain this discipline by
several means, described next in Sec.~\ref{sec:expansion-typing}-\ref{sec:hygiene}.

\subsubsection{Expansion Typing}
\label{sec:expansion-typing}
To support abstract reasoning about the type of the expansion,
livelit definitions declare an \emph{expansion type}.
The declarations of the livelits in Fig.~\ref{fig:grading},
eliding their implementations, are:
\begin{lstlisting}[numbers=none,xleftmargin=0cm]
livelit $dataframe at Dataframe {...}
livelit $grade_cutoffs(averages: List(Float)) at
  (.A Float, .B Float, .C Float, .D Float) {...}
livelit $slider (min: Int) (max: Int) at Int {...}
\end{lstlisting}
The expansion type of \li{\$dataframe} is \li{Dataframe},
which classifies tabular floating point data together with string row and column names (see the expansion above).
The expansion type of \li{\$grade_cutoffs} is a labeled product of grade cutoffs (field labels are written \li{.label}
rather than \li{label:} in Hazel).
Hazel displays the information in the livelit declaration when the cursor is on the livelit's name,
just as it displays typing information in other situations (not shown) \cite{hazeltutor}.

\subsection{Compositionality}\label{sec:splicing-and-parameterization}
Livelit invocations are compositional: they can embed sub-expressions
in the form of parameters and splices.

\subsubsection{Parameters}\label{sec:parameterization}
Livelit can declare a finite number of parameters of specified types.
For example, \li{\$grade_cutoffs} above declares one parameter,
the averages to be plotted, of type \li{List(Float)}.
Parameters are applied
using function application notation
as seen in Fig.~\ref{fig:grading} or
using the pipelining (i.e. reverse function application) operators, \li{<|} and \li{|>},
which allow multiple livelits to form dataflows (not shown).

Livelit abbreviations can partially apply parameters. For example,
we can partially apply the first parameter of \li{\$slider} to define a parameterized unsigned slider livelit:
\begin{lstlisting}[numbers=none,xleftmargin=0cm]
  let $uslider = $slider 0 in ...
\end{lstlisting}
Only livelits with no remaining parameters can be invoked,
so writing \li{\$uslider} in expression position will display as a ``missing livelit parameter'' error.
In Hazel, erroneous expressions
are automatically placed inside holes and do not prevent other parts of the program from evaluating
\cite{HazelnutLive}.
A livelit invocation can also indicate that no 
expansion is available with a custom error message, e.g. due to non-sensical bounds,
which is also displayed to the user (not shown).
% \footnote{\label{footnote:typing}}


\begin{figure*}
  \begin{center}
    \includegraphics[width=40pc]{img-filter.png}
    % \begin{subfigure}[t]{0.5\textwidth}
    %   \centering
    %   \includegraphics[width=15pc]{img-filter-1.png}
    %   \caption{}
    % \end{subfigure}\begin{subfigure}[t]{0.5\textwidth}
    %   \centering
    %   \includegraphics[width=15pc]{img-filter-2.png}
    %   \caption{}
    % \end{subfigure}
  \end{center}
  \caption{Case Study: Image Transformation. The image shown is determined based on the selected closure.}
  \label{fig:img-transformation}
\end{figure*}


\subsubsection{Splices}\label{sec:splices}
Spliced sub-expressions, or \emph{splices}, appear directly inside the livelit GUI.
Splices can be filled with Hazel expressions of any form, including other livelit invocations.
For example, each cell in the \li{\$dataframe} GUI in Fig.~\ref{fig:grading}
has a corresponding splice. The formula bar at the top
allows the user to edit the splice corresponding to the selected cell,
and all of Hazel's editing affordances are available when the client does so.
% Not all splices necessarily appear in the GUI.
Unlike parameters, the number of splices can change
as the user interacts with the livelit, e.g. when adding or removing rows or columns in a \li{\$dataframe}.
% The cell itself displays the live value of the spliced expression---
% (we return to live evaluation in Sec.~\ref{sec:live-evaluation} below.

The livelit provides an expected type for each splice when it is created.
For example, the splices for the row and column keys in Fig.~\ref{fig:grading} have expected type \li{String},
and the remaining cells have expected type \li{Float}.
Hazel displays and uses the expected type when the cursor is on the splice \cite{hazeltutor}.
% If an expression of invalid type is entered, it will display in an error hole as usual,
% and in Hazel this will not prevent evaluation of other expressions (see Footnote \ref{footnote:typing}).

% Unlike parameters, the number of splices is not fixed in the livelit declaration. Splices can be created,
% deleted, and filled through user interaction with the livelit. For example, clicking the \li{+} buttons
% in Fig.~\ref{fig:grading} will create new rows or columns, which will in turn generate new splices.

\subsubsection{Binding Discipline}\label{sec:hygiene}


Ensuring that clients can reason about binding while leaving expansions
 abstract
 requires enforcing two key properties: \emph{capture avoidance}
and \emph{context independence} \cite{TLMs,adamsHygiene,DBLP:conf/popl/ClingerR91}.

\textbf{Capture Avoidance.}
Splices and parameters are passed into the expansion as function arguments, 
so they cannot capture any bindings internal to the expansion. All bindings 
are lexically scoped to the livelit invocation site.

Consider an alternative where 
splices could directly appear anywhere within the expansion, 
e.g. in the body of a generated function or \li{let} binding.
Na\"ively, this could cause inadvertent capture of the bound variables in the expansion by a free variable
in the parameter or splice. For example, consider a livelit that generates an expansion
of the following seemingly innocuous form:
\begin{lstlisting}[numbers=none]
  let len = strlen <splice1> in
  if len > 0 then Some (<splice2> + len) else None 
\end{lstlisting}
Here, \li{<splice2>} appears under the binding of \li{len}. If the client has filled
\li{<splice2>} with an expression that refers to a client-side binding of \li{len},
these references would na\"ively be captured. This would not occur in \li{<splice1>},
because the \li{let} is not recursive.
This breaks abstraction and is notoriously difficult to debug,
both for the livelit provider, who has no way to predict which variables a client will use,
 and the client, who does not know which variables the provider used.

To avoid this situation, parameters and splices must be placed in the expansion
in a capture-avoiding manner: variables in splices
always refer to the bindings visible to the client,
rather than bindings that are hidden inside the expansion.
In other macro systems, e.g. that of \citet{TLMs}, 
capture avoiding substitution is used. 
% This allows splices to appear 
% multiple times and also appear in branches that may or may not evaluate. 
To simplify reasoning about live evaluation, discussed below, we take 
an even more restrictive approach by passing splices in as function 
arguments. In an eagerly evaluated language, this ensures that 
splice evaluation is not conditional on computations invisible to the 
client (such as the length check above), 
and it also ensures that evaluation will not inadvertently 
occur multiple times or not occur at all. 
% We discuss how this is implemented in Sec.~\ref{sec:expansion}.

The trade-off is that this prevents the expression of new 
binding constructs and even control flow constructs as livelits.
We take this as a strength, in that it simplifies 
reasoning for client programmers. However, this is an independent 
design decision: 
more permissive designs that nevertheless 
maintain the critical capture avoidance property are possible.


% This is implemented by alpha-renaming bindings internal to the expansion as necessary.
% (We discuss potentially relaxed variations of this hygiene discipline in Sec.~\ref{sec:discussion}.)

\textbf{Context Independence.}
The example expansion above used a library function, \li{strlen}.
Na\"ively, this expansion would break if placed
in client contexts where \li{strlen} is not bound, or bound to
an unexpected value.
To avoid requiring clients to determine and satisfy these invisible
dependencies, the livelits mechanism enforces \emph{context independence}:
generated expansions are valid in any context. Dependencies are bound
relative to the livelit definition site (see Sec.~\ref{sec:expansion}).

\subsection{Live Evaluation}\label{sec:live-evaluation}
Livelits have the ability to evaluate a splice or a parameter
in order to provide better feedback about run-time behavior to the client.
The \li{\$dataframe} livelit uses this facility to display
the evaluation result for each cell, like a spreadsheet.
The \li{\$grade_cutoffs} livelit uses this facility to plot the grades, which were
passed in as a parameter, on the number line.

\subsubsection{Closure Collection}\label{sec:closure-collection-example} The subtlety is that 
evaluation in Hazel is defined for closed expressions as usual,
but parameters and splices can be open, i.e. refer to surrounding variables.
To provide a environment that binds these variables,
Hazel performs \textbf{closure collection} in two phases.

In the first phase, \emph{proto-closure collection},
Hazel replaces each livelit with a uniquely numbered hole and then evaluates the program
using the semantics for evaluating programs with holes developed by \citet{HazelnutLive}.
Evaluation proceeds around these holes, producing a result containing
corresponding hole closures, i.e. holes with environments.

For example, there is one closure for \li{\$dataframe} in Fig.~\ref{fig:grading}.
It contains the value of \li{q1\_max} and the other variables in scope.
These values can be used to evaluate
splices that use these variables, such as the cell selected in Fig.~\ref{fig:grading}.

Similarly, the closure for \li{\$grade_cutoffs} in Fig.~\ref{fig:grading} includes
the necessary \li{averages} variable, but
its value depends on \li{grades}, which is determined by \li{\$dataframe}.
If we stop after proto-closure collection,
no useful value will be available:
\li{averages} will be \emph{indeterminate}, because \li{\$dataframe} has been replaced with a hole \cite{HazelnutLive}.
For this reason, there is a second phase of closure collection, \emph{closure resumption},
where any livelit holes
in the collected livelit closures are \emph{resumed}, i.e. the hole is filled with the expansion
 and evaluation resumes.
% Livelit expansions do not contain livelits, so no subsequent expansion / resumption phases are necessary.

% Hazel does not need to evaluate the program multiple times in order to support live closure collection
% as well as the usual live evaluation services it offers, because the final evaluation result
% can also be determined by resumption.
% (We discuss how to address subtleties that arise
% in the presence of non-commutative side effects in Sec.~\ref{sec:calculus-closure-collection}.)

\subsubsection{Indeterminate Results}
Even after closure resumption, some elements of the closure may remain indeterminate, e.g. due to holes that
are not filled with livelits.
When a livelit requests an evaluation result, it must be able to handle these indeterminate results.
For example, if there were missing grades,
then \li{\$grade_cutoffs} would have degraded functionality:
it would display only the list elements that are values on the timeline, skipping indeterminate elements.
We will return to how this occurs in Sec.~\ref{sec:live-evaluation-def}.



\subsubsection{Case Study: Live Image Filters}\label{sec:image-transformation}
Our next case study
considers the situation where multiple closures are collected, and
the workflows it enables.

We interviewed a photographer
who described a typical workflow:
they use the Lightroom application to
apply a set of adjustments
across all photos in a collection before making
individual adjustments.
Many photographers do this one photo at a time,
though this photographer had recently learned how to
use Lightroom's saved presets to
apply adjustments to multiple photos at once.
However, they remained dissatisfied by the workflow.
They wanted to be able to see how the shared settings affected
multiple photos as they tweaked them, without having to
save and reapply the preset.
They also wanted to be able to change the applied preset
even after making individual adjustments.
Finally, they
expressed interest in parameterizing
presets, and in automating parts of the post-production process.

% For our next case study, we
% developed a library of
% image transformation livelits,
% motivated by a desire to automate



% The livelits we have considered so far generate simple values.

Motivated by this interview,
we prototyped a collection of photo filter livelits.
One of these, \li{\$basic_adjustments}, is demonstrated in Fig.~\ref{fig:img-transformation}. 
This livelit contains two splices of type \li{Int},
one to adjust the contrast, and the other to adjust the brightness.
In this example, we have filled those splices with \li{\$percent}, but
as above we could enter any expression of type \li{Int}, e.g. a variable.
The livelit shows a live preview of the transformed image.
The expansion generates calls to a browser image processing framework, 
not shown.

This livelit is used within a function, \li{classic_look}, that creates a ``preset'' filter. 
This function is mapped over a list of images (loaded by URL) at the bottom of the figure. 
% We show two for space reasons.
The provided URL is passed into to livelit as a parameter.

Because the livelit appears inside a function applied (by \li{map}) twice, 
there are now two closures associated with the livelit. 
Hazel allows the programmer to select between the closures when 
the cursor is on the livelit expression via the sidebar toggle,
shown in the middle of Fig.~\ref{fig:img-transformation}.
This allows the client to see how the filter being designed will affect a
number of example images by quickly toggling between closures.
The underlying expansion remains abstract, i.e. it refers to the image via the \li{url} variable.

We showed this and similar examples to the photographer we had interviewed.
They expressed enthusiasm for this approach despite having only
limited programming experience (with Python). They made the fair point that it
would take substantial effort to match Lightroom's breadth of filters,
but stated that this approach could be more powerful than
Lightroom's point-and-click interface while retaining many of its benefits (specifically
mentioning sliders). Although this was only a single interview,
it is consistent with the body of evidence summarized in Sec.~\ref{sec:background}.

% \subsection{Additional Examples}\label{sec:additional-examples}
% It would be nice to have a gallery-style figure and a brief overview of some other case studies
% and how they exercise the novel features of the livelits mechanism. Maybe some statistics on how
% many lines of code it took.

% Ideas:
% \begin{itemize}
%   \item derivation trees like Joomy's system (\url{https://joom.github.io/proof-tree-builder/src/})
% \end{itemize}