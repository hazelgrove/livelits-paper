% !TEX root = livelits-paper.tex

%% no citations in Abstract
%%
%% \cite{popl-paper}

\begin{abstract}
    Although textual program manipulation is powerful, 
    some types of expressions are better represented and manipulated
    graphically.
    % Examples include expressions that compute 
    % colors,
    % animation parameters,
    % musical sequences,
    % tabular data, 
    % plot parameters,
    % GUI widgets, 
    % and various other domain-specific data structures.
    This paper introduces \emph{live literals}, or \emph{livelits}, 
    a mechanism that allows
    client programmers to fill holes of types like these 
    by directly manipulating a     
    provider-specified GUI embedded persistently into otherwise 
    symbolic code. 
    Uniquely, livelits are compositional: the GUI can itself contain
    spliced expressions, which can in turn contain other livelits. 
    Splices are typed and the system 
    ensures that the livelit treats each splice hygienically. 
    Livelits are also uniquely live: 
    they can offer {immediate feedback} about the run-time implications 
    of the programmer's choices 
    even when the spliced expressions mention bound variables, 
    because the system gathers closures associated with the 
    hole that the livelit is tasked with filling.
    This paper introduces livelits with case studies that exercise these novel capabilities. 
    We implement livelits in Hazel, 
    a live programming environment able to 
    typecheck and run programs with holes. 
    We then define a typed lambda calculus that captures the essence of livelits, which
    operate as live, graphical macros.     
    The macro expansion process has been mechanized in Agda.
\end{abstract}
