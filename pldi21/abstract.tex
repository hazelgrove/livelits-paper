% !TEX root = livelits-paper.tex

%% no citations in Abstract
%%
%% \cite{popl-paper}

\begin{abstract}
    Text editing is powerful, but 
    some types of expressions are more naturally represented and manipulated
    graphically.
    Examples include expressions that compute 
    colors,
    music,
    animations,
    tabular data, 
    plots,
    GUI widgets, 
    mathematical diagrams, 
    and other domain-specific data structures.
    This paper introduces \emph{live literals}, or \emph{livelits}, 
    which allow
    clients to fill holes of types like these   
    by directly manipulating a     
    user-defined GUI embedded persistently into code. 
    Uniquely, livelits are \emph{compositional}: a livelit GUI can itself embed 
    spliced expressions, which are typed, lexically scoped, enjoy full editor support and can in turn embed 
    other livelits. 
    Livelits are also uniquely \emph{live}: 
    a livelit can provide {continuous feedback} about the run-time implications 
    of the client's choices 
    even when splices mention bound variables, 
    because the system continuously gathers closures associated with the 
    hole that the livelit is tasked with filling.
    We integrate livelits into Hazel, 
    a live programming environment able to 
    typecheck and run programs with holes, 
    and describe several case studies that exercise these novel capabilities. 
    We then define a minimal typed livelit calculus, which precisely 
    specifies how livelits operate as 
    live graphical macros.     
    The metatheory of macro expansion has been mechanized in Agda.
\end{abstract}
