\section{Formal System}

\subsection{Background: \Hazelnut}

\begin{itemize}

\item typed expressions with holes

\end{itemize}

\parahead{Static Semantics}

\begin{itemize}

\item \cite{Hazelnut}

\end{itemize}

\parahead{Dynamic Semantics}

\begin{itemize}

\item \cite{HazelnutLive}

\end{itemize}

\subsection{Palette Expressions}

\parahead{Palette Macros}

\begin{itemize}

\item macros / analytic

\item macros / synthetic

\item synthetic requires type splices

\item can there be an ``untyped'' version that is like synthetic but relies on casts rather than type splices?

\end{itemize}

\parahead{Palette Functions}

\begin{itemize}

\item functions / analytic

\end{itemize}

\subsection{Edit Actions}

\parahead{Palette Initialization}

\begin{itemize}

\item init 

\end{itemize}

\parahead{Palette-Specific Actions}

\begin{itemize}

\item defined and registered with the editor, can be used for the ``normal''
action tool box

\item like other actions, can be recorded in action history and, in the future,
scripted in edit action macros

\end{itemize}

\subsection{Live Palettes}

\begin{itemize}

\item for view function

\item hole for palette application, run, collect hole closures, pass to view

\end{itemize}

\subsection{Extensions}

\begin{itemize}

\item parameterized palettes (value arguments)

\item polymorphic analytic palettes (type arguments)

\end{itemize}



\newcommand{\dexpand}{\dexp_{\mathsf{expand}}}
\newcommand{\tmodel}{\htyp_{\mathsf{model}}}
\newcommand{\texpansion}{\htyp_{\mathsf{expansion}}}
\newcommand{\dmodel}{\dexp_{\mathsf{model}}}
\newcommand{\denc}{d_{\mathsf{enc}}}
\newcommand{\eexpanded}{\hexp_{\mathsf{expanded}}}
\newcommand{\tsplice}{\htyp_{\mathsf{splice}}}
\newcommand{\psplice}{\pexp_{\mathsf{splice}}}
\newcommand{\esplice}{\hexp_{\mathsf{splice}}}

\newcommand{\eexpand}{\hexp_{\mathsf{expand}}}
\newcommand{\emodel}{\hexp_{\mathsf{model}}}

\begin{figure*}
        \newcommand{\absgrammar}[2]{c\,|\,$x$\,|\,\hap{#1}{#1}\,|\,\halam{$x$}{#2}{#1}}
        \newcommand{\absgrammargen}[2]{\absgrammar{#1}{#2}\,|\,\hlam{$x$}{#1}\,|\,#1 : #2\,|\,\hehole{\mvar}\,|\,\hhole{#1}{\mvar}}
\begin{grammar}
 Palette definitions
 & $\pDef$
   & $\bnfas$ &
     $\pDefRecord{\dexp}{\htyp}{\htyp}$
%
\\[1ex]
 Basic HTyps
 & $\htyp$
   & $\bnfas$ &
     b\,|\,$\tehole$\,|\,$\tarr{\htyp}{\htyp}$
%
\\[1ex]
 HExps
 & $\hexp$
   & $\bnfas$ &
     $\absgrammargen{\hexp}{\htyp}$
%
\\[1ex]
 IHExps
 & $\dexp$
   & $\bnfas$ &
     $\absgrammar{\dexp}{\htyp}\,|\,\dehole{\mvar}{\subst}{}\,|\,\dhole{\dexp}{\mvar}{\subst}{}$
 \\ &&& $\bnfaltbrk \dcasttwo{\dexp}{\htyp}{\htyp}$
 \\ &&& $\bnfaltbrk \dcastfail{\dexp}{\htyp}{\htyp}$
%
\\[1ex]
 Palette HExps
 & $\pexp$
   & $\bnfas$ &
     $\absgrammargen{\pexp}{\htyp}$
 \\ &&& $\bnfaltbrk \pexpPalLet{\rho}{\pDef}{\pexp}$ & Palette definition $\pDef$ as $\rho$ in $\pexp$
 \\ &&& $\bnfaltbrk \pexpPalAp{\rho}{\dexp}{\htyp}{\pexp}$
                                                & Palette definition $\rho$ expands model $\dexp$ into a function
 \\ &&&                                         & that takes $\pexp$ of type $\htyp$ and returns the resultant HExp
%
\end{grammar}
\hfill \\ \hfill \\ \hfill \\ \hfill \\ TODO where do we define the meanings of $\encExp$, $\downArrowsTo{}{}$, and $\decode{}{}$?
\begin{mathpar}
\\\\
\inferrule[SPELetPal]{
    \pi = \pDefRecord{\dexpand}{\tmodel}{\texpansion} \\\\
    \hasType{\EmptyDelta}{\EmptyhGamma}{\dexpand}{\tarr{\tmodel}{\encExp}} \\ \\
    \pexpandSyn{\hGamma}{\pPhi, \rho : \pi}{\pexp}{\hexp}{\tau}
  }{
    \pexpandSyn{\hGamma}{\pPhi}{\pexpPalLet{\rho}{\pi}{\pexp}}{\hexp}{\tau}
  }
\\\\
\inferrule[SPEApPal]{
    \rho:\pDefRecord{\dexpand}{\tmodel}{\texpansion} \in \pPhi \\\\
    \pexpandAna{\hGamma}{\pPhi}{\psplice}{\esplice}{\tsplice} \\\\
    \hasType{\EmptyDelta}{\EmptyhGamma}{\dmodel}{\tmodel} \\ \\
    \downArrowsTo{\dap{\dexpand}{\dmodel}}{\denc} \\\\
    \decode{\denc}{\eexpanded} \\ \\
    \hana{\EmptyhGamma}{\eexpanded}{\tarr{\tsplice}{\texpansion}}
  }{
    \pexpandSyn{\hGamma}{\pPhi}{\pexpPalAp{\rho}{\dmodel}{\tsplice}{\psplice}}{\hap{\left( \eexpanded : \tarr{\tsplice}{\texpansion} \right)}{\esplice}}{\texpansion}
  }
\\\\
\inferrule[APELetPal]{
    \pi = \pDefRecord{\dexpand}{\tmodel}{\texpansion} \\\\
    \hasType{\EmptyDelta}{\EmptyhGamma}{\dexpand}{\tarr{\tmodel}{\encExp}} \\ \\
    \pexpandAna{\hGamma}{\pPhi, \rho : \pi}{\pexp}{\hexp}{\tau}
  }{
    \pexpandAna{\hGamma}{\pPhi}{\pexpPalLet{\rho}{\pi}{\pexp}}{\hexp}{\tau}
  }
\\
\end{mathpar}
\end{figure*}

\begin{figure*}
\begin{mathpar}
  \text{Typed palette expansion theorems:}
  \\
  \text{If} \, \pexpandSyn{\hGamma}{\pPhi}{\pexp}{\hexp}{\htyp} \, \text{then}\,
     \hsyn{\hGamma}{\hexp}{\htyp}.
  \\
  \text{If} \, \pexpandAna{\hGamma}{\pPhi}{\pexp}{\hexp}{\htyp} \, \text{then}\,
     \hana{\hGamma}{\hexp}{\htyp}.
\end{mathpar}
\end{figure*}

\begin{figure*}
Function-style palettes
\begin{mathpar}
\\\\
\inferrule[SPELetPal]{
    \pi = \pDefRecordF{\eexpand}{\tmodel}{\tsplice}{\texpansion} \\\\
    \hsyn{\EmptyhGamma}{\eexpand}{\tarr{\tmodel}{\tarr{\tsplice}{\texpansion}}} \\ \\
    \pexpandSyn{\hGamma}{\pPhi, \rho : \pi}{\pexp}{\hexp}{\tau}
  }{
    \pexpandSyn{\hGamma}{\pPhi}{\pexpPalLet{\rho}{\pi}{\pexp}}{\hexp}{\tau}
  }
\\\\
\inferrule[SPEApPal]{
    \rho:\pDefRecordF{\eexpand}{\tmodel}{\tsplice}{\texpansion} \in \pPhi \\\\
    \pexpandAna{\hGamma}{\pPhi}{\psplice}{\esplice}{\tsplice} \\ \\
    \hana{\EmptyhGamma}{\emodel}{\tmodel} \\\\
    \hsyn{\EmptyhGamma}{\eexpand}{\tarr{\tmodel}{\tarr{\tsplice}{\texpansion}}}
  }{
    \pexpandSyn{\hGamma}{\pPhi}{\pexpPalApF{\rho}{\emodel}{\psplice}}{\hap{\hap{\eexpand}{\emodel}}{\esplice}}{\texpansion}
  }
\\\\
\inferrule[APELetPal]{
    \pi = \pDefRecordF{\eexpand}{\tmodel}{\tsplice}{\texpansion} \\\\
    \hsyn{\EmptyhGamma}{\eexpand}{\tarr{\tmodel}{\tarr{\tsplice}{\texpansion}}} \\ \\
    \pexpandAna{\hGamma}{\pPhi, \rho : \pi}{\pexp}{\hexp}{\tau}
  }{
    \pexpandAna{\hGamma}{\pPhi}{\pexpPalLet{\rho}{\pi}{\pexp}}{\hexp}{\tau}
  }
\\
\end{mathpar}
\end{figure*}

\begin{figure*}
  \vskip 1cm
  For all $\hGamma$, $\pPhi$, $\rho$, $\dmodel$, $\psplice$, $\tsplice$, $\hexp$, and $\htyp$,
  \\
  if $\pexpandSyn{\hGamma}{\pPhi}{\pexpPalAp{\rho}{\dmodel}{\tsplice}{\psplice}}{\hexp}{\htyp}$
  \\
  then $\rho$ must be defined (i.e. $\rho : \pDefRecord{\dexpand}{\tmodel}{\texpansion} \in \pPhi$)
  \\
  and there must exist $\eexpanded$ and $\esplice$ such that the following hold:
  \vskip 0.5cm
  \begin{enumerate}
    \item
      \textbf{Expanded Application Form}: The expansion result has a specific form; particularly that of a function application,
      where the function part is ascripted with an arrow type from the splice type (specified by the original expression) to the
      expansion result type.
      \\
      \hskip 2em $e = \hap{\left( \eexpanded : \tarr{\tsplice}{\htyp} \right)}{\esplice}$
      \\
      This form satisfies the \textbf{Capture Avoidance} principle described in (TODO - cite the TLM paper),
      as it renders incorrect capturing structurally impossible.
      \vskip 0.3cm
    \item
      \textbf{Expansion Typing}: The expanded expression synthesizes the result type (a consequence of the "Typed Palette Expansion"
      theorem), and the result type is the same as the expansion type specified in the palette definition.
      \\
      \hskip 2em $\hsyn{\hGamma}{\hexp}{\htyp}$ and $\htyp = \texpansion$
      \vskip 0.3cm
    \item
      \textbf{Responsibility}: The palette definition's $\mathsf{expand}$ function is responsible for expanding the model -
      the result of this expansion is evaluated to a value, then that value is decoded into an hexp.
      \\
      \hskip 2em $\downArrowsTo{\dap{\dexpand}{\dmodel}}{\denc}$ and $\decode{\denc}{\eexpanded}$
      \vskip 0.3cm
    \item
      \textbf{Splice Typing}: The splice expression expands into the argument part of the expanded form,
      which can be analyzed against the type specified by the original expression.
      \\
      \hskip 2em $\pexpandAna{\hGamma}{\pPhi}{\psplice}{\esplice}{\tsplice}$ and therefore $\hana{\hGamma}{\esplice}{\tsplice}$
      \vskip 0.3cm
    \item
      \textbf{Context Independence}: The function part of the expanded form has no free variables, guaranteeing that it does not
      rely on any bindings in the application site context.
      \\
      \hskip 2em $\mathsf{free\_vars} \left( \eexpanded : \tarr{\tsplice}{\htyp} \right) = \emptyset$
  \end{enumerate}
\end{figure*}
