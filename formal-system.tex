\newcommand{\dexpand}{\dexp_{\mathsf{expand}}}
\newcommand{\tmodel}{\htyp_{\mathsf{model}}}
\newcommand{\texpansion}{\htyp_{\mathsf{expansion}}}
\newcommand{\dmodel}{\dexp_{\mathsf{model}}}
\newcommand{\denc}{d_{\mathsf{enc}}}
\newcommand{\eexpanded}{\hexp_{\mathsf{expanded}}}
\newcommand{\tsplice}{\htyp_{\mathsf{splice}}}
\newcommand{\psplice}{\pexp_{\mathsf{splice}}}
\newcommand{\esplice}{\hexp_{\mathsf{splice}}}

\begin{figure*}
        \newcommand{\absgrammar}[2]{c\,|\,$x$\,|\,\hap{#1}{#1}\,|\,\halam{$x$}{#2}{#1}}
        \newcommand{\absgrammargen}[2]{\absgrammar{#1}{#2}\,|\,\hlam{$x$}{#1}\,|\,#1 : #2\,|\,\hehole{\mvar}\,|\,\hhole{#1}{\mvar}}
\begin{grammar}
 Palette definitions
 & $\pDef$
   & $\bnfas$ &
     $\pDefRecord{\dexp}{\htyp}{\htyp}$
%
\\[1ex]
 Basic HTyps
 & $\htyp$
   & $\bnfas$ &
     b\,|\,$\tehole$\,|\,$\tarr{\htyp}{\htyp}$
%
\\[1ex]
 HExps
 & $\hexp$
   & $\bnfas$ &
     $\absgrammargen{\hexp}{\htyp}$
%
\\[1ex]
 IHExps
 & $\dexp$
   & $\bnfas$ &
     $\absgrammar{\dexp}{\htyp}\,|\,\dehole{\mvar}{\subst}{}\,|\,\dhole{\dexp}{\mvar}{\subst}{}$
 \\ &&& $\bnfaltbrk \dcasttwo{\dexp}{\htyp}{\htyp}$
 \\ &&& $\bnfaltbrk \dcastfail{\dexp}{\htyp}{\htyp}$
%
\\[1ex]
 Palette HExps
 & $\pexp$
   & $\bnfas$ &
     $\absgrammargen{\pexp}{\htyp}$
 \\ &&& $\bnfaltbrk \pexpPalLet{\rho}{\pDef}{\pexp}$ & Palette definition $\pDef$ as $\rho$ in $\pexp$
 \\ &&& $\bnfaltbrk \pexpPalAp{\rho}{\dexp}{\htyp}{\pexp}$
                                                & Palette definition $\rho$ expands model $\dexp$ into a function
 \\ &&&                                         & that takes $\pexp$ of type $\htyp$ and returns the resultant HExp
%
\end{grammar}
\hfill \\ \hfill \\ \hfill \\ \hfill \\ TODO where do we define the meanings of $\encExp$, $\downArrowsTo{}{}$, and $\decode{}{}$?
\begin{mathpar}
\\\\
\inferrule[SPELetPal]{
    \pi = \pDefRecord{\dexpand}{\tmodel}{\texpansion} \\\\
    \hasType{\EmptyDelta}{\EmptyhGamma}{\dexpand}{\tarr{\tmodel}{\encExp}} \\ \\
    \pexpandSyn{\hGamma}{\pPhi, \rho : \pi}{\pexp}{\hexp}{\tau}
  }{
    \pexpandSyn{\hGamma}{\pPhi}{\pexpPalLet{\rho}{\pi}{\pexp}}{\hexp}{\tau}
  }
\\\\
\inferrule[SPEApPal]{
    \rho:\pDefRecord{\dexpand}{\tmodel}{\texpansion} \in \pPhi \\\\
    \pexpandAna{\hGamma}{\pPhi}{\psplice}{\esplice}{\tsplice} \\\\
    \hasType{\EmptyDelta}{\EmptyhGamma}{\dmodel}{\tmodel} \\ \\
    \downArrowsTo{\dap{\dexpand}{\dmodel}}{\denc} \\\\
    \decode{\denc}{\eexpanded} \\ \\
    \hana{\EmptyhGamma}{\eexpanded}{\tarr{\tsplice}{\texpansion}}
  }{
    \pexpandSyn{\hGamma}{\pPhi}{\pexpPalAp{\rho}{\dmodel}{\tsplice}{\psplice}}{\hap{\left( \eexpanded : \tarr{\tsplice}{\texpansion} \right)}{\esplice}}{\texpansion}
  }
\\\\
\inferrule[APELetPal]{
    \pi = \pDefRecord{\dexpand}{\tmodel}{\texpansion} \\\\
    \hasType{\EmptyDelta}{\EmptyhGamma}{\dexpand}{\tarr{\tmodel}{\encExp}} \\ \\
    \pexpandAna{\hGamma}{\pPhi, \rho : \pi}{\pexp}{\hexp}{\tau}
  }{
    \pexpandAna{\hGamma}{\pPhi}{\pexpPalLet{\rho}{\pi}{\pexp}}{\hexp}{\tau}
  }
\\
\end{mathpar}
\end{figure*}

\begin{figure*}
\begin{mathpar}
  \text{Typed palette expansion theorems:}
  \\
  \text{If} \, \pexpandSyn{\hGamma}{\pPhi}{\pexp}{\hexp}{\htyp} \, \text{then}\,
     \hsyn{\hGamma}{\hexp}{\htyp}.
  \\
  \text{If} \, \pexpandAna{\hGamma}{\pPhi}{\pexp}{\hexp}{\htyp} \, \text{then}\,
     \hana{\hGamma}{\hexp}{\htyp}.
\end{mathpar}
\end{figure*}

\vskip 1cm

\begin{figure*}
  If $\pexpandSyn{\hGamma}{\pPhi}{\pexpPalAp{\rho}{\dmodel}{\tsplice}{\psplice}}{\hexp}{\htyp}$ then the following must hold:
  \vskip 0.5cm
  \begin{enumerate}
    \item
      \textbf{Expanded Application Form}: The expansion result has a specific form; particularly that of a function application,
      where the function part is ascripted with an arrow type from the splice type (specified by the original expression) to the
      expansion result type.
      \\
      \hskip 2em $e = \hap{\left( \eexpanded : \tarr{\tsplice}{\htyp} \right)}{\esplice}$ for some $\eexpanded$, $\esplice$
      \\
      This form satisfies the \textbf{Capture Avoidance} principle described in (TODO - cite the TLM paper),
      as it renders incorrect capturing structurally impossible.
      \vskip 0.3cm
    \item
      \textbf{Context Lookup}: The palette context contains a definition for the specified palette name.
      \\
      \hskip 2em $\rho : \pDefRecord{\dexpand}{\tmodel}{\texpansion} \in \pPhi$
      \vskip 0.3cm
    \item
      \textbf{Expansion Typing}: The expanded expression synthesizes the result type (a consequence of the "Typed Palette Expansion"
      theorem), and the result type is the same as the expansion type specified in the palette definition.
      \\
      \hskip 2em $\rho : \hsyn{\hGamma}{\hexp}{\htyp}$ and $\htyp = \texpansion$
      \vskip 0.3cm
    \item
      \textbf{Model Typing}: The model specified in the original expression has the type specified in the palette definition, and it
      does not rely on any bindings in the application site context.
      \\
      \hskip 2em $\hasType{\EmptyDelta}{\EmptyhGamma}{\dmodel}{\tmodel}$
      \vskip 0.3cm
    \item
      \textbf{Responsibility}: The palette definition's $\mathsf{expand}$ function is responsible for expanding the model -
      the result of this expansion is evaluated to a value, then that value is decoded into an hexp.
      \\
      \hskip 2em $\downArrowsTo{\dap{\dexpand}{\dmodel}}{\denc}$ and $\decode{\denc}{\eexpanded}$
      \vskip 0.3cm
    \item
      \textbf{Splice Typing}: The splice expression expands into the argument part of the expanded form,
      which can be analyzed against the type specified by the original expression.
      \\
      \hskip 2em $\pexpandAna{\hGamma}{\pPhi}{\psplice}{\esplice}{\tsplice}$ and therefore $\hana{\hGamma}{\esplice}{\tsplice}$
      \vskip 0.3cm
    \item
      \textbf{Context Independence}: The function part of the expanded form has no free variables, guaranteeing that it does not
      rely on any bindings in the application site context.
      \\
      \hskip 2em $\mathsf{free\_vars} \left( \eexpanded : \tarr{\tsplice}{\htyp} \right) = \emptyset$
  \end{enumerate}
\end{figure*}
