\section{A Simply Typed Livelit Calculus}\label{sec:livelit-calculus}
In order to precisely capture the semantics of livelits
independently
of the specifics of the Hazel programming environment and GUI
framework, we will now specify a simply typed \emph{livelit calculus}.

Fig.~\ref{fig:syntax} specifies the syntax of the livelit calculus.
Programs are written as \emph{unexpanded expressions}, $\hat e$, which are \emph{expanded} to
\emph{external expressions}, $e$, before being \emph{elaborated}
to internal expressions, $d$, for evaluation. All three stages are governed
by the same types, $\tau$. We include numbers, partial functions, products, sums, and recursive
types, all in their standard form \cite{pfpl}, but this specific type structure is not critical.
Any language expressive
enough to encode its own syntax (here as values of a recursive sum type)
would be a suitable basis. We begin with a terse
overview of the external and internal languages,
which are adapted straightforwardly from Hazelnut Live \cite{HazelnutLive} and therefore
serve as background material, in Sec.~\ref{sec:external-and-internal-lang}.
Only unexpanded expressions contain
livelits, so they are the main focus of the remaining sections,
which cover livelit expansion (Sec.~\ref{sec:calculus-expansion}),
livelit update actions (Sec.~\ref{sec:calculus-updates}),
liveness via closure collection (Sec.~\ref{sec:calculus-closure-collection}),
and parameterization (Sec.~\ref{sec:calculus-parameterization}).


\subsection{Background: External and Internal Language}\label{sec:external-and-internal-lang}
The external and internal languages are straightforward adaptations of the
external and internal languages of \emph{Hazelnut Live},
a typed lambda calculus that assigns static and dynamic meaning to programs with holes,
notated externally $\hehole{u}$ where $u$ is a \emph{hole name} \cite{HazelnutLive}.
We omit non-empty holes (which allow reasoning about type inconsistencies \cite{Hazelnut}) and type holes
(which operate like the unknown type from gradual type theory \cite{Siek06a,Hazelnut})
because these mechanisms are orthogonal to livelits.

The external language is governed by a typing judgement of the standard form, $\hasType{\Gamma}{e}{\tau}$,
where the typing context, $\Gamma$,
is a finite set of typing assumptions of the form $x : \tau$ \cite{pfpl}.

The internal language is a contextual type theory \cite{Nanevski2008}, i.e. the typing judgement is
of the form $\hasTypeD{\Delta}{\Gamma}{d}{\tau}$ where $\Delta$ is a finite set of hole typing
assumptions of the form $u :: \tau[\Gamma]$.
We need this hole typing context only for the internal language because, although hole names are assumed
unique in the external language, they can be duplicated during evaluation of internal expressions.
Consequently, we need a context to ensure that each closure associated with hole $u$, written $\dehole{u}{\sigma}$,
has the same type and can be filled
with expressions valid under the same context.

External expressions, $e$, are given dynamic meaning by typed elaboration to internal expressions, $d$,
according to the typed elaboration judgement, $\elabs{\Gamma}{e}{d}{\tau}{\Delta}$.
The main purpose of this elaboration step is to initialize the substitution $\sigma$ on each hole closure,
which operates to capture
the substitutions that have occurred around that hole during evaluation. The key rule is:
\begin{mathpar}
\inferrule[Elab-Hole]{ }{
    \elabs{\Gamma}{\hehole{u}}{\dehole{u}{\idof{\Gamma}}}{\tau}{u :: \tau[\Gamma]}
}
\end{mathpar}
The substitution is initially the identity substitution, $\idof{\Gamma}$, i.e. the
substitution that maps each variable in $\Gamma$ to itself, because no substitutions have yet occurred. For example,
\[ \elabs{ }{\helet{x}{\hnum{5}}{\hehole{u}} + \hnum{6}}{\helet{x}{\hnum{5}}{\dehole{u}{[x/x]} + \hnum{6}}}{\tnum}{u :: \tnum[x : \tnum]} \]
During evaluation, $\evalsTo{d}{d'}$, the closure's substitution accumulates the substitutions that occur. For example,
the internal expression above evalues as follows:
\[
  \evalsTo{\helet{x}{\hnum{5}}{\dehole{u}{[x/x]} + \hnum{6}}}{
      \dehole{u}{[\hnum{5}/x]} + \hnum{6}
  }
\]
We will use hole closures to support live programming with livelits in Sec.~\ref{sec:calculus-closure-collection}.


All of these judgements are adapted directly from those of Hazelnut Live,
differing only in that we use a simpler declarative formulation rather than an algorithmic (bidirectional)
formulation for the sake of simplicity and generality. Rather than restating the rules, we will simply state the governing
metatheorems and refer the reader to the prior work for the full detalis \cite{HazelnutLive}.

First, we have that an elaboration always exists and it preserves typing.
\begin{theorem}[Typed Elaboration]
    If $\hasType{\Gamma}{e}{\tau}$ then $\elabs{\Gamma}{e}{d}{\tau}{\Delta}$ for some $d$ and $\Delta$ such
    that $\hasTypeD{\Delta}{\Gamma}{d}{\tau}$.
\end{theorem}

Next, we have that evaluation of an expression with holes leads to a final (i.e. irreducible) result of the same type (this is a corollary
of the type safety properties established in the prior work.)
\begin{theorem}[Preservation]
    If $\hasTypeD{\Delta}{\cdot}{d}{\tau}$ and $\evalsTo{d}{d'}$ then $\isFinal{d'}$ and $\hasTypeD{\Delta}{\cdot}{d'}{\tau}$.
\end{theorem}

\subsection{Expansion}

\begin{figure}
    \[
    \arraycolsep=4pt\begin{array}{llcl}
        \mathsf{Typ} & \tau & ::= &
                                    \tnum ~\vert~
                                    \tarr{\tau_1}{\tau_2} ~\vert~
                                    \tprod{\tau_1}{\tau_2} ~\vert~
                                    \tunit ~\vert~
                                    \tsum{\tau_1}{\tau_2} ~\vert~
                                    t ~\vert~
                                    \trec{t}{\tau}\\
        \mathsf{UExp} & \hat{e} & ::= & \hnum{n} ~\vert~
                                 x ~\vert~
                                 \hlam{x}{\hat e} ~\vert~
                                 \hap{\hat e_1}{\hat e_2} ~\vert~
                                 \cdots ~\vert~
                                %  \hpair{\hat e_1}{\hat e_2} ~\vert~
                                %  \hprl{\hat e} ~\vert~
                                %  \hprr{\hat e} ~\vert~
                                %  \htriv ~\vert~
                                %  \hinL{\hat e} ~\vert~
                                %  \hinR{\hat e} ~\vert~
                                %  \hroll{\hat e} ~\vert~
                                %  \hunroll{\hat e} \\
                                 \hehole{u} ~\vert~
                                 \haplivelit{u}{a}{\dtxt{model}}{\splices}\\
        \mathsf{EExp} & e & ::= & \hnum{n} ~\vert~ x ~\vert~ \hlam{x}{e} ~\vert~ \hap{e_1}{e_2} ~\vert~ \cdots ~\vert~ \hehole{u}\\
        \mathsf{IExp} & d & ::= & \hnum{n} ~\vert~ x ~\vert~ \hlam{x}{d} ~\vert~ \hap{d_1}{d_2} ~\vert~ \cdots ~\vert~ \dehole{u}{\sigma}\\
        \mathsf{Splice} & \psi & ::= & \hsplice{\hat e}{\tau}
    \end{array}
    \]
    \caption{Syntax of types, $\tau$, unexpanded expressions, $\hat{e}$, expanded expressions, $e$, and internal expressions, $d$.
    Metavariable $x$ ranges over variables, $u$ over hole names, and $\livelitname{a}$ over livelit names.
    We write $\splat{\psi_i}$ for a finite sequence of $n \geq 0$ splices,
    and $\sigma$ for finite substitutions of $n \geq 0$ internal expressions for variables, $[d_1/x_1, \cdots, d_n/x_n]$.
    We elide standard expression forms
    related to product, sum, and recursive types.
    }
    \label{fig:syntax}
    \end{figure}

    \begin{figure}
    \begin{mathpar}
        \inferrule[EVar]{
            x : \tau \in \Gamma
        }{
            \expands{\Phi}{\Gamma}{x}{x}{\tau}
        }

        \inferrule[ELam]{
            \expands{\Phi}{\Gamma, x : \taut{in}}{\hat e}{e}{\taut{out}}
        }{
            \expands{\Phi}{\Gamma}{\hlam{x}{\hat e}}{\hlam{x}{e}}{\tarr{\taut{in}}{\taut{out}}}
        }

        \inferrule[EAp]{
            \expands{\Phi}{\Gamma}{\hat e_1}{e_1}{\tarr{\taut{in}}{\taut{out}}}\\\\
            \expands{\Phi}{\Gamma}{\hat e_2}{e_2}{\taut{in}}
        }{
            \expands{\Phi}{\Gamma}{\hap{\hat e_1}{\hat e_2}}{\hap{e_1}{e_2}}{\taut{out}}
        }

        \cdots

        \inferrule[EApLivelit]{
            \livelitCtxEntry{a}{\taut{expand}}{\taut{model}}{\dtxt{expand}} \in \Phi\\\\
            \hasType{ }{\dtxt{model}}{\taut{model}}\\
            \evalsTo{\hap{\dtxt{expand}}{\dtxt{model}}}{\dtxt{encoded}}\\
            \decodeExp{\dtxt{encoded}}{\etxt{pexpansion}}\\\\
            \hasType{ }{\etxt{pexpansion}}{\tarr{\{\tau_i\}_{i < n}}{\taut{expand}}}\\
            \{ \expands{\Phi}{\Gamma}{\hat e_i}{e_i}{\tau_i} \}_{i < n}
        }{
            \expands{\Phi}{\Gamma}{\haplivelit{u}{a}{\dtxt{model}}{\splat{\hsplice{\hat e_i}{\tau_i}}}}{
                \hap{\etxt{pexpansion}}{\{e_i\}_{i < n}}
            }{\taut{expand}}
        }
    \end{mathpar}
    \caption{Expansion}
    \label{fig:expansion}
    \end{figure}
The novelty of the livelit calculus lies entirely in its handling of unexpanded expressions,
$\hat e$, which are given meaning by typed expansion to external expressions,
$e$, according to the judgement $\expands{\Phi}{\Gamma}{\hat e}{e}{\tau}$
defined in Fig.~\ref{fig:expansion}.
Unexpanded expressions mirror external expressions but for the presence of livelits, discussed below.
The rules for standard
forms like variables, functions and function application, shown in Fig.~\ref{fig:expansion},
are straightforward.

\subsubsection{Livelit Contexts}
Livelit definitions are collected in the livelit context, $\Phi$, which
maps livelit names to livelit definitions of the following form:
\[ \livelitCtxEntry{a}{\taut{expand}}{\taut{model}}{\dtxt{expand}} \]
Here, $\taut{expand}$ is the expansion type, $\taut{model}$ is the model type,
and $\dtxt{expand}$ is the expansion function, which generates an expansion given a model.\todo{add view and update}

For a livelit context to be well-formed, we require that each livelit definition be well-formed.
This in turn requires that the expansion function be hole-free and have the correct type.
\begin{definition}[Livelit Context Well-Formedness]
    A livelit context $\Phi$ is well-formed if and only if for each livelit definition,
    $\livelitCtxEntry{a}{\taut{expand}}{\taut{model}}{\dtxt{expand}} \in \Phi$,  we have
    $\hasType{}{\dtxt{expand}}{\tarr{\taut{model}}{\expType}}$.
\end{definition}
Here, $\expType$ stands for a type whose values isomorphically encode
external expressions. The isomorphism is mediated in one direction by
the encoding judgement $\encodeExp{e}{d}$ and the other by the decoding judgement $\decodeExp{d}{e}$.
Any isomorphism is sufficient, so we leave the details as a matter of implementation.
The simplest approach would be to define $\expType$ as a recursive sum type,
with one arm for each form of external expression (cf. \cite{TSLs} for an example).

For simplicity, we assume that the livelit context is provided \emph{a priori} and therefore
that functions like $\dtxt{expand}$ are already closed and fully elaborated.
In practice, the livelit context would be controlled by a definition form in the language
that allows the expansion function to itself use livelits.
This would require a staging mechanism,
because we need to execute expansion functions in prior definitions to be able to
expand subsequent definitions.
There are a number of ways to support the necessary staging in practice, e.g.
via explicit staging primitives \todo{cite ocaml macros paper, Racket stuff}{},
by requiring that these definitions appear in separately compiled packages \cite{TLMs},
or by using live
programming mechanisms such as those available in Hazel
to evaluate ``up to'' each definition before proceeding.
This choice is orthogonal to the mechanisms of interest in this section.

\subsubsection{Hygienic Livelit Expansion}
Unexpanded expressions are unique in that they include livelits:
 \[\haplivelit{u}{a}{\dtxt{model}}{\splices}\]
Here, $\livelitname{a}$ names the livelit
 being applied. Livelits can be understood as filling holes, so $u$ identifies the hole
 that is, conceptually, being filled.
The current state of the livelit is determined by the current model value, $\dtxt{model}$,
together with a collection of splices, $\splices$. Each splice $\psi_i$
is of the form $\hsplice{\hat e_i}{\tau_i}$, where $\hat e_i$ is the spliced expression
itself (unexpanded, so it may contain other livelits) and $\tau_i$ is the type of that splice,
as determined when the livelit definition first requested the splice (discussed in Sec.~\ref{sec:calculus-updates}).

Rule \rulename{EApLivelit} performs livelit expansion. Its premises, in order, operate as follows:
\begin{enumerate}
    \item \textbf{Lookup.} The first premise looks up the livelit name in the livelit context.
    \item \textbf{Model Validation.} The second premise serves to ensure that the model value, $\dtxt{model}$, is of the
    specified model type, $\taut{model}$.
    \item \textbf{Expansion.} The third premise applies the expansion function, $\dtxt{expand}$, to the model value, $\dtxt{model}$,
    producing the encoded expansion, $\dtxt{encoded}$, which, by the definitions given above, is of type $\mathsf{Exp}$.
    \item \textbf{Decoding.} The fourth premise decodes $\dtxt{encoded}$, producing the external expression it encodes, $\etxt{pexpansion}$.
    \item \textbf{Expansion Validation.} We call $\etxt{expansion}$ the \emph{parameterized expansion} because, according to the fifth premise,
    it must be a function that returns a value of the expansion type, $\taut{expand}$, when applied (in curried fashion, though this is not critical)
    to the splices, whose types, $\splat{\tau_i}$, are explicitly recorded in the livelit.
    We must take care to ensure that the parameterized expansion is \emph{context independent},
    i.e. that it cannot depend on the particular bindings available in the call site typing context, $\Gamma$.
    Context independence is one facet of what is colloquially known as \emph{hygiene} \cite{TLMs}.
    In this simple formulation of the system, we maintain context independence simply by
    requiring that the parameterized expansion be
    closed. Consequently, any necessary helper functions used in the expansion must be provided explicitly by the client
    via a splice. In Sec.~\ref{sec:calculus-parameterization}, we will see how this context independence
    restriction can be made
    less onerous by passing definition-site bindings along as partially applied parameters.\todo{remove?}{}
    \item \textbf{Splice Expansion.} Finally, the sixth premise inductively expands each of the spliced expressions in the same context as the livelit
    expression itself.
\end{enumerate}
The conclusion of the rule then applies the parameterized expansion to the expanded splices.
By applying the splices as arguments, we maintain \emph{capture avoidance} -- splices cannot capture variables
bound internally to the expansion. Capture avoidance is the other facet of \emph{hygiene} \cite{TLMs}.

The typed expansion process is governed by the following metatheorem, which establishes that the expansion
is indeed an external expression of the indicated type.\todo{change to proof sketch}{}

\begin{theorem}[Typed Expansion]
    If $\expands{\Phi}{\Gamma}{\hat e}{e}{\tau}$ then $\hasType{\Gamma}{e}{\tau}$.
\end{theorem}
\begin{proof}
    We proceed by rule induction on the assumption.
    The cases involving the standard forms follow by straightforward induction.
    The only interesting case is the \rulename{EApLivelit} case.
    In this case, we have that $e = \hap{\etxt{pexpansion}}{\splat{e_i}}$.
    By assumption, we have $\{ \expands{\Phi}{\Gamma}{\hat e_i}{e_i}{\tau_i} \}_{i < n}$.
    By induction, we therefore have $\{ \hasType{\Gamma}{e_i}{\tau_i} \}_{i < n}$.
    In addition, by assumption we have that $\hasType{}{\etxt{pexpansion}}{\tarr{\splat{\tau_i}}{\taut{expand}}}$.
    By weakening, we therefore have $\hasType{\Gamma}{\etxt{pexpansion}}{\tarr{\splat{\tau_i}}{\taut{expand}}}$.
    Finally, by iterated application of the typing rule for function application over the splices,
    using the judgements just established,
    we have $\hasType{\Gamma}{\hap{\etxt{pexpansion}}{\splat{e_i}}}{\taut{expand}}$ as desired.
\end{proof}

When composed with the Typed Elaboration theorem and the Type Safety properties established for the internal
language, we achieve end-to-end type safety for unexpanded expressions (every well-typed unexpanded
expression expands to a well-typed external expression, which in turn expands to a well-typed internal
expression, which evaluates in a type safe manner.)

\subsubsection{Agda Mechanization}
We have mechanized the typed expansion mechanism and proven the Typed Expansion theorem
using the Agda proof assistant.\todo{say more}{}

\subsection{Live Feedback via Closure Collection}
In order to support live feedback, a livelit needs to be able to ask the system
to evaluate expressions under one of the closures associated with the livelit.
This mechanism was introduced by example in Sec.~\ref{sec:image-transformation}.
In this section, we will formalize the process of efficiently collecting closures
for the livelits that appear in a program.

The key idea is that we generate an alternative expansion,
called the \emph{cc-expansion},
where each livelit expands to an empty hole applied to the splices. In other words,
we leave a hole in place of the parameterized expansion, which we expand and record in the \emph{cc-context}, $\Omega$, on the side.
The key rule for the cc-expansion judgement, $\ccexpands{\Phi}{\Gamma}{\hat e}{e}{\tau}{\Omega}$, is:
\begin{mathpar}
\inferrule[CCApLivelit]{
    \{ \ccexpands{\Phi}{\Gamma}{\hat e_i}{e_i}{\tau_i}{\Omega_i} \}_{i < n}\\
    \expands{\Phi}{\Gamma}{\haplivelit{u}{a}{\dtxt{model}}{\splat{\hsplice{\hat e_i}{\tau_i}}}}{
        \hap{\etxt{pexpansion}}{\splat{e_i}}
    }{\taut{expand}}\\
    \elabs{ }{\etxt{pexpansion}}{\dtxt{pexpansion}}{\tarr{\splat{\tau_i}}{\taut{expand}}}{\Delta}
}{
    \ccexpands{\Phi}{\Gamma}{\haplivelit{u}{a}{\dtxt{model}}{\splat{\hsplice{\hat e_i}{\tau_i}}}}{\hap{\hehole{u}}{\splat{e_i}}}{\taut{expand}}
    {\cup_{i<n} \Omega_i \cup \{\Omegaitem{u}{\dtxt{pexpansion}}\}}
}
\end{mathpar}

We then elaborate and evaluate the cc-expansion
following Hazelnut Live's dynamic semantics for programs with holes, which were summarized
in Sec.~\ref{sec:external-and-internal-lang}. The result will contain some number of hole closures
for each livelit hole. We call these the proto-closures and their associated environments the proto-environments for that livelit hole, formally defined as follows.
\begin{definition}[Proto-Environment Collection]
If $\ccexpands{\Phi}{\cdot}{\hat e}{e}{\tau}{\Omega}$ and $\elabs{}{e}{d}{\tau}{\Delta}$
and $\evalsTo{d}{d'}$ and $u \in \domof{\Omega}$ then $\protoEnvsOf{\Phi}{\hat e}{u} = \{ \sigma \mid \dehole{u}{\sigma} \in d' \}$.
    % protoclosures(\hat e, u) = { sigma | hole^u_sigma in d' } where \hat e cc-expands to e and e elaborates to d and d evaluates to d'
\end{definition}
A proto-environment for a livelit hole might itself contain a proto-closure for another livelit hole.
For example, in Fig.~\ref{fig:color}, the proto-environment for the \li{\$color} livelit contained the
proto-closure for the \li{\$slider} livelit, via the \li{gray\_level} variable.
If we use proto-environments for live evaluation, then the actual value of that variable
would not be available (here, the color could not be displayed).

Consequently, the second step of closure selection is to fill any livelit holes that appear in the proto-environments for other livelit holes.
We do so by filling them using the parameterized expansions gathered in $\Omega$ and then resuming
evaluation where appropriate.
Formally, this involves the hole filling operation $\instantiate{d_1}{u}{d_2}$ for Hazelnut Live
(which derives from the metavariable instantation operation of contextual modal type theory \cite{HazelnutLive,Nanevski2008}).
This operation
fills every closure for hole $u$ in $d_2$ with $d_1$.
Unlike substitution, hole filling is
not capture-avoiding. Instead, the environment on each of these closures is applied to $\dtxt{pexpansion}$
as a substitution, i.e. the delayed substitutions captured in the environment are realized.
In this case, however, the parameterized expansion is guaranteed to be closed due to
the context independence discipline we maintain in Rule \rulename{EApLivelit},
so hole filling acts simply as a syntactic replacement.

Formally, we begin by defining an operation $\fillof{\Omega}{\sigma}$ which acts on proto-environments
to fill any livelit holes that appear.
\begin{definition}[Livelit Hole Filling] ~
    \begin{enumerate}
        \item $\fillof{\Omega}{[d_1/x_1, \ldots, d_n/x_n]} = [\fillof{\Omega}{d_1}/x_1, \ldots, \fillof{\Omega}{d_n}/x_n]$
        \item $\fillof{\Omega}{d} = \instantiateB{\dtxt{pexpansion}}{u}_{\Omegaitem{u}{\dtxt{pexpansion}} \in \Omega}{d}$
    \end{enumerate}
\end{definition}

This first step may cause certain expressions to become non-final, because the filled hole is no longer
blocking evaluation. We therefore define an operation $\resumeof{\sigma}$ that resumes evalution for all closed expressions in $\sigma$.
(Open expressions can remain because some closures appear under binders in the final result, so a substitution has not yet been recorded
for the bound variables.)
\begin{definition}[Environment Resumption] ~
    \begin{enumerate}
        \item $\resumeof{[d_1/x_1, \ldots, d_n/x_n]} = [\resumeof{d_1}/x_1, \ldots, \resumeof{d_n}/x_n]$
        \item $\resumeof{d} = d'$ if $\fvof{d} = \emptyset$ and $\evalsTo{d}{d'}$
        \item $\resumeof{d} = d$ if $\fvof{d} \neq \emptyset$
    \end{enumerate}
\end{definition}

Finally, we can produce the final set of environments by filling and resuming the proto-environments.
\begin{definition}[Environment Collection]
    If $\ccexpands{\Phi}{\cdot}{\hat e}{e}{\tau}{\Omega}$
    % and $\elabs{}{e}{d}{\tau}{\Delta}$
    % and $\evalsTo{d}{d'}$ and $u \in \domof{\Omega}$
    then \[\envsOf{\Phi}{\hat e}{u} = \{\resumeof{\fillof{\Omega}{\sigma}} \mid \sigma \in \protoEnvsOf{\Phi}{\hat e}{u}\}\]
\end{definition}

This same fill and resume operation can be used to avoid recomputation when evaluating the fully expanded version of the user's program.
If the editor has already performed environment collection, then it can simply continue from where it left off
by filling and resuming
the remaining top-level livelit holes (those that do not appear in a proto-environment).

The correctness of the mechanisms described in this section rest on the fact that evaluation commutes with hole filling.
This only holds when the language is pure: evaluation order does not matter
so the system is free to evaluate around holes initially,
and then return to finish the job once they have been filled.

\begin{theorem}[Post-Collection Resumption]
    If $\ccexpands{\Phi}{\cdot}{\hat e}{\etxt{cc}}{\tau}{\Omega}$ and $\elabs{}{\etxt{cc}}{\dtxt{cc}}{\tau}{\Delta}$
    and $\evalsTo{\dtxt{cc}}{\dtxt{cc}'}$ and $\resumeof{\fillof{\Omega}{\dtxt{cc}'}} = d_1$
    and $\expands{\Phi}{\cdot}{\hat e}{\etxt{full}}{\tau}$
    and $\elabs{}{\etxt{full}}{\dtxt{full}}{\tau}{\Delta}$
    and $\evalsTo{\dtxt{full}}{d_2}$ then $d_1 = d_2$.
\end{theorem}
\begin{proof}
    The key observation is that filling the livelit holes in the cc-expansion gives the full expansion,
    i.e. $\fillof{\Omega}{\dtxt{cc}} = \dtxt{full}$. Resumption is simply evaluation for closed expressions.
    By commutativity of hole filling, established in the prior work \cite{HazelnutLive},
    we can delay hole filling until $\dtxt{cc}$ has first been evaluated to
    $\dtxt{cc}'$.
\end{proof}

In a language with side effects, one would instead need to modify evaluation of
cc-expansions such that the full expansion of each livelit application is evaluated at the same time as the
temporary expansion,
with the result stored for subsequent hole filling. This would ensure that
side effects happen in the same order and only once. The full details are beyond the scope of this work.

\subsection{Parameterization}

Add livelit expressions, livelit abbreviations,
partial application support. Talk about how to work around the strict context
independence discipline we have imposed.

\subsection{Edit Action Semantics}
So far, we have considered only the semantics of an individual editor state.
However, livelits are useful because they evolve as the user interacts
with them. In particular, the livelit's model, $\dtxt{model}$ can evolve
due to livelit-specific user interactions. Furthermore, a livelit can also
create and delete splices when a user interaction demands it (unlike functions, which have a fixed
number of arguments).

