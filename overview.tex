\section{Overview Examples: Using Palettes}
\label{sec:overview}

\begin{figure*}

\includegraphics[scale=0.50]{images/MrSmileyFace.png}

\caption{Example Palettes in Hazel.}
\end{figure*}

\begin{itemize}

\item figure with 2/3 Hazel palettes:

  \begin{enumerate}
    \item table (synthetic)
    \item grade cutoffs (analytic)
    \item color? (analytic)
  \end{enumerate}

\item highlight persistence

\item highlight composition

\item two dimensions for palette definitions:

  \begin{enumerate}
    \item macros (expressions) vs. functions (values)
    \item analytic (annotated) vs. synthetic (unannotated)
      \begin{itemize}
        \item synthetic requires type splices
        \item an ``untyped'' version that is like synthetic but relies on casts rather than type splices?
      \end{itemize}
  \end{enumerate}

\item live environments (for view function)

  \begin{itemize}
    \item hole for palette application, run, collect hole closures, pass to view
  \end{itemize}

\item two prototype UIs:

  \begin{enumerate}
    \item \Hazel{}: (palette macros;) structure edit actions; expression formula bar; inline layout
    \item \sns{}: (palette functions;) text editor; pop-up menu
  \end{enumerate}

\item (nice-to-have: parameterized palettes; partial application)

\item (nice-to-have: palette definitions in source lang)

\end{itemize}
