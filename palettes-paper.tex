\PassOptionsToPackage{svgnames,dvipsnames,svgnames}{xcolor}

% From https://pldi19.sigplan.org/track/pldi-2019-papers#Call-for-Papers
\documentclass[sigplan,10pt,review,anonymous]{acmart}
\settopmatter{printfolios=true,printccs=false,printacmref=false}

%\documentclass[acmsmall,review,anonymous]{acmart}

%% For double-blind review submission, w/o CCS and ACM Reference (max submission space)
%% \documentclass[sigplan,review,anonymous]{acmart}\settopmatter{printfolios=true,printccs=false,printacmref=false}
%% For double-blind review submission, w/ CCS and ACM Reference
%\documentclass[sigplan,review,anonymous]{acmart}\settopmatter{printfolios=true}
%% For single-blind review submission, w/o CCS and ACM Reference (max submission space)
%\documentclass[sigplan,review]{acmart}\settopmatter{printfolios=true,printccs=false,printacmref=false}
%% For single-blind review submission, w/ CCS and ACM Reference
%\documentclass[sigplan,review]{acmart}\settopmatter{printfolios=true}
%% For final camera-ready submission, w/ required CCS and ACM Reference
%\documentclass[sigplan]{acmart}\settopmatter{}


%% Conference information
%% Supplied to authors by publisher for camera-ready submission;
%% use defaults for review submission.
\acmConference[PL'18]{ACM SIGPLAN Conference on Programming Languages}{January 01--03, 2018}{New York, NY, USA}
\acmYear{2018}
\acmISBN{} % \acmISBN{978-x-xxxx-xxxx-x/YY/MM}
\acmDOI{} % \acmDOI{10.1145/nnnnnnn.nnnnnnn}
\startPage{1}

%% Copyright information
%% Supplied to authors (based on authors' rights management selection;
%% see authors.acm.org) by publisher for camera-ready submission;
%% use 'none' for review submission.
\setcopyright{none}
%\setcopyright{acmcopyright}
%\setcopyright{acmlicensed}
%\setcopyright{rightsretained}
%\copyrightyear{2018}           %% If different from \acmYear

%% Bibliography style
\bibliographystyle{ACM-Reference-Format}
%% Citation style
%\citestyle{acmauthoryear}  %% For author/year citations
%\citestyle{acmnumeric}     %% For numeric citations
%\setcitestyle{nosort}      %% With 'acmnumeric', to disable automatic
                            %% sorting of references within a single citation;
                            %% e.g., \cite{Smith99,Carpenter05,Baker12}
                            %% rendered as [14,5,2] rather than [2,5,14].
%\setcitesyle{nocompress}   %% With 'acmnumeric', to disable automatic
                            %% compression of sequential references within a
                            %% single citation;
                            %% e.g., \cite{Baker12,Baker14,Baker16}
                            %% rendered as [2,3,4] rather than [2-4].


%% Some recommended packages.
\usepackage{booktabs}   %% For formal tables:
                        %% http://ctan.org/pkg/booktabs
\usepackage{subcaption} %% For complex figures with subfigures/subcaptions
                        %% http://ctan.org/pkg/subcaption

%% Cyrus packages
\usepackage{microtype}
\usepackage{mdframed}
\usepackage{colortab}
\usepackage{mathpartir}
\usepackage{enumitem}
\usepackage{bbm}
\usepackage{stmaryrd}
\usepackage{mathtools}
\usepackage{leftidx}
\usepackage{todonotes}
\usepackage{xspace}
\usepackage{wrapfig}


\usepackage{listings}%
\lstloadlanguages{ML}
\lstset{tabsize=2, 
basicstyle=\footnotesize\ttfamily, 
% keywordstyle=\sffamily,
commentstyle=\itshape\ttfamily\color{gray}, 
stringstyle=\ttfamily\color{purple},
mathescape=false,escapechar=\#,
numbers=left, numberstyle=\scriptsize\color{gray}\ttfamily, language=ML, showspaces=false,showstringspaces=false,xleftmargin=15pt, 
morekeywords={string, float, int},
classoffset=0,belowskip=\smallskipamount, aboveskip=\smallskipamount,
moredelim=**[is][\color{red}]{SSTR}{ESTR}
}
\newcommand{\li}[1]{\lstinline[basicstyle=\ttfamily\fontsize{9pt}{1em}\selectfont]{#1}}
\newcommand{\lismall}[1]{\lstinline[basicstyle=\ttfamily\fontsize{9pt}{1em}\selectfont]{#1}}

%% Joshua Dunfield macros
\def\OPTIONConf{1}%
\usepackage{joshuadunfield}

%% Can remove this eventually
\usepackage{blindtext}

\usepackage{enumitem}

%%%%%%%%%%%%%%%%%%%%%%%%%%%%%%%%%%%%%%%%%%%%%%%%%%%%%%%%%%%%%%%%%%%%%%%%%%%%%
%% Matt says: Cyrus, this package `adjustbox` seems directly related
%% to the `clipbox` error; To get rid of the error, I moved it last
%% (after other usepackages) and I added the line just above it, which
%% permits it to redefine `clipbox` (apparently also defined in
%% `pstricks`, and due to latex's complete lack of namespace
%% management, these would otherwise names clash).
\let\clipbox\relax
\usepackage[export]{adjustbox}% http://ctan.org/pkg/adjustbox
%%%%%%%%%%%%%%%%%%%%%%%%%%%%%%%%%%%%%%%%%%%%%%%%%%%%%%%%%%%%%%%%%%%%%%%%%%%%%%%%%


%%%%%%%%%%%%%%%%%%%%%%%%%%%%%%%%%%%%%%%%%%%%%%%%%%%%%%%%%%%%%%%%%%%%%%%%%%%%%%%%%
%\usepackage{draftwatermark}
%\SetWatermarkText{DRAFT}
%\SetWatermarkScale{1}
%%%%%%%%%%%%%%%%%%%%%%%%%%%%%%%%%%%%%%%%%%%%%%%%%%%%%%%%%%%%%%%%%%%%%%%%%%%%%%%%%


% A macro for the name of the system being described by ``this paper''
\newcommand{\HazelnutLive}{\textsf{Hazelnut Live}\xspace}
\newcommand{\Hazelnut}{\textsf{Hazelnut}\xspace}
% The mockup, work-in-progress system.
\newcommand{\Hazel}{\textsf{Hazel}\xspace}

% \newtheorem{theorem}{Theorem}[chapter]
% \newtheorem{lemma}[theorem]{Lemma}
% \newtheorem{corollary}[theorem]{Corollary}
% \newtheorem{definition}[theorem]{Definition}
% \newtheorem{assumption}[theorem]{Assumption}
% \newtheorem{condition}[theorem]{Condition}

\newtheoremstyle{slplain}% name
  {.15\baselineskip\@plus.1\baselineskip\@minus.1\baselineskip}% Space above
  {.15\baselineskip\@plus.1\baselineskip\@minus.1\baselineskip}% Space below
  {\slshape}% Body font
  {\parindent}%Indent amount (empty = no indent, \parindent = para indent)
  {\bfseries}%  Thm head font
  {.}%       Punctuation after thm head
  { }%      Space after thm head: " " = normal interword space;
        %       \newline = linebreak
  {}%       Thm head spec
\theoremstyle{slplain}
\newtheorem{thm}{Theorem}  % Numbered with the equation counter
\numberwithin{thm}{section}
\newtheorem{defn}[thm]{Definition}
\newtheorem{lem}[thm]{Lemma}
\newtheorem{prop}[thm]{Proposition}
% \newtheorem{cor}[section]{Corollary}     
% \newtheorem{lem}[section]{Lemma}         
% \newtheorem{prop}[section]{Proposition}  

% \setlength{\abovedisplayskip}{0pt}
% \setlength{\belowdisplayskip}{0pt}
% \setlength{\abovedisplayshortskip}{0pt}
% \setlength{\belowdisplayshortskip}{0pt}



\makeatletter\if@ACM@journal\makeatother
%% Journal information (used by PACMPL format)
%% Supplied to authors by publisher for camera-ready submission
\acmJournal{PACMPL}
\acmVolume{1}
\acmNumber{1}
\acmArticle{1}
\acmYear{2018}
\acmMonth{3}
\acmDOI{10.1145/nnnnnnn.nnnnnnn}
\startPage{1}
\else\makeatother
%% Conference information (used by SIGPLAN proceedings format)
%% Supplied to authors by publisher for camera-ready submission
% \acmConference[]{ACM SIGPLAN Conference on Programming Languages}{January 01--03, 2017}{New York, NY, USA}

\acmYear{2018}
\acmISBN{978-x-xxxx-xxxx-x/YY/MM}
\acmDOI{10.1145/nnnnnnn.nnnnnnn}
\startPage{1}
\fi


%% Copyright information
%% Supplied to authors (based on authors' rights management selection;
%% see authors.acm.org) by publisher for camera-ready submission
\setcopyright{none}             %% For review submission
%\setcopyright{acmcopyright}
%\setcopyright{acmlicensed}
%\setcopyright{rightsretained}
%\copyrightyear{2017}           %% If different from \acmYear


\fancyfoot{} % suppresses the footer (also need \thispagestyle{empty} after \maketitle below)


%% Bibliography style
\bibliographystyle{ACM-Reference-Format}
%% Citation style
%% Note: author/year citations are required for papers published as an
%% issue of PACMPL.
\citestyle{acmauthoryear}   %% For author/year citations

\input{commands}
\input{macros}

\setlength{\abovecaptionskip}{4pt plus 3pt minus 2pt} % Chosen fairly arbitrarily
\setlength{\belowcaptionskip}{-4pt plus 3pt minus 2pt} % Chosen fairly arbitrarily


\begin{document}

%% Title information
%% \title{Live Hole Filling by Direct Manipulation}         %% [Short Title] is optional;
% \title{Nested Palettes:\\Programmable GUIs for Programming}
\title{Live and Direct Functional Programming \\with Palette Expressions}
%% \title{Nested Palettes: Composable GUIs for Programming}

                                        %% when present, will be used in
                                        %% header instead of Full Title.
% \titlenote{with title note}             %% \titlenote is optional;
                                        %% can be repeated if necessary;
                                        %% contents suppressed with 'anonymous'
% \subtitle{Subtitle}                     %% \subtitle is optional
% \subtitlenote{with subtitle note}       %% \subtitlenote is optional;
                                        %% can be repeated if necessary;
                                        %% contents suppressed with 'anonymous'


%% Author information
%% Contents and number of authors suppressed with 'anonymous'.
%% Each author should be introduced by \author, followed by
%% \authornote (optional), \orcid (optional), \affiliation, and
%% \email.
%% An author may have multiple affiliations and/or emails; repeat the
%% appropriate command.
%% Many elements are not rendered, but should be provided for metadata
%% extraction tools.

%% Author with single affiliation.
\author{Cyrus Omar}
% \authornote{with author1 note}          %% \authornote is optional;
                                        %% can be repeated if necessary
% \orcid{nnnn-nnnn-nnnn-nnnn}             %% \orcid is optional
\affiliation{
  % \position{Position1}
  % \department{Department1}              %% \department is recommended
  \institution{University of Chicago}            %% \institution is required
  % \streetaddress{Street1 Address1}
  % \city{City1}
  % \state{State1}
  % \postcode{Post-Code1}
  % \country{Country1}
}
\email{comar@cs.uchicago.edu}          %% \email is recommended

\author{Ian Voysey}
% \authornote{with author1 note}          %% \authornote is optional;
                                        %% can be repeated if necessary
% \orcid{nnnn-nnnn-nnnn-nnnn}             %% \orcid is optional
\affiliation{
  % \position{Position1}
  % \department{Department1}              %% \department is recommended
  \institution{Carnegie Mellon University}            %% \institution is required
  % \streetaddress{Street1 Address1}
  % \city{City1}
  % \state{State1}
  % \postcode{Post-Code1}
  % \country{Country1}
}
\email{iev@cs.cmu.edu}          %% \email is recommended

\author{Ravi Chugh}
% \authornote{with author1 note}          %% \authornote is optional;
                                        %% can be repeated if necessary
% \orcid{nnnn-nnnn-nnnn-nnnn}             %% \orcid is optional
\affiliation{
  % \position{Position1}
  % \department{Department1}              %% \department is recommended
  \institution{University of Chicago}            %% \institution is required
  % \streetaddress{Street1 Address1}
  % \city{City1}
  % \state{State1}
  % \postcode{Post-Code1}
  % \country{Country1}
}
\email{rchugh@cs.uchicago.edu}          %% \email is recommended

\author{Matthew A. Hammer}
% \authornote{with author1 note}          %% \authornote is optional;
                                        %% can be repeated if necessary
% \orcid{nnnn-nnnn-nnnn-nnnn}             %% \orcid is optional
\affiliation{
  % \position{Position1}
  % \department{Department1}              %% \department is recommended
  \institution{University of Colorado Boulder}            %% \institution is required
  % \streetaddress{Street1 Address1}
  % \city{City1}
  % \state{State1}
  % \postcode{Post-Code1}
  % \country{Country1}
}
\email{matthew.hammer@colorado.edu}          %% \email is recommended


% %% Author with two affiliations and emails.
% \author{First2 Last2}
% \authornote{with author2 note}          %% \authornote is optional;
%                                         %% can be repeated if necessary
% \orcid{nnnn-nnnn-nnnn-nnnn}             %% \orcid is optional
% \affiliation{
%   \position{Position2a}
%   \department{Department2a}             %% \department is recommended
%   \institution{Institution2a}           %% \institution is required
%   \streetaddress{Street2a Address2a}
%   \city{City2a}
%   \state{State2a}
%   \postcode{Post-Code2a}
%   \country{Country2a}
% }
% \email{first2.last2@inst2a.com}         %% \email is recommended
% \affiliation{
%   \position{Position2b}
%   \department{Department2b}             %% \department is recommended
%   \institution{Institution2b}           %% \institution is required
%   \streetaddress{Street3b Address2b}
%   \city{City2b}
%   \state{State2b}
%   \postcode{Post-Code2b}
%   \country{Country2b}
% }
% \email{first2.last2@inst2b.org}         %% \email is recommended


%% Paper note
%% The \thanks command may be used to create a "paper note" ---
%% similar to a title note or an author note, but not explicitly
%% associated with a particular element.  It will appear immediately
%% above the permission/copyright statement.
% \thanks{with paper note}                %% \thanks is optional
                                        %% can be repeated if necesary
                                        %% contents suppressed with 'anonymous'


%% Abstract
%% Note: \begin{abstract}...\end{abstract} environment must come
%% before \maketitle command
% !TEX root = livelits-paper.tex

%% no citations in Abstract
%%
%% \cite{popl-paper}

\begin{abstract}
    Text editing is powerful, but 
    some types of expressions are more naturally represented and manipulated
    graphically.
    Examples include expressions that compute 
    colors,
    music,
    animations,
    tabular data, 
    plots,
    GUI widgets, 
    mathematical diagrams, 
    and other domain-specific data structures.
    This paper introduces \emph{live literals}, or \emph{livelits}, 
    which allow
    clients to fill holes of types like these   
    by directly manipulating a     
    user-defined GUI embedded persistently into code. 
    Uniquely, livelits are \emph{compositional}: a livelit GUI can itself embed 
    spliced expressions, which are typed, lexically scoped, enjoy full editor support and can in turn embed 
    other livelits. 
    Livelits are also uniquely \emph{live}: 
    a livelit can provide {continuous feedback} about the run-time implications 
    of the client's choices 
    even when splices mention bound variables, 
    because the system continuously gathers closures associated with the 
    hole that the livelit is tasked with filling.
    We integrate livelits into Hazel, 
    a live programming environment able to 
    typecheck and run programs with holes, 
    and describe several case studies that exercise these novel capabilities. 
    We then define a minimal typed livelit calculus, which precisely 
    specifies how livelits operate as 
    live graphical macros.     
    The metatheory of macro expansion has been mechanized in Agda.
\end{abstract}

\newcommand{\fmap} [3] {\{ #1 ~ \rotatebox[origin=c]{180}{$\Lsh$} ~ #3 \}_{#1 \in #2}}

% Ordinary (value) types (ty-)
\newcommand{\htyp}[0]{\tau} % Meta variable for types
\newcommand{\htypc}[0]{\tau_c} % Meta variable for types with cells

% Expression forms (e-)
\newcommand{\hexp}{\iota} % Hexp without cells or palettes
\newcommand{\hexpc}{\iota_c} % Hexps with cells that can be substituted with pexps
\newcommand{\pexp}{\rho} % Meta variable for palette expressions
\newcommand{\cell}[1]{$Cell_{#1}$}
\newcommand{\pexpPalLet}[3]{\keyword{let palette}\,#1\,\keyword{=}\,#2\,\keyword{in}\,#3}
\newcommand{\pexpPalAp}[4]{#2\,\keyword{as palette}\,#1\,\keyword{where}\,\fmap{#3}{#4}{\pexp}}

% Palette definitions
\newcommand{\pDef}{\pi} % Meta variable for palette definitions
\newcommand{\pDefRecord}[3]{\{ \keyword{expand}: #1, \, \keyword{modelTyp}: #2, \, \keyword{expandTyp}: #3\}}



%% 2012 ACM Computing Classification System (CSS) concepts
%% Generate at 'http://dl.acm.org/ccs/ccs.cfm'.
% \begin{CCSXML}
% <ccs2012>
% <concept>
% <concept_id>10011007.10011006.10011008</concept_id>
% <concept_desc>Software and its engineering~General programming languages</concept_desc>
% <concept_significance>500</concept_significance>
% </concept>
% <concept>
% <concept_id>10003456.10003457.10003521.10003525</concept_id>
% <concept_desc>Social and professional topics~History of programming languages</concept_desc>
% <concept_significance>300</concept_significance>
% </concept>
% </ccs2012>
% \end{CCSXML}

% \ccsdesc[500]{Software and its engineering~General programming languages}
% \ccsdesc[300]{Social and professional topics~History of programming languages}
%% End of generated code


%% Keywords
%% comma separated list
% \keywords{keyword1, keyword2, keyword3}  %% \keywords is optional
\keywords{Live Programming, Palettes, Hazel, Code Editors}


%% \maketitle
%% Note: \maketitle command must come after title commands, author
%% commands, abstract environment, Computing Classification System
%% environment and commands, and keywords command.
\maketitle
\thispagestyle{empty} % suppresses the footer

\section{Introduction}

Palettes...

\section{Examples}
\subsection{Boolean}
Checkbox

\subsection{Matrix}
Simple example

\subsection{Grade Cutoffs}
Uses liveness more obviously

\subsection{Table}
Uses type reflection

\subsection{Pixel Art}
Cool example

\subsection{Regex}
Similar to Graphite paper, can cite the empirical study we did there

\subsection{Forms}
Show off composition + mention full-screening stuff

\subsection{Equation Editor, Judgement Editor and Category Diagrams}
Things that PL people like 

\subsection{TikZ diagrams}
...

\clearpage
\section{Defining Palettes}
Palette definitions take the following general form:
\begin{lstlisting}
palette $name 
  (arg1 : t1) 
  ... 
  (argn : tn) 
  at t 
  implementation P in package pkg;
\end{lstlisting}
The static semantics requires the following:
\begin{enumerate}
\item \li{arg1} ... \li{argn} for \li{n} $\geq 0$ are distinct labels
\item \li{t1} ...\li{t1} are valid types
\item \li{t} is a valid type
\item and \li{P} identifies a module that can be loaded from package \li{pkg} such that \begin{lstlisting}
P : IPalette
    with type args = { 
      arg1 : HoleRef.t<t1>, 
      ..., 
      argn : HoleRef.t<tn> 
    } 
    with type output = t
\end{lstlisting}
where \li{IPalette} is defined in Fig.~\ref{fig:IPalette} and \li{HoleRef} in Fig.~\ref{fig:HoleRefs}.
\end{enumerate}

\begin{figure}
\begin{lstlisting}
module type IPalette = {
  type args
  type output
  type model
  type msg
  val init    : args -> HRG.t(model)
  val update  : (model, msg) -> HRG.t(model)
  val view    : model -> HRE.t(Html.t(msg))
  val compute : model -> HRE.t(output)
}
\end{lstlisting}
\caption{\li{IPalette} module type}
\label{fig:IPalette}
\end{figure}

\begin{figure}
\begin{lstlisting}
(* Abstract type of hole refs *)
module HoleRef : {
  type t('a)
}

(* HRG is the hole ref generation monad *)
module HRG : {
  type t('a)
  val fresh('a) : t(HoleRef.t('a))
  val bind : t('a) -> 
             ('a -> t('b)) -> 
             t('b)
  val return : 'a -> t('a)
}

(* HRE is the hole ref evaluation monad *)
module HRE : {
  type t('a) 
  type result('a) = Value('a * Exp)
                  | Indet(Exp)
  val eval('a) : HoleRef.t('a) -> 
                 t(result('a))
  val bind : t('a) -> 
             ('a -> t('b)) -> 
             t('b)
  val return : 'a -> t('a)
}
\end{lstlisting}
\caption{Modules for working with hole refs}
\label{fig:HoleRefs}
\end{figure}

\subsection{Background: Elm architecture}

\subsection{GUIs with Typed Holes}

\subsection{Palette-Specific Actions?}

\subsection{Reasoning Principles}

\subsection{Deriving Palettes from Type Definitions?}

\section{Formal System}
\begin{figure*}
        \newcommand{\absgrammar}[2]{c\,|\,#1 : #2\,|\,$x$\,|\,\hlam{$x$}{#1}\,|\,\halam{$x$}{#2}{#1}\,|\,\hehole{\mathbb{N}}\,|\,\hhole{#1}{\mathbb{N}}\,|\,#1\,#1}
\begin{grammar}
 Basic HTyps
 & $\htyp$
   & $\bnfas$ &
     b\,|\,$\tehole$\,|\,$\tarr{\htyp}{\htyp}$
%
\\[1ex]
 HTyps with cells
 & $\htypc$
   & $\bnfas$ &
     b\,|\,$\tehole$\,|\,$\tarr{\htypc}{\htypc}$
%
\\[1ex]
 HExps
 & $\hexp$
   & $\bnfas$ &
     $\absgrammar{\hexp}{\htyp}$
%
\\[1ex]
 HExps with cells
 & $\hexpc$
   & $\bnfas$ &
     $\absgrammar{\hexpc}{\htypc}$
 \\ &&& $\bnfaltbrk$ \cell{x}
%
\\[1ex]
 Palette definitions
 & $\pDef$
   & $\bnfas$ &
     $\pDefRecord{\hexpc}{\htypc}{\htyp}$
%
\\[1ex]
 Palette HExps
 & $\pexp$
   & $\bnfas$ &
     $\absgrammar{\pexp}{\htyp}$
 \\ &&& $\bnfaltbrk \pexpPalLet{p}{\pi}{\pexp}$ & Palette definition $\pi$ as $p$ in $\pexp$
 \\ &&& $\bnfaltbrk \pexpPalAp{p}{\hexpc}{x}{(\htyp, \pexp)}{C}$
                                                & $\hexpc$ (which can contain cells labeled with $x$ in $C$)
 \\ &&&                                         & is a model that is expanded by palette definition $p$
%
\end{grammar}
\begin{mathpar}
\newcommand{\icexpand}{\iota_{c_{expand}}}
\newcommand{\tcmodel}{\tau_{c_{model}}}
\newcommand{\texpansion}{\tau_{expansion}}
\newcommand{\tcexpansion}{\tau_{c_{expansion}}}
\newcommand{\icmodel}{\iota_{c_{model}}}
\newcommand{\icexpanded}{\iota_{c_{expanded}}}
\newcommand{\htypx}{\htyp_x}
\newcommand{\pexpx}{\pexp_x}
\newcommand{\hexpx}{\hexp_x}
\\\\
\inferrule[PELetPal]{
    \pi = \pDefRecord{\icexpand}{\tcmodel}{\texpansion} \\\\
    \hsyn{\EmptyhGamma}{\icexpand}{\tarr{\tcmodel}{\tcexpansion}} \\ \\
    \tconsistentc{\tcexpansion}{\texpansion} \\\\
    \pexpandAna{\hGamma}{\pPhi, p : \pi}{e}{\iota}{\tau}
  }{
    \pexpandAna{\hGamma}{\pPhi}{\pexpPalLet{p}{\pi}{e}}{\iota}{\tau}
  }
\\\\
\inferrule[PEApPal]{
    p:\pDefRecord{\icexpand}{\tcmodel}{\texpansion} \in \pPhi \\\\
    \hana{\EmptyhGamma}{\icmodel}{\tcmodel} \\ \\
    \ceval{\EmptyhGamma}{\hap{\icexpand}{\icmodel}}{\icexpanded} \\\\
    \forall x \in C, \left( \pexpandAna{\hGamma}{\pPhi}{\pexpx}{\hexpx}{\htypx} \right) \\ \\
    \celab{\EmptyhGamma}{\fmap{x}{C}{\hexpx}}{\icexpanded}{\iota}{\texpansion}
  }{
    \pexpandSyn{\hGamma}{\pPhi}{\pexpPalAp{p}{\icmodel}{x}{(\htypx, \pexpx)}{C}}{\iota}{\texpansion}
  }
\\
\end{mathpar}
\end{figure*}


\section{Related Work}
Graphite, Relit, projectional editors, notebooks / Mathematica, 

\begin{figure*}
\begin{lstlisting}
                  Graphite    Projectional Editors    TLMs     Live Palette Expressions
interactive       Y           Y                       N        Y
extensible        Y           N                       Y        Y
persistent        N           Y                       Y        Y
compositional     N           N                       Y        Y
live              N           N                       N        Y
formalized        N           N                       Y        Y
\end{lstlisting}
\caption{Related Work Table}
\label{fig:related-work}
\end{figure*}

\section{Discussion}

%\clearpage
\bibliography{references,all.short,hazel_NSF}

% \clearpage
% \appendix
% \input{implementation-appendix}

\end{document}
