\section{Implementation}\label{sec:implementation}
Implementing livelits requires tight integration between a rich editor, 
a type checker, and a live evaluator. The evaluator is used
both for running the model-view-update-expand system that controls 
each livelit and for evaluating splices. Consequently, 
the evaluator must be capable of evaluating incomplete programs and gathering hole closures
as specified by Hazelnut Live \cite{HazelnutLive}.

\subsection{Hazel}
Hazelnut Live is the formal foundation for the Hazel programming environment, so 
it was natural to choose the the Hazel programming environment for our primary implementation. 
 Hazel is a browser-based live functional programming environment 
organized from the ground up around typed holes. Uniquely, every editor state in Hazel is 
semantically meaningful: it has a type, it can be evaluated, and it can be transformed 
in a type-aware manner. This implies that livelits remain fully functional at all times, 
even when the program is incomplete or erroneous.

Hazel is implemented in OCaml and compiled to JavaScript using the \li{js_of_ocaml} compiler \cite{DBLP:conf/aplas/RadanneVB16}.
The core of Hazel is written in a pure functional style, so our implementation is able to closely 
follow the formalism. The Hazel language has been evolving in parallel with our work, 
and has only recently become expressive enough for it to be feasible to implement livelits within Hazel itself.
The examples in this paper were instead implemented in OCaml as modules satisfying the signature 
implied by Fig.~\ref{fig:color-impl}. A functor then adapts these OCaml-based livelit implementations 
to livelit definitions suitable for inclusion in the livelit context used in the semantics.

Hazel achieves its ``gap-free'' liveness guarantee by automatically inserting explicit holes as necessary 
while the user edits 
the program. Formally, Hazel is a type-aware structure editor \cite{Hazelnut}, rather than a text editor, 
although the developers are aiming to maintain a text-like programming experience when editing 
symbolic code (this effort is orthogonal to our own). 
This means that it is conceptually straightforward to hide the model and splice data and instead display 
the computed view. However, there are a number of non-trivial layout challenges that come up, which we discuss
in Sec.~\ref{sec:layout} below.

\subsection{Text Editor Integration}
Before delving into layout issues, let us note that livelits do not require the use of a structure editor. 
We have also developed a 
proof-of-concept textual program editor, based on Sketch-n-Sketch \cite{sns-pldi,sns-uist}, with support for livelits.
This proof-of-concept is not at feature parity with 
our main implementation, but it demonstrates that a syntax-recognizing text editor \cite{DBLP:journals/tosem/BallanceGV92,DBLP:conf/sde/HorganM84} 
is sufficient to support livelits, albeit with some gaps in livelit UI availability when 
there are critical errors.

\subsection{Layout}\label{sec:layout}
Whether implemented in a structure editor or a text editor, livelits present 
interesting layout challenges--it can be difficult to sensibly 
interleave symbolic code with livelit views in a clean, readable manner.

Hazel uses an optimizing pretty printer to determine layout, an approach 
based on the work of \citet{DBLP:journals/pacmpl/Bernardy17}. This system relies 
fundamentally on character counts. Consequently, our implementation asks each 
livelit to specify dimensions in terms of character counts rather than pixels.
Livelits can be laid out in one of two ways: as inline livelits, like \li{\$slider},
which must be exactly one character high and appear inline with the code,
or as \emph{multi-line livelits}, which occupy up to the full width of the editor 
and a specified number of code lines (we omitted these specifications in Fig.~\ref{fig:color-impl}). 
The pretty printer lays out multi-line livelits
on their own line, as seen in Fig.~\ref{fig:grading}. 

We are currently exploring a more flexible system whereby 
multi-line livelits can be named and applied within a larger expression, but
the GUI is not displayed until the next line, even if there are intervening characters.
This is inspired by the \verb|~| placeholder for large textual literals in the 
Wyvern programming language \cite{TSLs}. We are also considering a number of other layout 
options, e.g. pop-up livelits, livelits pinned to sidebars, and livelits that are rendered  
on a separate canvas or document while still formally being located within an underlying functional program. 

This latter option 
would be particularly interesting for end-user programming scenarios: users with limited
programming experience 
could interact with a collection of livelits laid out separately in the popular ``dashboard'' style, 
without necessarily
even being aware that their interactions are actually edits to an underlying typed
functional program. More experienced users on the team could then make programmatic use of the entered
data and parameters in a seamless manner. We are also excited about the possibility of end users
getting curious enough to ``View Source'' in this setting, and in so doing 
being exposed to typed functional programming concepts gradually.

The layout of splices within livelits is also an interesting problem, and our current approach, where 
the provider must specify a fixed size for each splice, is not particularly flexible. 
A more principled system for GUI layout that integrates with program pretty 
printing could be of substantial use in some settings. Formula bars, like the one in the \li{\$dataframe}
livelit, are a useful design pattern to avoid splice layout issues, and it may be possible to provide 
a generalized formula bar in the editor user interface that multiple livelits could share.
While splices can in principle contain other livelits, it is difficult for small splice editors to 
visually accomodate large livelit views in practice. 
Fortunately, variables can be used to support composition via substitution 
rather than nesting. A linter might suggest a refactoring to a variable in these scenarios.

The Hazel pretty printer uses memoization and incremental diffs to maintain reasonable performance,
and adding livelits to the system did not substantially interfere with the existing approach.
For the relatively simple livelits that we have developed, performance has been adequate.
However, more complex  livelits or livelits that update quickly 
could easily cause problems with editor responsiveness.
This could perhaps be resolved by performing livelit-related computations asynchronously. Care 
would need to be taken to ensure that intervening edits do not cause synchronization issues.

Many livelit-related computations can be intelligently memoized, e.g. view computations and unchanged splice evaluations,
especially with the edit awareness that Hazel provides. 
We leave further optimizations along these lines to future work.
Prior work on incrementally updating UIs after code changes could be relevant to this effort \cite{burckhardt2013s}.