\section{Discussion and Conclusion}\label{sec:discussion}\label{sec:conclusion}
\begin{quote}
  %
  \textit{
  %
  The arithmetical symbols are written diagrams and the geometrical figures are graphic formulas.
  %
  }
  
  \vspace{3pt}
  
  \hfill{}--- David Hilbert~\cite{hilbert1902mathematical}
  \end{quote}

  Diagrams have played a pivotal role in mathematical thought since antiquity,
  indeed long predating symbolic mathematics \cite{cajori1993history}. 
  Popular computing and creative tooling, too, has embraced visual representation and direct manipulation 
  interfaces for decades.
  Programming, however, has remained stubbornly mired in textual user interfaces. 
  Certainly, one must acknowledge that textual and symbolic notation
  is an indispensable tool for abstract thought. 
  Indeed, it is now widely recognized that variables and functions are at the 
  very foundation of computing. 
  Our hope with this paper is to demonstrate that principled, mathematically structured
  programming is not only compatible with live graphical user interfaces, but that the 
  combination of these two holds tremendous promise for the future of programming.

  There remain a number of avenues for future work. First, we have only just started
  to apply the livelits mechanism in various application domains. As the implementation
  matures, we plan to introduce enthusiasts in a wide variety of problem domains
  to livelits and continue the empirical evaluations started by \citet{Graphite}.

  In particular, we would like to better understand the conceptual and practical 
  difficulties that livelit providers face, with the aim of making livelit 
  implementation an extremely low cost activity. Livelits could themselves be 
  useful for this task, if we develop a GUI widget library with support for livelits.
In addition, recent approaches to deriving type-specific structure editors automatically from pretty printing logic \cite{hempeltiny} could perhaps 
be adapted to generate livelit implementations. Mechanisms for deriving simple 
livelit definitions from type definitions, perhaps similar to Haskell's \li{deriving} directive, 
may also prove helpful \cite{magalhaes2010generic}. The GEC toolkit has also explored  
the automatic generation of simple user interfaces from type definitions \cite{DBLP:conf/afp/AchtenEPW04}.

  The livelits mechanism as described in this paper operates only on expressions,
  but livelits might be useful for generating other sorts of terms, such as types
  and entire modules. Support for type and module splices, too, would likely be useful.
  With these, better support for reflecting on the provided type might allow for 
  the development of more general livelits, e.g. a variant of \li{\$datatype} 
  with individually typed columns, rather than only floating point data.

  The strict hygiene discipline that we maintain has, we believe, substantial 
  benefits---programmers will inevitably encounter unfamiliar livelits, and 
  the reasoning principles that we enforce are likely to help them ``reason around''
  the situation. However, it may be useful in certain circumstances to carefully
  relax these restrictions. For example, some livelits might benefit from the 
  ability to introduce bindings into splices, which is currently prevented by 
  the capture avoidance discipline. Perhaps this could be done in an 
  editor capable of highlighting the unusual situation.

  Another direction for future work has to do with pushing edits from computed results
  back into livelits. For example, a slider expands to a number, which may 
  then flow through a computation. Bidirectional evaluation techniques may allow
  the user to edit a number in the result of a computation and see the necessary
  change to a slider in the program \cite{sns-pldi,sns-uist}.

  Programming and authoring have much in common. Documents often contain structured
  information, and programs are written to manipulate structured information.
  Another future direction for livelits is as the basis for a programmable authoring 
  system, where each element on a page is actually an expression in a programming language
  being rendered by a livelit. Taking this further, a networked collection of these
  documents could form a powerful computational wiki. This would 
  require addressing the difficult problem of supporting collaborative interactions 
  involving edits to arbitrary user interfaces. 
  We believe that these problems are surmountable.


% \begin{itemize}
%   \item pattern matching
%   \item type splices
%   \item module splices
%   \item explicitly make bindings available in splices
%   \item better UI for closure provenance / connect with control flow / call stack better
%   \item deriving livelits from type definitions
%   \item bidirectional evaluation ala SnS
%   \item integration with structure editing more cleanly
%   \item \dots
%   \item collaborative editing?
%   \item side effects?
%   \item full screening livelits (or compositions thereof) as a way to create end-user workflows
%   \item big vision: authoring environment based on livelits
% \end{itemize}
