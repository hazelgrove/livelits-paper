\section{Discussion and Conclusion}\label{sec:discussion}\label{sec:conclusion}
\begin{quote}
  %
  \textit{
  %
  ``The arithmetical symbols are written diagrams and the geometrical figures are graphic formulas.''
  %
  }
  
  \vspace{3pt}
  
  \hfill{}--- David Hilbert~\cite{XXX}
  \end{quote}

Future work:
Recent approaches to deriving type-specific structure editors automatically from pretty printing logic \cite{hempeltiny} could perhaps 
be adapted to generate livelit implementations. Mechanisms for deriving simple 
livelit definitions from type definitions, perhaps similar to Haskell's \li{deriving} directive, 
may also prove helpful \cite{magalhaes2010generic}. The GEC toolkit has explored  
the automatic generation of simple user interfaces from type definitions \cite{DBLP:conf/afp/AchtenEPW04}.


\begin{itemize}
  \item pattern matching
  \item type splices
  \item module splices
  \item explicitly make bindings available in splices
  \item better UI for closure provenance / connect with control flow / call stack better
  \item deriving livelits from type definitions
  \item bidirectional evaluation ala SnS
  \item integration with structure editing more cleanly
  \item \dots
  \item collaborative editing?
  \item side effects?
  \item full screening livelits (or compositions thereof) as a way to create end-user workflows
  \item big vision: authoring environment based on livelits
\end{itemize}
